% This file contains the content for a main section
\regularsectionformat	% Change formatting to that of a main, numbered section
%% Modify below this line %%
\chapter{Use Cases}
ACES Metadata Files (AMFs) are intended to contain the minimum required metadata for transferring information about ACES viewing pipelines during production, post-production, and archival.

Typical use cases for AMF files are the application of ``show LUT'' LMTs in cameras and on-set systems, the capture of shot-to-shot looks generated on-set using ASC-CDL, and communication of both to dailies, editorial, VFX, and post-production mastering facilities.

AMF supports the transfer of looks by embedding ASC-CDL values within the AMF file or by referencing sidecar look files containing LMTs, such as CLF (Common LUT Format) files.  

\section{Look Development}
The development of a creative look before the commencement of production is common. Production uses this look to produce a pre-adjusted reference for on-set monitoring. The creative look may be a package of files containing a viewing transform (also known as a “Show LUT”), CDL grades, or more. There are no consistent standards specifying how to produce them, and exchanging them is complex due to a lack of metadata.

AMF contains the ability to completely specify the application of a creative look. This automates the exchange of these files and the recreation of the look when applying the AMF. In an ACES workflow, one specifies the creative look as one or more Look Modification Transforms (LMT). AMF can include references to any number of these transforms, and maintains their order of operations.

The input and output of an LMT is always a triplet of ACES RGB relative exposure values, as defined in SMPTE ST 2065-1. This will likely need a robust transform, such as CLF, that can handle linear input data and output data.

AMF offers an unambiguous description of the full ACES viewing pipeline for on-set look management software to load and display images as intended.

\section{On Set}
Before production begins, an AMF may be created and shared with production as a ``look template'' for use during on-set monitoring or look management.

Cameras with AMF support can load or generate AMFs to configure or communicate the viewing pipeline of images viewed out of the camera's live video signal.

On-set color grading software can load or generate AMFs, allowing the communication of the color adjustments created on set.

\section{Dailies}
Dailies can apply AMFs from production to the camera files to reproduce the same images seen on set. There is no single method of exchange between production and dailies. AMFs should be agnostic to the given exchange method.

It is possible, or even likely, that one will update AMFs in the dailies stage. For example, a dailies colorist may choose to balance shots at this stage and update the look. Another example could be that dailies uses a different ODT than the one used in on-set monitoring.

This specification does not define how one should transport AMFs between stages. Existing exchange formats may reference them, or image files themselves may embed them. One may also transport AMFs independently of any other files.

\section{VFX}
The exchange of shots for VFX work requires perfect translation of each shot’s viewing pipeline, or ‘color recipe’. If the images cannot be accurately reproduced from VFX plates, effects will be created with an incorrect reference.

AMF provides a complete and unambiguous translation of ACES viewing pipelines. If they travel with VFX plates, they can describe how to view each plate along with any associated looks.

VFX software should have the ability to read AMF to configure its internal viewing pipeline. Or, AMF will inform the configuration of third party color management software, such as OpenColorIO.

\section{Finishing}
In finishing, the on-set or dailies viewing reference can be automatically recreated upon reading an AMF. This stage typically uses a higher quality display, which may warrant the use of a different ODT than one specified in an ingested AMF.

AMF can seamlessly provide the colorist a starting point that is consistent with the creative intent of the filmmakers on-set. This removes any necessity to recreate a starting look from scratch.

\section{Archival}
AMF enables the ability to establish a complete ACES archive, and effectively serves as a snapshot of creative intent for preservation and remastering purposes. All components required to recreate the look of an ACES archive are meaningfully described and preserved within the AMF.

One possible method for this could be the utilization of SMPTE standards such as ST.2067-50 (IMF App \#5) -- commonly referred to as ``ACES IMF'' -- and SMPTE RDD 47 (ISXD) -- a virtual track file containing XML data -- in order to form a complete and flexible ACES archival package.

Another method could be to use SMPTE ST.2067-9 (Sidecar Composition Map) which would allow linking of a single AMF to a CPL (Composition Playlist) in the case where there is a single AMF for an entire playlist.
