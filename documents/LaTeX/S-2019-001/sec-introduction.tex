% This file contains the content for a main section
\regularsectionformat	% Change formatting to that of a main, numbered section
%% Modify below this line %%
\chapter{Introduction}

\section{Why is metadata needed for ACES?}

ACES defines a standard color encoding (SMPTE ST 2065-1) for exchange of images, along with Input Transforms to convert from different image sources to ACES, and Output Transforms in order to view ACES images on different types of displays.

However, when exchanging ACES images during production, there is often missing information required to fully describe the viewing pipeline or ``creative intent'' of that particular image.

Examples of such information:
\begin{itemize}
    \item ACES Version -- \textit{which version of ACES was used?}
    \item Look Transform -- \textit{is there a creative look (CDL or LUT?)}
    \item Output Transform -- \textit{on what display was this viewed?}
\end{itemize}

For production, carriage of this information is crucial in order to unambiguously exchange ACES images, looks, and overall creative intent through its various stages (e.g. on-set, dailies, editorial, VFX, finishing).

For archival, storage of this information is crucial in order to form a record of creative intent for historical and remastering purposes.

\section{What is AMF?}
The ACES Metadata File (``AMF'') is a sidecar XML file intended to exchange the metadata required to recreate ACES viewing pipelines. It describes the transforms necessary to configure an ACES viewing pipeline for a collection of related image files.

An AMF may be associated with a single frame or clip. Additionally, it may remain unassociated with an image, and existing purely as a translation of an ACES viewing pipeline, or creative look, that could be applied to any image.

Images are formed at several stages of production and post-production, including:

\begin{itemize}
    \item Digital cameras
    \item Film scanners 
    \item Animation and VFX production
    \item Virtual production
    \item Editorial and color correction systems
\end{itemize}

AMF can be compatible with any digital image, and is not restricted to those encoded in the ACES (SMPTE ST 2065-1). They may be camera native file formats or other encodings if they have associated Input Device Transforms (IDTs) (using the \texttt{\textbf{<inputTransform>}} element) so they may be displayed using an ACES viewing pipeline.

AMFs may also embed creative look adjustments as one or more LMTs (using the \texttt{\textbf{<lookTransform>}} elements). These looks may be in the form of ASC CDL values, or a reference to an external look file or LUT.  Multiple \texttt{<lookTransform>} elements are allowed, and the order of operations in which they are applied shall be the order in which they appear in the AMF.

AMFs can also serve as effective archival elements. When paired with finished ACES image files, they form a complete archival record of how image content is intended to be viewed (for example, using the \texttt{\textbf{<outputTransform>}} and \texttt{\textbf{<systemVersion>}} elements).

AMFs do not contain ``timeline'' metadata such as edit points. Timeline management files such as Edit Decision Lists (EDLs) or Avid Log Exchange files (ALEs) may reference AMFs, attaching them to editing events and thus enable standardized color management throughout all stages of production.

Figure \ref{fig:overviewDiagram} shows the overall structure of an AMF in simplified form.

\begin{figure}[h]
    \includegraphics[width=\textwidth]{./images/amfDiagram.pdf}
    \caption{A simplified diagram showing the overall structure of an ACES Metadata File.}
    \label{fig:overviewDiagram}
    \centering
\end{figure}

