\appendixchapter{Example empirical LMT supporting PFE}{i}
\label{appendixB}

This example shows how a 3DLUT can be used to implement an LMT that emulates the look of projected film print.

\vspace{18pt}

\begin{lstlisting}
<ProcessList xmlns="urn:NATAS:ASC:LUT:1.1" id="ProdLMTv1" name="WarmNight_v1">
<Description> Create a warm nightime look </Description>
<InputDescriptor>ACES</InputDescriptor>
<OutputDescriptor>ACES</OutputDescriptor>
<LUT1D id="uuid:f81d4fae-7dec-11d0-a765-00a0c91e6bf8" name="ConverttoACESccToneScale" 
    inBitDepth="16f" outBitDepth="16f">
	<Description> Shaper LUT for interpolation into 3DLUT uses positive 
	    value ACEScc spacing. </Description>
	<Array dim="65535 3">
		/* OMIT */
	</Array>
</LUT1D>
<LUT3D id=" uuid:f81d4fae-7dec-11d0-a765-00a0c91e6bf2" name="PFE_Vision2_12Nov2003" 
    interpolation="tetrahedral" inBitDepth="16f" outBitDepth="16f" >
	<Description> 3D LUT calculated in ACEScc outputs ACES values </Description>
	<Array dim="64 64 64 3">
			0.0 0.0 0.0
			/* OMIT */
			65504.0 65504.0 65504.0
	</Array>
</LUT3D>
</ProcessList>
\end{lstlisting}