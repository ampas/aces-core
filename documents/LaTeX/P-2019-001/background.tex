% This file contains the content for a main section
\numberedformat
%% Modify below this line %%
\chapter{Background}

The Academy Color Encoding System is a free, open-source, device-independent color management and image interchange system designed for production, mastering and long-term archiving of motion pictures. ACES has also experienced adoption in television, gaming, and virtual/augmented/mixed reality applications.

ACES is a project of the Academy's Science and Technology Council and was launched in 2004. Since inception, the project has been operated under the Academy’s rules for committee-driven activities with extensive consideration to industry participation in development, implementation and end-user adoption. A primary goal of the ACESnext initiative is to further enable community participation in and responsibility for continued enhancement of ACES capabilities and its industry-wide adoption. With the recent launch of the Academy Software Foundation (\href{http://aswf.io}{http://aswf.io}), the Linux Foundation's open source software model for community-driven development work has been introduced to our industry, and ACES Leadership believes the ACES project can benefit from the collaborative practices intrinsic to this model.

It is with these practices  in mind that the Academy is defining a new project organization structure for ACES, described in this document. This structure is intended to provide more ACES project oversight, development, test and implementation responsibility by the community of engineers, end-users and other stakeholders who rely on the system.