\newpage
\section{ACES Architecture Technical Advisory Council}
The ACES Architecture Technical Advisory Council (Architecture TAC) is responsible for the peer review and acceptance of Working Group deliverables that impact or are impacted by ACES architecture. The Architecture TAC ensures Working Group deliverables are consistent with ACES architecture strategy and goals. The Architecture TAC is also responsible for ensuring Working Group activities are within their approved scope and that timelines are met. The Architecture TAC does not engineer solutions to particular problems but serves as a body to provide guidance 
and expert advice to the Working Groups it oversees. As such the TAC will be composed of both subject matter experts and technologists with a broad understanding of motion picture workflows and the impact of those workflows on their organization's objectives.  TAC members should be invested in the success of ACES and have deep working knowledge of how ACES would strategically benefit the companies using it and the entertainment industry as a whole.  Architecture TAC membership is by invitation of the TAC chair consistent with the parameters described below. Architecture TAC recommendations requiring a vote are approved by a simple majority of voting members. TAC members may not always agree on specific decisions but should seek to reach consensus that is in the best interest the ACES community. TAC recommendations are forwarded to ACES Leadership for ratification.  Membership on the TAC is for 1 year, renewable for up to 2 additional years.

\subsection{Leadership and Voting Members}
The Architecture TAC should have between 12--18 members.  The voting membership of the Architecture TAC should be composed as follows:

\begin{itemize}
    \item The ACES Architecture TAC Chair, who is one of the ACES Project Vice Chairs
    \item Motion picture studio or production company representatives comprising approximately 30--40\% of the TAC.
    \item Motion picture production subject matter experts in fields such as post-production, visual effects, cinematography, editorial, and archiving comprising approximately 30--40\% of the TAC with the goal of having one TAC representative having subject matter expertise in each of the named fields.
    \item Game development studio representatives comprising approximately 10--20\% of the TAC.
    \item Color science subject matter experts comprising approximately 10--20\% of the TAC
\end{itemize}

No more than 2 members of the TAC should be employed by the same company.

\subsection{Scope and Responsibilities}
The scope and responsibilities of the ACES Architecture TAC shall include:

\begin{itemize}
    \item ACES architecture definition revisions
    \item Core ACES transforms, encodings, color space details, etc.
    \item ACES metadata definitions and philosophy
    \item ACES file formats (ADSM, ACES/ADX Container Files, CLF, AMF, etc.)
    \item Other ACES architecture related-topics as needed and agreed to by ACES Leadership
\end{itemize}

\subsection{Meetings}
Meetings of the Architecture TAC will occur quarterly and will be open to the public. Meeting notifications will be posted to ACEScentral.com in advance. Non-voting meeting attendees may participate in the meeting discussions and provide relevant commentary at the discretion of the Architecture TAC Chair. Efforts will be made to record the meetings for later review. 