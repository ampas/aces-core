\appendixchapter{Interpolation}{n}
\label{appendix:interpolation}

When an input value falls between sampled positions in a LUT, the output value must be calculated as a proportion of the distance along some function that connects the nearest surrounding values in the LUT. There are many different types of interpolation possible, but only three types of interpolation are currently specified for use with the Common LUT Format (CLF). 

The first type -- linear interpolation -- is specified for use with a \xml{LUT1D} node. The other two -- trilinear and tetrahedral interpolation -- are specified for use with a \xml{LUT3D} node.

\section{Linear Interpolation}
With a table of the sampled input values in $inValue[i]$ where $i$ ranges from $0$ to $(n-1)$, and a table of the corresponding output values in $outValue[j]$ where $j$ is equal to $i$,

\begin{center}
\begin{tabularx}{3in}{ccXcc}
    index $i$ & inValue && index $j$ & outValue \\ \hline
    0 & 0 && 0 & 0 \\
    $\vdots$ & $\vdots$ && $\vdots$ & $\vdots$ \\
    $n-1$ & 1 && $n-1$ & 1000 \\
\end{tabularx}
\end{center}

the $output$ resulting from $input$ can be calculated after finding the nearest $inValue[i] < input$. 

When $inValue[i] = input$, the result is evaluated directly.

\begin{center}
$output = \dfrac{input-inValue[i]}{inValue[i+1]-inValue[i]} \times (outValue[j+1]-outValue[j])+outValue[j]$ 
\end{center}

\section{Trilinear Interpolation}
Trilinear interpolation implements linear interpolation in three-dimensions by successively interpolating each direction. 

\begin{figure}[htbp]
\begin{center}
    \includegraphics[width=2in]{images/interp-pointInMesh.png}
    \includegraphics[width=2in]{images/interp-pointInCubelet.png}
\caption{Illustration of a sampled point located within a basic 3D LUT mesh grid (left) and the same point but with only the vertices surrounding the sampled point (right).}
\label{fig:lut-sampling}
\end{center}
\end{figure}

\begin{figure}[htbp]
\begin{center}
    \includegraphics[width=3.5in]{images/interp-labeledCubelet.png}
\caption{Labeling the mesh points surrounding the sampled point $(r,g,b)$.}
\label{fig:mesh-points}
\end{center}
\end{figure}

\note{The convention used for notation is uppercase variables for mesh points and lowercase variables for points on the grid.}

Consider a sampled point as depicted in Figure \ref{fig:mesh-points}. Let $V(r,g,b)$ represent the value at the point with coordinate $(r,g,b)$. The distance between each node per color coordinate shows the proportion of each mesh point's color coordinate values that contribute to the sampled point.

\begin{equation}
    \Delta_r = \frac{r - R_0}{R_1 - R_0} \hspace{0.25in}
    \Delta_g = \frac{g - G_0}{G_1 - G_0} \hspace{0.25in}
    \Delta_b = \frac{b - B_0}{B_1 - B_0}
\end{equation}

The general expression for trilinear interpolation can be expressed as:

\begin{equation}
    V(r,g,b) = c_0 + c_1\Delta_b + c_2\Delta_r + c_3\Delta_g + c_4\Delta_b\Delta_r + c_5\Delta_r\Delta_g + c_6\Delta_g\Delta_b + c_7\Delta_r\Delta_g\Delta_b
\end{equation}
\tabto{0.5in}where: \\
\begin{equation*}
\begin{aligned}
    c_0 &= V(R_0, G_0, B_0) \\
    c_1 &= V(R_0, G_0, B_1) - V(R_0, G_0, B_0) \\
    c_2 &= V(1_0, G_0, B_0) - V(R_0, G_0, B_0) \\
    c_3 &= V(R_0, G_1, B_0) - V(R_0, G_0, B_0) \\
    c_4 &= V(R_1, G_1, B_1) - V(R_1, G_0, B_0) - V(R_0, G_0, B_1) + V(R_0, G_0, B_0) \\
    c_5 &= V(R_1, G_1, B_0) - V(R_0, G_1, B_0) - V(R_1, G_0, B_0) + V(R_0, G_0, B_0) \\
    c_6 &= V(R_0, G_1, B_1) - V(R_1, G_1, B_0) - V(R_0, G_0, B_1) + V(R_0, G_0, B_0) \\
    c_7 &= V(R_1, G_1, B_1) - V(R_1, G_1, B_0) - V(R_0, G_1, B_1) - V(R_1, G_0, B_1) \\ 
        &+ V(R_0, G_0, B_1) + V(R_0, G_1, B_0) + V(R_1, G_0, B_0) - V(R_0, G_0, B_0) 
\end{aligned}    
\end{equation*}

Expressed in matrix form:
\begin{gather}
    \matr{C} = [c_0 \enspace c_1 \enspace c_2 \enspace c_3 \enspace c_4 \enspace c_5 \enspace c_6 \enspace c_7]^T \\
    \matr{\Delta} = [1 \quad \Delta_b \quad \Delta_r \quad \Delta_g \quad \Delta_b\Delta_r \quad \Delta_r\Delta_g \quad \Delta_g\Delta_b \quad \Delta_r\Delta_g\Delta_b]^T \\
    V(r,g,b) = \matr{C}^T \matr{\Delta}
\end{gather}

\begin{equation} \label{eq:linInterpMatrixForm}
    \begin{bmatrix}
    c_0 \\ c_1 \\ c_2 \\ c_3 \\ c_4 \\ c_5 \\ c_6 \\ c_7
    \end{bmatrix} =
    \begin{bmatrix}
    1 & 0 & 0 & 0 & 0 & 0 & 0 & 0 \\
    -1 & 0 & 0 & 0 & 1 & 0 & 0 & 0 \\
    -1 & 0 & 1 & 0 & 0 & 0 & 0 & 0 \\
    -1 & 1 & 0 & 0 & 0 & 0 & 0 & 0 \\
    1 & 0 & -1 & 0 & -1 & 0 & 1 & 0 \\
    1 & -1 & -1 & 1 & 0 & 0 & 0 & 0 \\
    1 & -1 & 0 & 0 & -1 & 1 & 0 & 0 \\
    -1 & 1 & 1 & -1 & 1 & -1 & -1 & 1
    \end{bmatrix}
    \begin{bmatrix}
    V(R_0, G_0, B_0) \\
    V(R_0, G_1, B_0) \\
    V(R_1, G_0, B_0) \\
    V(R_1, G_1, B_0) \\
    V(R_0, G_0, B_1) \\
    V(R_0, G_1, B_1) \\
    V(R_1, G_0, B_1) \\
    V(R_1, G_1, B_1)
    \end{bmatrix}    
\end{equation}

The expression in Equation \ref{eq:linInterpMatrixForm} can be written as: $\matr{C} = \matr{A}\matr{V}$.

Trilinear interpolation shall be done according to $V(r,g,b) = \matr{C}^T \matr{\Delta} = \matr{V}^T \matr{A}^T \matr{\Delta}$.

\note{ The term $\matr{V}^T \matr{A}^T$ does not depend on the variable $(r,g,b)$ and thus can be computed in advance for optimization. Each sub-cube can have the values of the vector $\matr{C}$ already stored in memory. Therefore the algorithm can be summarized as: 
\begin{enumerate}
    \item Find the sub-cube containing the point \((r,g,b)\)
    \item Select the vector \(\matr{C}\) corresponding to that sub-cube
    \item Compute \(\Delta_r\), \(\Delta_g\), \(\Delta_b\)
    \item Return \(V(r,g,b) = \matr{C}^T \matr{\Delta}\)
\end{enumerate}
}

\section{Tetrahedral Interpolation}
Tetrahedral interpolation subdivides the cubelet defined by the vertices surrounding a sampled point into six tetrahedra by segmenting along the main (and usually neutral) diagonal (Figure \ref{fig:tetrahedra}). 

\begin{figure}[htbp]
\begin{center}
    \includegraphics[width=6in]{images/interp-tetrahedrons.png}
\caption{Illustration of the six subdivided tetrahedra.}
\label{fig:tetrahedra}
\end{center}
\end{figure}

To find the tetrahedron containing the point \((r,g,b)\):
\begin{itemize}
    \item if \(\Delta_b > \Delta_r > \Delta_g\), then use the first tetrahedron, \(t1\)
    \item if \(\Delta_b > \Delta_g > \Delta_r\), then use the first tetrahedron, \(t2\)
    \item if \(\Delta_g > \Delta_b > \Delta_r\), then use the first tetrahedron, \(t3\)
    \item if \(\Delta_r > \Delta_b > \Delta_g\), then use the first tetrahedron, \(t4\)
    \item if \(\Delta_r > \Delta_g > \Delta_b\), then use the first tetrahedron, \(t5\)
    \item else, use the sixth tetrahedron, \(t6\)
\end{itemize}

The matrix notation is:

\begin{equation}
    \matr{V} = \begin{bmatrix}
    V(R_0, G_0, B_0) \\
    V(R_0, G_1, B_0) \\
    V(R_1, G_0, B_0) \\
    V(R_1, G_1, B_0) \\
    V(R_0, G_0, B_1) \\
    V(R_0, G_1, B_1) \\
    V(R_1, G_0, B_1) \\
    V(R_1, G_1, B_1)
    \end{bmatrix}\\ \\
\end{equation}
\begin{equation}
    \matr{\Delta_t} = [1 \enspace \Delta_b \enspace \Delta_r \enspace \Delta_g]^T
\end{equation}

\begin{equation}
\begin{aligned} \label{eq:tetrahedral-matrices}
    \matr{T}_1 = \begin{bmatrix}
    1 & 0 & 0 & 0 & 0 & 0 & 0 & 0 \\
    -1 & 0 & 0 & 0 & 1 & 0 & 0 & 0 \\
    0 & 0 & 0 & 0 & -1 & 0 & 1 & 0 \\
    0 & 0 & 0 & 0 & 0 & 0 & -1 & 1
    \end{bmatrix} \hspace{0.5in}
    \matr{T}_2 = \begin{bmatrix}
    1 & 0 & 0 & 0 & 0 & 0 & 0 & 0 \\
    -1 & 0 & 0 & 0 & 1 & 0 & 0 & 0 \\
    0 & 0 & 0 & 0 & 0 & -1 & 0 & 1 \\
    0 & 0 & 0 & 0 & -1 & 1 & 0 & 0
    \end{bmatrix} \\
    \matr{T}_3 = \begin{bmatrix}
    1 & 0 & 0 & 0 & 0 & 0 & 0 & 0 \\
    0 & -1 & 0 & 0 & 0 & 1 & 0 & 0 \\
    0 & 0 & 0 & 0 & 0 & -1 & 0 & 1 \\
    -1 & 1 & 0 & 0 & 0 & 0 & 0 & 0
    \end{bmatrix} \hspace{0.5in}
    \matr{T}_4 = \begin{bmatrix}
    1 & 0 & 0 & 0 & 0 & 0 & 0 & 0 \\
    0 & 0 & -1 & 0 & 0 & 0 & 1 & 0 \\
    -1 & 0 & 1 & 0 & 0 & 0 & 0 & 0 \\
    0 & 0 & 0 & 0 & 0 & 0 & -1 & 1
    \end{bmatrix} \\
    \matr{T}_5 = \begin{bmatrix}
    1 & 0 & 0 & 0 & 0 & 0 & 0 & 0 \\
    0 & 0 & 0 & -1 & 0 & 0 & 0 & 1 \\
    -1 & 0 & 1 & 0 & 0 & 0 & 0 & 0 \\
    0 & 0 & -1 & 1 & 0 & 0 & 0 & 0
    \end{bmatrix} \hspace{0.5in}
    \matr{T}_6 = \begin{bmatrix}
    1 & 0 & 0 & 0 & 0 & 0 & 0 & 0 \\
    0 & 0 & 0 & -1 & 0 & 0 & 0 & 1 \\
    0 & -1 & 0 & 1 & 0 & 0 & 0 & 0 \\
    -1 & 1 & 0 & 0 & 0 & 0 & 0 & 0
    \end{bmatrix}
\end{aligned}
\end{equation}

Trilinear interpolation shall be done according to:
\begin{gather}
    V(r,g,b)_{t1} = \matr{\Delta}^T_t \matr{T}_1 \matr{V}\\
    V(r,g,b)_{t2} = \matr{\Delta}^T_t \matr{T}_2 \matr{V}\\
    V(r,g,b)_{t3} = \matr{\Delta}^T_t \matr{T}_3 \matr{V}\\
    V(r,g,b)_{t4} = \matr{\Delta}^T_t \matr{T}_4 \matr{V}\\
    V(r,g,b)_{t5} = \matr{\Delta}^T_t \matr{T}_5 \matr{V}\\
    V(r,g,b)_{t6} = \matr{\Delta}^T_t \matr{T}_6 \matr{V}
\end{gather} 

\note{ The vectors \(\matr{T}_i \matr{V}\) for \(i = 1,2,3,4,5,6\) does not depend on the variable \((r,g,b)\) and thus can be computed in advance for optimization.}
