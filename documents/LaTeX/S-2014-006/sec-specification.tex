% This file contains the content for a main section
\regularsectionformat
%% Modify below this line %%
\chapter{Specification}
\label{sec:specification}

\section{General}
A Common LUT Format (CLF) file shall be written using Extensible Markup Language (XML) and adhere to a defined XML structure. A CLF file shall have the file extension `.clf'. 

The top level element in a CLF file defines a \xml{ProcessList} which represents a sequential set of color transformations. The result of each individual color transformation feeds into the next transform in the list to create a daisy chain of transforms.

An application reads a CLF file and initializes a transform engine to perform the operations in the list. The transform engine reads as input a stream of code values of pixels, performs the calculations and/or interpolations, and writes an output stream representing a new set of code values for the pixels.  

In the sequence of transformations described by a \xml{ProcessList}, each \xml{ProcessNode} performs a transform on a stream of pixel data, and only one input line (input pixel values) may enter a node and only one output line (output pixel values) may exit a node. A \xml{ProcessList} may be defined to work on either 1-component or 3-component pixel data, however all transforms in the list must be appropriate, especially in the 1-component case (black-and-white) where only 1D LUT operations are allowed. Implementation may process 1-component transforms by applying the same processing to R, G, and B.
 
\begin{figure}[htbp]
\begin{center}
    \includegraphics[width=3.5in]{processList.png}
\caption{Example of a Process List containing a sequence of multiple Process Nodes}
\label{fig:lmt}
\end{center}
\end{figure}

The file format does not provide a mechanism to assign color transforms to either image sequences or image regions. However, the XML structure defining the LUT transform, a \texttt{ProcessList}, may be encapsulated in a larger XML structure potentially providing that mechanism. This mechanism is beyond the scope of this document.

Each CLF file shall be completely self-contained requiring no external information or metadata. The full content of a color transform must be included in each file and a color transform may not be incorporated by reference to another CLF file. This restriction ensures that each CLF file can be an independent archival element. 

Each \xml{ProcessList} shall be given a unique ID for reference. 

The data for LUTs shall be an ordered array that is either all floats or all integers. When three RGB color components are present, it is assumed that these are red, green, and blue in that order. There is only one order for how the data array elements are specified in a LUT, which is in general from black to white (from the minimum input value position to the maximum input value position). Arbitrary ordering of list elements is not supported in the format (see \autoref{sec:XMLelements} for details).

\note{For 3D LUTs, the indexes to the cube are assumed to have regular spacing across the range of input values. To accommodate irregular spacing, a \xml{"halfDomain"} 1D LUT or Log node should be used as a shaper function prior to the 3D LUT.}


\section{XML Structure}
\subsection{General}
A CLF file shall contain a single occurrence of the XML root element known as the \xml{ProcessList}. 
The \xml{ProcessList} element shall contain one or more elements known as \xml{ProcessNode}s. The order and number of process nodes is determined by the designer of the CLF file.  

An example of the overall structure of a simple CLF file is thus:

\lstset{frame=none}
\begin{lstlisting}
<ProcessList id="123">
	<Matrix id="1">
		data & metadata
	</Matrix>
	<LUT1D id="2">
 		data & metadata
	</LUT1D>
	<Matrix id="3">
		data & metadata
	</Matrix>
</ProcessList>	
\end{lstlisting}


\subsection{XML Version and Encoding}
A CLF file shall include a starting line that declares XML version number and character encoding. This line is mandatory once in a file and looks like this:

\lstset{frame=none}
\begin{lstlisting}
<?xml version="1.0" encoding="UTF-8"?>
\end{lstlisting}

\subsection{Comments}
The file may also contain XML comments that may be used to describe the structure of the file or save information that would not normally be exposed to a database or to a user. XML comments are enclosed in brackets like so:

\begin{lstlisting}
<!--   This is a comment    -->
\end{lstlisting}
\lstset{frame=single}

\subsection{Language}
It is often useful to identify the natural or formal language in which text strings of XML documents are written.  The special attribute named xml:lang may be inserted in XML documents to specify the language used in the contents and attribute values of any element in an XML document. The values of the attribute are language identifiers as defined by IETF RFC 3066. In addition, the empty string may be specified.

The language specified by xml:lang applies to the element where it is specified (including the values of its attributes), and to all elements in its content unless overridden with another instance of xml:lang. In particular, the empty value of xml:lang can be used to override a specification of xml:lang on an enclosing element, without specifying another language.

\begin{figure}[htbp]
\begin{center}
    \includegraphics[width=\textwidth]{images/clf-diagram.pdf}
\caption{Object Model of XML Elements}
\label{fig:object-model}
\end{center}
\end{figure}

\subsection{White Space}
Particularly when creating CLF files containing certain elements (such as \xml{Array}, \xml{LUT1D}, or \xml{LUT3D}) it is desirable that single lines per entry are maintained so that file contents can be scanned more easily by a human reader. There exist some difficulties with maintaining this behavior as XML has some non-specific methods for handling white-space. Especially if files are re-written from an XML parser, white space will not necessarily be maintained. To maintain line layout, XML style sheets may be used for reviewing and checking the CLF file's entries.

\subsection{Newline Control Characters}
Different end of line conventions, including \textless{}CR\textgreater{}, \textless{}LF\textgreater{}, and \textless{}CRLF\textgreater{}, are utilized between Mac, Unix, and Windows systems. Different newline characters may result in the collapse of values into one long line of text. To maintain intended linebreaks, CLF specifies that the `newline' string, i.e. the byte(s) to be interpreted as ending each line of text, shall be the single code value $10_{10} = 0\mathrm{A}_{16}$ (ASCII `Line Feed' character), also indicated \textless{}LF\textgreater{}. 

\note{Parsers of CLF files may choose to interpret Microsoft's \textless{}CR\textgreater{}\textless{}LF\textgreater{} or older-MacOS' \textless{}CR\textgreater{} newline conventions, but CLF files should only be generated with the \textless{}LF\textgreater{} encoding.}

\note{\textless{}LF\textgreater{} is the newline convention native to all *nix operating systems (including Linux and modern macOS).}