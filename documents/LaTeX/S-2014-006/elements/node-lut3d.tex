\subsection{\texttt{LUT3D}}

\emph{Description:} \par
This node shall contain a 3D LUT in the form of an \xml{Array}. In a \xml{LUT3D}, the 3 color components of the input value are used to find the nearest indexed values along each axis of the 3D cube. The 3-component output value is calculated by interpolating within the volume defined by the nearest corresponding positions in the LUT. 

\xml{LUT3D}s have the same dimension on all axes (i.e. \xml{Array} dimensions are of the form ``n n n 3"). A \xml{LUT3D} with axial dimensions greater than 128x128x128 should be avoided.

The scaling of the array values is based on the \xml{outBitDepth} (the \xml{inBitDepth} is not considered).


\emph{Attributes:}
\begin{xmlfields}
    \xmlitem[interpolation][optional] a string indicating the preferred algorithm used to interpolate values in the \xml{3DLUT}. This attribute is optional with a default of \xml{"trilinear"} if the attribute is not present. \par   Supported values are: 
        \begin{itemize}
            \item[-] \xml{"trilinear"}: perform trilinear interpolation
            
            \item[-] \xml{"tetrahedral"}: perform tetrahedral interpolation
        \end{itemize}
    \note{Interpolation methods are specified in Appendix \ref{appendix:interpolation}.}
\end{xmlfields}

\emph{Elements:}
\begin{xmlfields}
    \xmlitem[Array][required] 
    an array of numeric values that are the output values of the 3D LUT.  The \xml{Array} shall contain the table entries for the \xml{LUT3D} from the minimum to the maximum input values, with the third component index changing fastest.
    
    \emph{Attributes:}
    \begin{xmlfields}
        \xmlitem[dim][required] 
        four integers that reperesent the dimensions of the 3D LUT and the number of color components. The first three values define the dimensions of the LUT and if multiplied shall equal the number of entries actually present in the array. The fourth value indicates the number of components per entry.
        
        4 entries have the dimensions of a 3D cube plus the number of components per entry.
        \begin{list}{}{\setlength{\itemsep}{4pt}\setlength{\topsep}{0pt}}
                \item e.g. \xml{dim = "17 17 17 3"} indicates a 17-cubed 3D lookup table with 3 component color
        \end{list}
        
    \end{xmlfields}
    
    \note{\xml{Array} is formatted differently when it is contained in a \xml{LUT1D} or \xml{Matrix} element (see \ref{sec:array}).}

\end{xmlfields}
        
\emph{Examples:}
\begin{lstlisting}[caption=Example of a simple \xml{LUT3D},label=ex:3dlut]
<LUT3D id="lut-24" name="green look" interpolation="trilinear" 
    inBitDepth="12i" outBitDepth="16f">
    <Description>3D LUT</Description>
    <Array dim="2 2 2 3">
        0.0 0.0 0.0
        0.0 0.0 1.0
        0.0 1.0 0.0
        0.0 1.0 1.0
        1.0 0.0 0.0
        1.0 0.0 1.0
        1.0 1.0 0.0
        1.0 1.0 1.0
    </Array>
</LUT3D>
\end{lstlisting}

