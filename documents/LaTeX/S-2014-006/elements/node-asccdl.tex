\subsection{\texttt{ASC\_CDL}}

\emph{Description:} \par
This node processes values according to the American Society of Cinematographers' Color Decision List (ASC CDL) equations. Color correction using ASC CDL is an industry-wide method of recording and exchanging basic color correction adjustments via parameters that set particular color processing equations.

The ASC CDL equations are designed to work on an input domain of floating-point values of [0 to 1.0] although values greater than 1.0 can be present. The output data may or may not be clamped depending on the processing style used.

\note{Equations \ref{eq:cdl-fwd}-\ref{eq:cdl-rev} assume that \var{in} and \var{out} are scaled to normalized floating-point range. If the \xml{ASC\_CDL} node has \xml{inBitDepth} or \xml{outBitDepth} that are integer types, then the input or output values must be normalized to or from 0-1 scaling. In other words, the slope, offset, power, and saturation values stored in the \xml{ProcessNode} do not depend on \xml{inBitDepth} and \xml{outBitDepth}; they are always interpreted as if the bit depths were float.}

\emph{Attributes:}
\begin{xmlfields}
	\xmlitem[id][optional] This should match the id attribute of the ColorCorrection element in the ASC CDL XML format.
	\xmlitem[style][optional] Determines the formula applied by the operator. The valid options are:
		\begin{xmlfields}
			\xmlitemd["Fwd"] implementation of v1.2 ASC CDL equation. \\
			This is the default value for \xml{style}.
			\xmlitemd["Rev"] inverse equation
			\xmlitemd["FwdNoClamp"] similar to the \xml{Fwd} equation, but without clamping
			\xmlitemd["RevNoClamp"] inverse equation, without clamping
		\end{xmlfields}
		The first two implement the math provided in version 1.2 of the ASC CDL specification.  The second two omit the clamping step and are intended to provide compatibility with the many applications that take that alternative approach.
\end{xmlfields}

\emph{Elements:}
\begin{xmlfields}
	\xmlitem[SOPNode][optional] The \xml{SOPNode} is optional, and if present, must contain each of the following sub-elements:
	\begin{xmlfields}
		\xmlitemd[Slope] three decimal values representing the R, G, and B slope values, which 
		is similar to gain, but changes the slope of the transfer function without shifting the black level established by \xml{offset} \\
		Valid values for slope must be greater than or equal to zero ($\geq 0$). \\
		The nominal value is 1.0 for all channels.
		\xmlitemd[Offset] three decimal values representing the R, G, and B offset values, which raise or lower overall brightness of a color component by shifting the transfer function up or down while holding the slope constant \\
		The nominal value is 0.0 for all channels.
		\xmlitemd[Power] three decimal values representing the R, G, and B power values, which change the intermediate shape of the transfer function \\
		Valid values for power must be greater than zero ($> 0$). \\
        The nominal value is 1.0 for all channels.
	\end{xmlfields}
	\xmlitem[SatNode][optional] The \xml{SatNode} is optional, but if present, must contain one of the following sub-element:
	\begin{xmlfields} 
		\xmlitemd[Saturation] a single decimal value applied to all color channels \\
		Valid values for saturation must be greater than or equal to zero ($\geq 0$). \\
        The nominal value is 1.0.
	\end{xmlfields}
\end{xmlfields}

\note{If either element is not specified, values should default to the nominal values for each element. If using the \xml{"noClamp"} style, the result of defaulting to the nominal values is a no-op.}

\note{The structure of this \xml{ProcessNode} matches the structure of the XML format described in the v1.2 ASC CDL specification. However, unlike the ASC CDL XML format, there are no alternate spellings allowed for these elements.}

The math for \xml{style="Fwd"} is:
\begin{equation} \label{eq:cdl-fwd}
    \var{out}_{SOP} = \mathrm{CLAMP}(\var{in} \times \var{slope} + \var{offset})^{\var{power}} \\
\end{equation}
\begin{equation}
	\begin{aligned}
		\var{luma} &= 0.2126 \times in_R + 0.7152 \times in_G + 0.0722 \times in_B \\
		\var{out} &= \mathrm{CLAMP}(luma + saturation \times (\var{out}_{SOP} - luma))
	\end{aligned}
\end{equation}

\tabto{0.5in}Where:
\tabto{0.75in} CLAMP() clamps the argument to [0,1]

The math for \xml{style="FwdNoClamp"} is the same as for \xml{"Fwd"} but the two clamp() functions are omitted. Also, if $(input \times slope + offset) < 0$, then no power function is applied.

The math for \xml{style="Rev"} is:
\begin{equation}
	\begin{aligned}
		\var{in}_{clamp} &= \mathrm{CLAMP}(in) \\
		luma &= 0.2126 \times \var{in}_{clamp,R} + 0.7152 \times \var{in}_{clamp,G} + 0.0722 \times \var{in}_{clamp,B} \\
		\var{out}_{SAT} &= luma + \frac{(in_{clamp} - luma)}{saturation}
	\end{aligned}
\end{equation}
\begin{equation}  \label{eq:cdl-rev}
	\var{out} = \mathrm{CLAMP}\left(\frac{\mathrm{CLAMP}(\var{out}_{SAT})^{\frac{1}{\var{power}}} - \var{offset}}{\var{slope}}\right) \\
\end{equation}

\tabto{0.5in}Where:
\tabto{0.75in} CLAMP() clamps the argument to [0,1]

The math for \xml{style="RevNoClamp"} is the same as for \xml{"Rev"} but the CLAMP() functions are omitted. Also, if $\var{out}_{SAT} < 0$, then no power function is applied.

\emph{Examples:}
\begin{lstlisting}[caption=Example of an \xml{ASC\_CDL} node,label=ex:asccdl]
<ASC_CDL id="cc01234" inBitDepth="16f" outBitDepth="16f" style="Fwd">
	<Description>scene 1 exterior look</Description>
	<SOPNode>
		<Slope>1.000000  1.000000  0.900000</Slope>
		<Offset>-0.030000  -0.020000  0.000000</Offset>
		<Power>1.2500000  1.000000  1.000000</Power>
	</SOPNode>
	<SatNode>
		<Saturation>1.700000</Saturation>
	</SatNode>
</ASC_CDL>
\end{lstlisting}
