\subsection{\texttt{LUT1D}}

\emph{Description:} \par
A 1D LUT transform uses an input pixel value, finds the two nearest index positions in the LUT, and then interpolates the output value using the entries associated with those positions. 

This node shall contain either a 1D LUT or a 3$\times$1D LUT in the form of an \xml{Array}. If the input to a \xml{LUT1D} is an RGB value, the same LUT shall be applied to all three color components. 

A 3$\times$1D LUT transform looks up each color component in a separate \xml{LUT1D} of the same length.   In a 3$\times$1D LUT, by convention, the \xml{LUT1D} for the first component goes in the first column of \xml{Array}.

The scaling of the array values is based on the \xml{outBitDepth} (the \xml{inBitDepth} is not considered).

The length of a 1D LUT should be limited to at most 65536 entries, and implementations are not required to support \xml{LUT1D}s longer than 65536 entries.

Linear interpolation shall be used for \xml{LUT1D}. More information about linear interpolation can be found in Appendix \ref{appendix:interpolation}.

\emph{Elements:}
\begin{xmlfields}
    \xmlitem[Array][required] an array of numeric values that are the output values of the 1D LUT.  \xml{Array} shall contain the table entries of a LUT in order from minimum value to maximum value. 
    
    For a 1D LUT, one value per entry is used for all color channels. For a 3$\times$1D LUT, each line should contain 3 values, creating a table where each column defines a 1D LUT for each color component. For RGB, the first column shall correspond to R's 1D LUT, the second column shall correspond to G's 1D LUT, and the third column shall correspond to B's 1D LUT. 

    \emph{Attributes:}
    \begin{xmlfields}
        \xmlitem[dim][required] 
        two integers that represent the dimensions of the array. The first value defines the length of the array and shall equal the number of entries (lines) in the LUT. The second value indicates the number of components per entry and shall equal 1 for a 1D LUT or 3 for a 3$\times$1D LUT. 
        
        \begin{list}{}{\setlength{\itemsep}{4pt}\setlength{\topsep}{0pt}}
            \item e.g. \xml{dim = "1024 3"} indicates a 1024 element 1D LUT with 3 component color (a 3$\times$1D LUT)
            \item e.g. \xml{dim = "256 1"} indicates a 256 element 1D LUT with 1 component color (a 1D LUT)     
        \end{list}
    \end{xmlfields}

    \note{\xml{Array} is formatted differently when it is contained in a \xml{LUT3D} or \xml{Matrix} element (see \ref{sec:array}).}
    
\end{xmlfields}


\emph{Attributes:}
\begin{xmlfields}
    \xmlitem[interpolation][optional] a string indicating the preferred algorithm used to interpolate values in the \xml{1DLUT}. This attribute is optional but, if present, shall be set to \xml{"linear"}.
    
    \note{Previous versions of this specification allowed for implementations to utilize different types of interpolation but did not define what those interpolation types were or how they should be labeled. For simplicity and to ensure similarity across implementations, 1D LUT interpolation has been limited to \xml{"linear"} in this version of the specification. Support for additional interpolation types could be added in future version.}
    
    \xmlitem[halfDomain][optional] If this attribute is present, its value must equal ``\xml{true}''. When true, the input domain to the node is considered to be all possible 16-bit floating-point values, and there must be exactly 65536 entries in the \xml{Array} element.
    
    \note{For example, the unsigned integer 15360 has the same bit-pattern (0011110000000000) as the half-float value 1.0, so the 15360th entry (zero-indexed) in the \xml{Array} element is the output value corresponding to an input value of 1.0.}

    \xmlitem[rawHalfs][optional] If this attribute is present, its value must equal ``\xml{true}''. When true, the \xml{rawHalfs} attribute indicates that the output array values in the form of unsigned 16-bit integers are interpreted as the equivalent bit pattern, half floating-point values. 
    
    \note{For example, to represent the value 1.0, one would use the integer 15360 in the \xml{Array} element because it has the same bit-pattern. This allows the specification of exact half-float values without relying on conversion from decimal text strings.}

\end{xmlfields}

\emph{Examples:}
\begin{lstlisting}[caption=Example of a very simple \xml{LUT1D}]
<LUT1D id="lut-23" name="4 Value Lut" inBitDepth="12i" outBitDepth="12i">
    <Description> 1D LUT - Turn 4 grey levels into 4 inverted codes </Description>
    <Array dim="4 1">
        3
        2
        1
        0
    </Array>
</LUT1D>
\end{lstlisting}