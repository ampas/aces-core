\subsection{\texttt{Matrix}}

\emph{Description:} \par
This node specifies a matrix transformation to be applied to the input values. The input and output of a \xml{Matrix} are always 3-component values. 

All matrix calculations should be performed in floating point, and input bit depths of integer type should be treated as scaled floats. If the input bit depth and output bit depth do not match, the coefficients in the matrix must incorporate the results of the `scale' factor that will convert the input bit depth to the output bit depth (e.g. input of \xml{10i} with an output of \xml{12i} requires the matrix coefficients already have a factor of 4095/1023 applied). Changing the input or output bit depth requires creation of a new set of coefficients for the matrix.

The output values are calculated using row-order convention (\autoref{eq:rowOrder}), which is equivalent in functionality to the equations in \ref{eq:rowOrderMath}.

\begin{equation} \label{eq:rowOrder}
    \begin{bmatrix*}
        a_{11} & a_{12} & a_{13} \\
        a_{21} & a_{22} & a_{23} \\
        a_{31} & a_{32} & a_{33} \\
    \end{bmatrix*} \\
    \begin{bmatrix}
        r_1\\
        g_1\\
        b_1
    \end{bmatrix}
    =
    \begin{bmatrix}
        r_2\\
        g_2\\
        b_2
    \end{bmatrix}
\end{equation}
\begin{equation} \label{eq:rowOrderMath}
    \begin{aligned}
        r_2 = (r_1 \cdot a_{11}) + (g_1 \cdot a_{12}) + (b_1 \cdot a_{13}) \\
	    g_2 = (r_1 \cdot a_{21}) + (g_1 \cdot a_{22}) + (b_1 \cdot a_{23}) \\
	    b_2 = (r_1 \cdot a_{31}) + (g_1 \cdot a_{32}) + (b_1 \cdot a_{33})
    \end{aligned}    
\end{equation}

Matrices using an offset calculation will have one more column than rows. An offset matrix may be defined using a 3$\times$4 \xml{Array} (\ref{eq:3x4matrix}), wherein the fourth column is used to specify offset terms, $k_1$, $k_2$, $k_3$, that are added to the result of the normal matrix calculations (\ref{eq:3x4math}). 

\begin{equation} \label{eq:3x4matrix}
    \begin{bmatrix*}
        a_{11} & a_{12} & a_{13} & k_1\\
        a_{21} & a_{22} & a_{23} & k_2\\
        a_{31} & a_{32} & a_{33} & k_3\\
    \end{bmatrix*} \\
    \begin{bmatrix}
        r_1\\
        g_1\\
        b_1\\
        1.0
    \end{bmatrix}
    =
    \begin{bmatrix}
        r_2\\
        g_2\\
        b_2
    \end{bmatrix}
\end{equation}
\begin{equation} \label{eq:3x4math}
    \begin{aligned}
	r_2 = (r_1 \cdot a_{11}) + (g_1 \cdot a_{12}) + (b_1 \cdot a_{13}) + k_1\\
	g_2 = (r_1 \cdot a_{21}) + (g_1 \cdot a_{22}) + (b_1 \cdot a_{23}) + k_2\\
	b_2 = (r_1 \cdot a_{31}) + (g_1 \cdot a_{32}) + (b_1 \cdot a_{33}) + k_3
    \end{aligned}    
\end{equation}

 
\emph{Elements:}
\begin{xmlfields}
	\xmlitem[Array][required] 
	a table that provides the coefficients of the transformation matrix. The matrix dimensions are either 3$\times$3 or 3$\times$4. The matrix is serialized row by row from top to bottom and from left to right, i.e., ``$a_{11}\ a_{12}\ a_{13}\ a_{21}\ a_{22}\ a_{23}\ \ldots$'' for a 3$\times$3 matrix.
		\begin{equation}
		    \begin{bmatrix*}
		        a_{11} & a_{12} & a_{13} \\
		        a_{21} & a_{22} & a_{23} \\
		        a_{31} & a_{32} & a_{33} \\
		    \end{bmatrix*} \\
		\end{equation}

    \emph{Attributes:}
    \begin{xmlfields}            
    	\xmlitem[dim][required] 
    	two integers that describe the dimensions of the matrix array. The first value define the number of rows and the second is the number of columns. 
    	
    	2 entries have the dimensions of a matrix
		\begin{list}{}{\setlength{\itemsep}{4pt}\setlength{\topsep}{0pt}}
				\item e.g. \xml{dim = "3 3"} indicates a 3$\times$3 matrix
				\item e.g. \xml{dim = "3 4"} indicates a 3$\times$4 matrix
		\end{list}
	\end{xmlfields}

	\note{Previous versions of this specification used three integers for the \xml{dim} attribute, rather than the current two. In order to facilitate backwards compatibility, implementations should allow a third value for the \xml{dim} attribute and may simply ignore it.}
	
	\note{\xml{Array} is formatted differently when it is contained in a \xml{LUT1D} or \xml{LUT3D} element (see \ref{sec:array}).}
\end{xmlfields}


\emph{Examples:}
\begin{lstlisting}[caption=Example of a \xml{Matrix} node with \xml{dim="3 3 3"}]
<Matrix id="lut-28" name="AP0 to AP1" inBitDepth="16f" outBitDepth="16f" >
	<Description>3x3 color space conversion from AP0 to AP1</Description>
	<Array dim="3 3">
         1.45143931614567	  -0.236510746893740	-0.214928569251925
        -0.0765537733960204    1.17622969983357	    -0.0996759264375522
         0.00831614842569772  -0.00603244979102103   0.997716301365324
	</Array>
</Matrix>
\end{lstlisting}

\begin{lstlisting}[caption=Example of a \xml{Matrix} node]
<Matrix id="lut-25" name="colorspace conversion" inBitDepth="10i" outBitDepth="10i" >
	<Description> 3x4 Matrix , 4th column is offset </Description>
	<Array dim="3 4">
		1.2  	0.0  	0.0   	0.002
		0.0 	1.03 	0.001 	-0.005
		0.004 	-0.007 	1.004  	0.0
	</Array>
</Matrix>
\end{lstlisting}