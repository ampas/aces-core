\subsection{\texttt{Exponent}}

\emph{Description:} \par
This node contains parameters for processing pixels through a power law function. Two main formulations are supported. The first follows a pure power law. The second is a piecewise function that follows a power function for larger values and has a linear segment that is followed for small and negative values. The latter formulation can be used to represent the Rec. 709, sRGB, and CIE L* equations.

\emph{Attributes:}
\begin{xmlfields}
	\xmlitem[style][required] specifies the form of the exponential function to be applied.\\
	    Supported values are: 
	    	\begin{itemize}
                \item[-] \xml{"basicFwd"}
                \item[-] \xml{"basicRev"}
                \item[-] \xml{"basicMirrorFwd"}
                \item[-] \xml{"basicMirrorRev"}
                \item[-] \xml{"basicPassThruFwd"}
                \item[-] \xml{"basicPassThruRev"}
                \item[-] \xml{"monCurveFwd"}
                \item[-] \xml{"monCurveRev"}
                \item[-] \xml{"monCurveMirrorFwd"}
                \item[-] \xml{"monCurveMirrorRev"}
            \end{itemize}

		The formula to be applied for each style is included in Equations \ref{eq:expBasicFwd}-\ref{eq:expmonCurveRev} for all of which: \\[10pt]
        \tabto{0.75in} $g =$ \xml{exponent} \tabto {2in} $k =$ \xml{offset}\\[10pt]
        \tabto{0.75in} MAX(${a,b}$) returns $a$ if $a > b$ and $b$ if $b \geq a$

		\begin{xmlfields}
			\xmlitemd["basicFwd"] applies a power law using the exponent value specified in the \xml{ExponentParams} element. \\
			Values less than zero are clamped.
				\begin{equation} \label{eq:expBasicFwd}
					\text{basicFwd}(x) = [\text{MAX}(0, x)]^{g}
				\end{equation}			
			\xmlitemd["basicRev"] applies power law using the exponent value specified in the \xml{ExponentParams} element. \\
			Values less than zero are clamped.
				\begin{equation} \label{eq:expBasicRev}
					\text{basicRev}(y) = [\text{MAX}(0, y)]^{1/g}
				\end{equation}
			\xmlitemd["basicMirrorFwd"] applies a basic power law using the exponent value specified in the \xml{ExponentParams} element for values greater than or equal to zero and mirrors the function for values less than zero (i.e. rotationally symmetric around the origin):
				\begin{equation} \label{eq:expbasicMirrorFwd}
					\text{basicMirrorFwd}(x) = 
					    \begin{dcases}
					        x^{g} & \text{if } x \geq 0 \\[6pt]
					        -\Big[(-x)^{g}\Big] & \text{otherwise}
					    \end{dcases}				\end{equation}			
			\xmlitemd["basicMirrorRev"] applies a basic power law using the exponent value specified in the \xml{ExponentParams} element for values greater than or equal to zero and mirrors the function for values less than zero (i.e. rotationally symmetric around the origin):
				\begin{equation} \label{eq:expbasicMirrorRev}
					\text{basicMirrorRev}(y) = 
					    \begin{dcases}
					        y^{1/g} & \text{if } y \geq 0 \\[6pt]
					        -\Big[(-y)^{1/g}\Big] & \text{otherwise}
					    \end{dcases}
				\end{equation}
			\xmlitemd["basicPassThruFwd"] applies a basic power law using the exponent value specified in the \xml{ExponentParams} element for values greater than or equal to zero and passes values less than zero unchanged:
				\begin{equation} \label{eq:expbasicPassThruFwd}
					\text{basicPassThruFwd}(x) = 
					    \begin{dcases}
					        x^{g} & \text{if } x \geq 0 \\[6pt]
					        x & \text{otherwise}
					    \end{dcases}				\end{equation}			
			\xmlitemd["basicPassThruRev"] applies a basic power law using the exponent value specified in the \xml{ExponentParams} element for values greater than or equal to zero and and passes values less than zero unchanged:
				\begin{equation} \label{eq:expbasicPassThruRev}
					\text{basicPassThruRev}(y) = 
					    \begin{dcases}
					        y^{1/g} & \text{if } y \geq 0 \\[6pt]
					        y & \text{otherwise}
					    \end{dcases}
				\end{equation}				
			\xmlitemd["monCurveFwd"] applies a power law function with a linear segment near the origin
				\begin{equation} \label{eq:monCurveFwd}
					\text{monCurveFwd}(x) = 
					    \begin{cases}
					        \left( \frac{x\:+\:k}{1\:+\:k} \right)^{g} & \text{if } x \geq xBreak \\[8pt]
					        x\:s & \text{otherwise}
					    \end{cases}
				\end{equation}
				\tabto{0.5in} Where: \\[4pt]
                \tabto{1.0in} $xBreak = \dfrac{k}{g-1}$ \\[20pt]
                \tabto{0.5in} And for Eq. \ref{eq:monCurveFwd} and Eq. \ref{eq:expmonCurveRev}: \\[14pt]
                \tabto{1.0in} $s = \left(\dfrac{g-1}{k}\right)  \left(\dfrac{k g}{(g-1)(1+k)}\right)^{g}$
            \xmlitemd["monCurveRev"] applies a power law function with a linear segment near the origin 
				\begin{equation} \label{eq:expmonCurveRev}
					\text{monCurveRev}(y) = 
					    \begin{cases}
					        (1 + k)\:y^{(1/g)} - k & \text{if } y \geq yBreak \\[8pt]
					        \dfrac{y}{s} & \text{otherwise}
					    \end{cases}
				\end{equation} \\[4pt]
				\tabto{0.5in} Where: \\[4pt]
                \tabto{1.0in} $yBreak = \left(\dfrac{k g}{(g-1)(1+k)}\right)^g$ \\[26pt]
			
			\xmlitemd["monCurveMirrorFwd"] applies a power law function with a linear segment near the origin and mirrors the function for values less than zero (i.e. rotationally symmetric around the origin):
				\begin{equation} \label{eq:expmonCurveMirrorFwd}
					\text{monCurveMirrorFwd}(x) = 
					    \begin{cases}
                            \text{monCurveFwd}(x) & \text{if } x \geq 0 \\[8pt]
					        -[\text{monCurveFwd}(-x)] & \text{otherwise}
					    \end{cases}
				\end{equation}
            
            \xmlitemd["monCurveMirrorRev"] applies a power law function with a linear segment near the origin and mirrors the function for values less than zero (i.e. rotationally symmetric around the origin):
				\begin{equation} \label{eq:expmonCurveMirrorRev}
					\text{monCurveMirrorRev}(y) = 
					    \begin{cases}
                            \text{monCurveRev}(y) & \text{if } y \geq 0 \\[8pt]
					        -[\text{monCurveRev}(-y)] & \text{otherwise}
					    \end{cases}
				\end{equation}		
			
			\end{xmlfields}
\end{xmlfields}
			
\note{The above equations assume that the input and output bit-depths are floating-point. Integer values are normalized to the range [0.0, 1.0].} 

\emph{Elements:}
\begin{xmlfields}
	\xmlitem[Description][optional] See \autoref{sec:process-node-elements}
	\xmlitem[ExponentParams][required] contains one or more attributes that provide the values to be used by the enclosing \xml{Exponent} element. \\
	If \xml{style} is any of the ``basic'' types, then only \xml{exponent} is required. \\
	If \xml{style} is any of the ``monCurve'' types, then \xml{exponent} and \xml{offset} are required. \par
	
		\emph{Attributes:}
		\begin{xmlfields}
		    \xmlitem["exponent"][required] the power to which the value is to be raised \\
		    If \xml{style} is any of the ``monCurve'' types, the valid range is [1.0, 10.0]. The nominal value is 1.0.
		    
		    \note{When using a ``monCurve'' style, a value of 1.0 assigned to \xml{exponent} could result in a divide-by-zero error. Implementors should protect against this case.}
		    
		    \xmlitem["offset"][optional] the offset value to use \\
		    If \xml{offset} is used, the enclosing \xml{Exponent} element's style attribute must be set to one of the ``monCurve'' types. Offset is not allowed when \xml{style} is any of the ``basic'' types.\\
            The valid range is [0.0, 0.9]. The nominal value is 0.0.
		    
		    \note{If zero is provided as a value for \xml{offset}, the calculation of $xBreak$ or $yBreak$ could result in a divide-by-zero error. Implementors should protect against this case.}
		    
		    \xmlitem["channel"][optional] the color channel to which the exponential function is applied. \\
		    Possible values are \xml{"R"}, \xml{"G"}, \xml{"B"}.
		    
		    If this attribute is utilized to target different adjustments per channel, up to three \\ \xml{ExponentParams} elements may be used, provided that \xml{"channel"} is set differently in each. If this attribute is not otherwise specified, the exponential function is applied identically to all three color channels.
		\end{xmlfields}
\end{xmlfields}



\emph{Examples:}
\begin{lstlisting}[caption=Using \xml{Exponent} node for applying a 2.2 gamma.]
<Exponent inBitDepth="32f" outBitDepth="32f" style="basicFwd">
	<Description>Basic 2.2 Gamma</Description>
	<ExponentParams exponent="2.2"/>
</Exponent>
\end{lstlisting}

\begin{lstlisting}[caption=Using \xml{Exponent} node for applying the intended EOTF found in IEC 61966-2-1:1999 (sRGB).]
<Exponent inBitDepth="32f" outBitDepth="32f" style="monCurveFwd">
	<Description>EOTF (sRGB)</Description>
	<ExponentParams exponent="2.4" offset="0.055" />
</Exponent>
\end{lstlisting}

\begin{lstlisting}[caption=Using \xml{Exponent} node to apply CIE L* formula.]
<Exponent inBitDepth="32f" outBitDepth="32f" style="monCurveRev">
	<Description>CIE L*</Description>
	<ExponentParams exponent="3.0" offset="0.16" />
</Exponent>
\end{lstlisting}

\begin{lstlisting}[caption=Using \xml{Exponent} node to apply Rec. 709 OETF.]
<Exponent inBitDepth="32f" outBitDepth="32f" style="monCurveRev">
	<Description>Rec. 709 OETF</Description>
	<ExponentParams exponent="1/0.45" offset="0.099" />
</Exponent>
\end{lstlisting}