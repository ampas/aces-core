\section{\texttt{ProcessList}}
\label{sec:processList}

\emph{Description:} \par
The \xml{ProcessList} is the root element for any CLF file and is composed of one or more \xml{ProcessNode}s.  A \xml{ProcessList} is required even if only one \xml{ProcessNode} will be present.  

\note{The last node of the \xml{ProcessList} is expected to be the final output of the LUT. A LUT designer can allow floating-point values to be interpreted by applications and thus delay control of the final encoding through user selections. }

\note{If needed, a \xml{Range} node can be placed at the end of a \xml{ProcessList} to control minimum and maximum output values and clamping.}

\emph{Attributes:}
\begin{xmlfields}
    \xmlitem[id][required] a string to serve as a unique identifier of the \xml{ProcessList}\\
    \xmlitem[compCLFversion][required] a string indicating the minimum compatible CLF specification version required to read this file //
    The \xml{compCLFversion} corresponding to this version of the specification is be \xml{"3.0"}.   \xmlitem[name][optional] a concise string used as a text name of the \xml{ProcessList} for display or selection from an application's user interface
    \xmlitem[inverseOf][optional] a string for linking to another \xml{ProcessList} \xml{id} (unique) which is the inverse of this one
\end{xmlfields}

\emph{Elements:}
\begin{xmlfields}
    \xmlitem[Description][required] a string for comments describing the function, usage, or any notes about the \xml{ProcessList}. A \xml{ProcessList} may contain one or more \xml{Description}s.
    \xmlitem[InputDescriptor][optional] an arbitrary string used to describe the intended source code values of the \xml{ProcessList}.
    \xmlitem[OutputDescriptor][optional] an arbitrary string used to describe the intended output target of the \xml{ProcessList} (e.g. target display)
    \xmlitem[ProcessNode][required] a generic XML element that in practice is substituted with a particular color operator. The \xml{ProcessList} must contain at least one \xml{ProcessNode}. The \xml{ProcessNode} is described in \autoref{sec:ProcessNode}.
    \xmlitem[Info][optional] optional element for including additional custom metadata not needed to interpret the transforms. Includes:
        \begin{xmlfields}
            \xmlitem[AppRelease][optional] a string used for indicating application software release level
            \xmlitem[Copyright][optional] a string containing a copyright notice for authorship of the CLF file
            \xmlitem[Revision][optional] a string used to track the version of the LUT itself (e.g. an increased resolution from a previous version of the LUT)
            \xmlitem[ACEStransformID][optional] a string containing an ACES transform identifier as described in Academy S-2014-002. If the transform described by the \xml{ProcessList} is the concatenation of several ACES transforms, this element may contain several ACES Transform IDs, separated by white space or line separators. This element is mandatory for ACES transforms and may be referenced from ACES Metadata Files. 
            \xmlitem[ACESuserName][optional] a string containing the user-friendly name recommended for use in product user interfaces as described in Academy TB-2014-002.
            \xmlitem[CalibrationInfo][optional] container element for calibration metadata used when making a LUT for a specific device. \xml{CalibrationInfo} can contain the following child elements:
                \begin{list}{}{\setlength{\itemsep}{4pt}}
                    \item \xml{DisplayDeviceSerialNum}
                    \item \xml{DisplayDeviceHostName}
                    \item \xml{OperatorName}
                    \item \xml{CalibrationDateTime}
                    \item \xml{MeasurementProbe}
                    \item \xml{CalibrationSoftwareName}
                    \item \xml{CalibrationSoftwareVersion}
                \end{list}
        \end{xmlfields}
\end{xmlfields}