\subsection{\texttt{Range}}

\emph{Description:} \par
This node maps the input domain to the output range by scaling and offsetting values. The \xml{Range} element can also be used to clamp values.

Unless otherwise specified, the node's default behavior is to scale and offset with clamping. If clamping is not desired, the \xml{style} attribute can be set to \xml{"noClamp"}.

To achieve scale and/or offset of values, all of \xml{minInValue}, \xml{minOutValue}, \xml{maxInValue}, and \xml{maxOutValue} must be present. In this explicit case, the formula for \xml{Range} shall be as in Equation \ref{eq:noClamp}. The scaling of \xml{minInValue} and \xml{maxInValue} depends on the input bit-depth, and the scaling of \xml{minOutValue} and \xml{maxOutValue} depends on the output bit-depth.

\begin{equation}\label{eq:noClamp}
	out = in \times scale + \xml{minOutValue} - \xml{minInValue} \times scale
\end{equation}

If \xml{style="Clamp"}, the output value from Eq \ref{eq:noClamp} is furthur modified via Eq. \ref{eq:clamp}.

\begin{equation}\label{eq:clamp}
	out_{clamped} = \mathrm{MIN}(\xml{maxOutValue}, \mathrm{MAX}( \xml{minOutValue}, out))
\end{equation}

\tabto{0.25in} Where: \\[6pt]
\tabto{0.5in} $scale = \dfrac{(\xml{maxOutValue} - \xml{minOutValue})}{(\xml{maxInValue} - \xml{minInValue})}$ \\[12pt]
\tabto{0.5in} MAX(${a,b}$) returns $a$ if $a > b$ and $b$ if $b \geq a$ \\[6pt]
\tabto{0.5in} MIN(${a,b}$) returns $a$ if $a < b$ and $b$ if $b \leq a$ \\[10pt]

The \xml{Range} element can also be used to clamp values on only the top or bottom end. In such instances, no offset is applied, and the formula simplifies because only one pair of min or max values are required. 

If only the minimum value pair is provided, then the result shall be clamping at the low end, according to Equation \ref{eq:minclamp}.

	\begin{equation}\label{eq:minclamp}
		out = \mathrm{MAX}( \xml{minOutValue}, in \times bitDepthScale)   
	\end{equation}

If only the maximum values pairs are provided, the result shall be clamping at the high end, according to Equation \ref{eq:maxclamp}.
	\begin{equation}\label{eq:maxclamp}
	    out = \mathrm{MIN}( \xml{maxOutValue}, in \times bitDepthScale)
	\end{equation}				

\tabto{0.25in} Where: \\[10pt]
\tabto{0.5in} $bitDepthScale = \dfrac{\mathrm{SIZE}(\xml{outBitDepth})}{\mathrm{SIZE}(\xml{inBitDepth})}$\\[14pt] 
\tabto{0.75in}and where: \\[10pt]
\tabto{1.0in} SIZE$(a) =$ 
\begin{cases}
    $2^{bitDepth}-1$ & \text{when $a \in \{$\xml{"8i"},\xml{"10i"},\xml{"12i"},\xml{"16i"}$\}$} \\
    $1.0$ & \text{when $a \in \{$\xml{"16f"},\xml{"32f"}$\}$}
\end{cases}

In both instances, values must be set such that $\xml{maxOutValue} = \xml{maxInValue} \times bitDepthScale$.

\note{The bit depth scale factor intentionally uses $2^{bitDepth} - 1$ and not $2^{bitDepth}$. This means that the scale factor created for scaling between different bit depths is "non-integer" and is slightly different depending on the bit depths being scaled between. While intinct might be that this scale should be a clean bit-shift factor (i.e. 2x or 4x scale), testing with a few example values plugged into the formula will show that the resulting non-integer scale is the correct and intended behavior.}

At least one pair of either minimum or maximum values, or all four values, must be provided.

\emph{Elements:}
\begin{xmlfields}
	\xmlitem[minInValue][optional] The minimum input value. Required if \xml{minOutValue} is present.
	\xmlitem[maxInValue][optional] The maximum input value. Required if \xml{maxOutValue} is present.
	\xmlitem[minOutValue][optional] The minimum output value. Required if \xml{minInValue} is present.
	\xmlitem[maxOutValue][optional] The maximum output value. Required if \xml{maxInValue} is present.
\end{xmlfields}

\emph{Attributes:}
\begin{xmlfields}
    \xmlitem[style][optional]
		Describes the preferred handling of the scaling calculation of the \xml{Range} node. If the style attribute is not present, clamping is performed.
		\begin{xmlfields}
			\xmlitemd["noClamp"] If present, scale and offset is applied without clamping, as in Eq. \ref{eq:noClamp}. \\
			(i.e. values below \xml{minOutValue} or above \xml{maxOutValue} are preserved) 
			\xmlitemd["Clamp"] If present, clamping is applied according to Eq. \ref{eq:clamp} upon the result of the scale and offset expressed in  Eq. \ref{eq:noClamp}
		\end{xmlfields}
\end{xmlfields}

\emph{Examples:}
\begin{lstlisting}[caption=Using \xml{"Range"} for scaling 10-bit full range to 10-bit SMPTE (legal) range.]
<Range inBitDepth="10i" outBitDepth="10i">
	<Description>10-bit full range to SMPTE range</Description>
	<minInValue>0</minInValue>
	<maxInValue>1023</minInValue>
	<minOutValue>64</minInValue>
	<maxOutValue>940</minInValue>
</Range>
\end{lstlisting}