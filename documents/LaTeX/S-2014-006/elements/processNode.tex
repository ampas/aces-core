\section{\texttt{ProcessNode}}
\label{sec:ProcessNode}

\emph{Description:} \par
A \xml{ProcessNode} element represents an operation to be applied to the image data. At least one \xml{ProcessNode} element must be included in a \xml{ProcessList}. The generic \xml{ProcessNode} element contains attributes and elements that are common to and inherited by the specific sub-types of the \xml{ProcessNode} element that can substitute for \xml{ProcessNode}. All \xml{ProcessNode} substitutes shall inherit the following attributes.

\emph{Attributes:}
\label{sec:process-node-attr}
\begin{xmlfields}
    \xmlitem[id][optional]  a unique identifier for the \xml{ProcessNode}
    \xmlitem[name][optional] a concise string defining a name for the \xml{ProcessNode} that can be used by an application for display in a user interface
    \xmlitem[inBitDepth][required] a string that is used by some \xml{ProcessNode}s to indicate how array or parameter values have been scaled
    \xmlitem[outBitDepth][required] a string that is used by some \xml{ProcessNode}s to indicate how array or parameter values have been scaled
    The supported values for both \xml{inBitDepth} and \xml{outBitDepth} are the same:
        \begin{itemize}
            \item[-] ``\xml{8i}'': 8-bit unsigned integer
            \item[-] ``\xml{10i}'': 10-bit unsigned integer
            \item[-] ``\xml{12i}'': 12-bit unsigned integer
            \item[-] ``\xml{16i}'': 16-bit unsigned integer
            \item[-] ``\xml{16f}'': 16-bit floating point (half-float)
            \item[-] ``\xml{32f}'': 32-bit floating point (single precision)
        \end{itemize}
\end{xmlfields}

\emph{Elements:}
\label{sec:process-node-elements}
\begin{xmlfields}
    \xmlitem[Description][optional] an arbitrary string for describing the function, usage, or notes about the \xml{ProcessNode}. A \xml{ProcessNode} can contain one or more \xml{Description}s.
\end{xmlfields}