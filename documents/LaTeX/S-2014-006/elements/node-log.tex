\subsection{\texttt{Log}}

\def\var#1{\text{\emph{#1}}}

\emph{Description:} \par
This node contains parameters for processing pixels through a logarithmic or anti-logarithmic function. A couple of main formulations are supported. The most basic formula follows a pure logarithm or anti-logarithm of either base 2 or base 10. Another supported formula allows for a logarithmic function with a gain factor and offset. This formulation can be used to convert from linear to Cineon. Another style of log formula follows a piece-wise function consisting of a logarithmic function with a gain factor, an offset, and a linear segment. This style can be used to implement many common ``camera-log" encodings.

\note{The equations for the \xml{Log} node assume integer data is normalized to floating-point scaling. \\\xml{LogParams} do not change based on the input and output bit-depths.}

\note{On occasion it may be necessary to transform a logarithmic function specified in terms of traditional Cineon-style parameters to the parameters used by CLF. Guidance on how to do this is provided in Appendix \ref{appendix:cineon-log}.}


\emph{Attributes:}
\begin{xmlfields}
	\xmlitem[style][required] specifies the form of the of log function to be applied \par
	Supported values for "style" are: 
        \begin{itemize}
            \item[-] \xml{"log10"}
            \item[-] \xml{"antiLog10"}
            \item[-] \xml{"log2"}
            \item[-] \xml{"antiLog2"}
            \item[-] \xml{"linToLog"}
            \item[-] \xml{"logToLin"}
            \item[-] \xml{"cameraLinToLog"}
            \item[-] \xml{"cameraLogToLin"}
        \end{itemize}
            
	The formula to be applied for each style is described by the equations below, for all of which: \\[6pt]
        \tabto{0.75in} $\texttt{FLT\_MIN} = 1.175494\mathrm{e}{-38}$\\[6pt]
        \tabto{0.75in} MAX(${a,b}$) returns $a$ if $a > b$ and $b$ if $b \geq a$

	    \begin{itemize}
            \item[-] \xml{"log10"}: applies a base 10 logarithm according to
            	\begin{equation}
					y = \text{log}_{10}(\mathrm{MAX}(x,\texttt{FLT\_MIN}))
				\end{equation}
				
            \item[-] \xml{"antiLog10"}: applies a base 10 anti-logarithm according to
				\begin{equation}
					x = 10^{y}
				\end{equation}
				
            \item[-] \xml{"log2"}: applies a base 2 logarithm according to
				\begin{equation}
					y = \text{log}_{2}(\mathrm{MAX}(x,\texttt{FLT\_MIN}))
				\end{equation}
				
            \item[-] \xml{"antiLog2"}: applies a base 2 anti-logarithm according to
				\begin{equation}
					x = 2^{y}
				\end{equation}
            
            \item[-] \xml{"linToLog"}: applies a logarithm according to
                \begin{equation} \label{eq:linToLog}
                    y = \var{logSideSlope} \times \text{log}_{\var{base}}(\mathrm{MAX}( \var{linSideSlope} \times x + \var{linSideOffset},\texttt{FLT\_MIN})) + \var{logSideOffset}
                \end{equation}            
            
            \item[-] \xml{"logToLin"}: applies an anti-logarithm according to
                \begin{equation} \label{eq:logToLin}
                    x = \frac{\left(\var{base}^{\frac{y-\var{logSideOffset}}{\var{logSideSlope}}} - \var{linSideOffset}\right)}{\var{linSideSlope}}
                \end{equation}
            
            \item[-] \xml{"cameraLinToLog"}: applies a piecewise function with logarithmic and linear segments on linear values, converting them to non-linear values \par

                \begin{equation} \label{eq:camLinToLog}
                \resizebox{.85\hsize}{!}{$
                    y = 
                    \begin{dcases}
                        \var{linearSlope} \times x + \var{linearOffset} & \text{if } x \leq \var{linSideBreak} \\
                        \var{logSideSlope} \times \text{log}_{\var{base}}(\mathrm{MAX}(\var{linSideSlope} \times x + \var{linSideOffset},\texttt{FLT\_MIN})) + \var{logSideOffset} & \text{otherwise} \\
                    \end{dcases}
                    $}
                \end{equation}
            
            \tabto{0.5in} where: \\
                \tabto{0.75in} \var{linearSlope} is calculated using Eq. \ref{eq:slopeAtBreak}
                \tabto{0.75in} \var{linearOffset} is calculated using Eq. \ref{eq:k}
            
            \item[-] \xml{"cameraLogToLin"}: applies a piecewise function with logarithmic and linear segments on non-linear values, converting them to linear values
            \begin{equation} \label{eq:camLogToLin}
                x = 
                \begin{dcases}
                    \frac{(y - \var{linearOffset})}{\var{linearSlope}} & \text{if } y \leq \var{logSideBreak} \\
                    \frac{\left(\var{base}^{\frac{y-\var{logSideOffset}}{\var{logSideSlope}}} - \var{linSideOffset}\right)}{\var{linSideSlope}} & \text{otherwise}
                \end{dcases}
            \end{equation}
            \tabto{0.5in} where: \\
                \tabto{0.75in} \var{logSideBreak} is calculated using Eq. \ref{eq:logSideBreak}
                \tabto{0.75in} \var{linearSlope} is calculated using Eq. \ref{eq:slopeAtBreak}
                \tabto{0.75in} \var{linearOffset} is calculated using Eq. \ref{eq:k}

        \end{itemize}
\end{xmlfields}

\emph{Elements:}
\begin{xmlfields}
	\xmlitem[LogParams][required - if \xml{"style"} is not a basic logarithm] contains the attributes that control the \xml{"linToLog"}, \xml{"logToLin"}, \xml{"cameraLinToLog"}, \\or \xml{"cameraLogToLin"} functions \par
	
	This element is required if \xml{style} is of type \xml{"linToLog"}, \xml{"logToLin"}, \xml{"cameraLinToLog"}, or \xml{"cameraLogToLin"}.

    \emph{Attributes:}	
    \begin{xmlfields}
    	\xmlitem["base"][optional] the base of the logarithmic function \\Default is 2.
    	\xmlitem["logSideSlope"][optional] ``slope" (or gain) applied to the log side of the logarithmic segment. \\Default is 1.
    	\xmlitem["logSideOffset"][optional] offset applied to the log side of the logarithmic segment. \\Default is 0.
    	\xmlitem["linSideSlope"][optional] slope of the linear side of the logarithmic segment. \\Default is 1.
    	\xmlitem["linSideOffset"][optional] offset applied to the linear side of the logarithmic segment. \\Default is 0.

    	\xmlitem["linSideBreak"][optional] the break-point, defined in linear space, at which the piece-wise function transitions between the logarithmic and linear segments. \\This is required if \xml{style="cameraLinToLog"} or \xml{"cameraLogToLin"}
    	
    	\xmlitem["linearSlope"][optional] the slope of the linear segment of the piecewise function. This attribute does not need to be provided unless the formula being implemented requires it. The default is to calculate using \xml{linSideBreak} such that the linear portion is continuous in value with the logarithmic portion of the curve, by using the value of the logarithmic portion of the curve at the break-point. This is described in the following note, using Equations \ref{eq:logSideBreak}-\ref{eq:k}.
    	
		\xmlitem["channel"][optional] the color channel to which the exponential function is applied. Possible values are \xml{"R"}, \xml{"G"}, \xml{"B"}.
        If this attribute is utilized to target different adjustments per channel, then up to three \xml{LogParams} elements may be used, provided that \xml{"channel"} is set differently in each. However, the same value of \xml{base} must be used for all channels. If this attribute is not otherwise specified, the logarithmic function is applied identically to all three color channels.  	
    \end{xmlfields}

\end{xmlfields}

\note{  \emph{linearOffset} is the offset of the linear segment of the piecewise function. This value is calculated using the position of the break point and the linear slope in order to ensure continuity of the two segments. Equations \ref{eq:logSideBreak}-\ref{eq:k} describe the steps for calculating \emph{linearOffset}.
        
First, the value of the break-point on the log-axis is calculated using the value of \xml{linSideBreak} as input to the logarithmic segment of Eq. \ref{eq:camLinToLog}, as shown in Eq. \ref{eq:logSideBreak}.
    	\begin{equation} \label{eq:logSideBreak}
        	\resizebox{0.8\textwidth}{!}{$
        	\var{logSideBreak} = \var{logSideSlope} \times \text{log}_{\var{base}}( \var{linSideSlope} \times \textbf{\var{linSideBreak}} + \var{linSideOffset}) + \var{logSideOffset}
        $}
    	\end{equation}

Then, if \xml{linearSlope} was not provided, the value of \xml{linSideBreak} is used again to solve for the derivative of Eq. \ref{eq:linToLog}. The value of \var{linearSlope} is set to equal the the instantaneous slope at the break-point, or derivative which is shown being solved for by Eq. \ref{eq:slopeAtBreak}:
    	\begin{equation} \label{eq:slopeAtBreak}
        	\var{linearSlope} = \var{logSideSlope} \times \left(\frac{\var{linSideSlope}}{(\var{linSideSlope} \times \textbf{\var{linSideBreak}} + \var{linSideOffset}) \times \text{ln}(\var{base})}\right)
    	\end{equation}
    	
Finally, the value of \var{linearOffset} can be solved for by rearranging the linear segment of Eq. \ref{eq:camLinToLog} to get Equation \ref{eq:k}, and using the values of \var{logSideBreak} (obtained from Eq. \ref{eq:logSideBreak}) and \var{linearSlope} (obtained from Eq. \ref{eq:slopeAtBreak}).
    	\begin{equation} \label{eq:k}
    	    \var{linearOffset} = \textbf{\var{logSideBreak}} - \textbf{\var{linearSlope}} \times \var{linSideBreak}
    	\end{equation}
}

\emph{Examples:}
\begin{lstlisting}[caption=Example \xml{Log} node applying a base 10 logarithm.]
<Log inBitDepth="16f" outBitDepth="16f" style="log10">
	<Description>Base 10 Logarithm</Description>
</Log>
\end{lstlisting}

\begin{lstlisting}[caption=Example \xml{Log} node applying the DJI D-Log formula.]
<Log inBitDepth="32f" outBitDepth="32f" style="cameraLinToLog">
	<Description>Linear to DJI D-Log</Description>
	<LogParams base="10" logSideSlope="0.256663" logSideOffset="0.584555" 
	    linSideSlope="0.9892" linSideOffset="0.0108" linSideBreak="0.0078" 
	    linearSlope="6.025"/>
</Log>
\end{lstlisting}