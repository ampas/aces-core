\appendixchapter{Cineon-style Log Parameters}{i}
\label{appendix:cineon-log}

When using a \xml{Log} node, it might be desirable to conform an existing logarithmic function that uses Cineon style parameters to the parameters used by CLF. A translation from Cineon-style parameters to those used by CLF's \xml{LogParams} element is quite straightforward using the following steps.

Traditionally, \var{refWhite} and \var{refBlack} are provided as 10-bit quantities, and if they indeed are, first normalize them to floating point by dividing by 1023.
\begin{align}
    \var{refWhite} = \frac{\var{refWhite}_{10i}}{1023.0} \\
    \var{refBlack} = \frac{\var{refBlack}_{10i}}{1023.0}        
\end{align}
\tabto{0.5in} Where:
\tabto{0.75in} subscript \emph{10i} indicates a 10-bit quantity

The density range is assumed to be:
\begin{equation}
    \var{range} = 0.002 \times 1023.0
\end{equation}

Then solve the following quantities:
\begin{align}
    \var{multFactor} =& \frac{\var{range}}{\var{gamma}} \\
    \var{gain} =& \frac{\var{highlight} - \var{shadow}}{1.0 - 10^{( MIN( \var{multFactor} \times (\var{refBlack}-\var{refWhite}), -0.0001)}} \\
    \var{offset} =& \var{gain} - (\var{highlight} - \var{shadow}) \\
\end{align}
\tabto{0.5in} Where:
\tabto{0.75in} $MIN(x,y)$ returns $x$ if $x<y$, otherwise returns $y$

The parameters for the \xml{LogParams} element are then:
\begin{align}
    \xml{base} =& 10.0 \\
    \xml{logSlope} =& \frac{1}{\var{multFactor}} \\
    \xml{logOffset} =& \var{refWhite} \\
    \xml{linSlope} =& \frac{1}{\var{gain}} \\
    \xml{linOffset} =& \frac{\var{offset}-\var{shadow}}{\var{gain}}
\end{align}