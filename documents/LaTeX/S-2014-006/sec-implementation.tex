% This file contains the content for a main section
\regularsectionformat
%% Modify below this line %%
\chapter{Implementation Notes}
\label{sec:implementation}

% -- Bit Depth Behavior --
\section{Bit Depth}

\subsection{Processing Precision} \label{sec:process-precision}
All processing shall be performed using 32-bit floating-point values. The values of the \xml{inBitDepth} and 
\xml{outBitDepth} attributes shall not affect the quantization of color values. 

\note{For some hardware devices, 32-bit float processing might not be possible. In such instances, processing should be performed at the highest precision available. Because CLF permits complex series of discrete operations, CLF LUT files are unlikely to run on hardware devices without some form of pre-processing. Any pre-processing to prepare a CLF for more limited hardware applications should adhere to the processing precision requirements.}

\subsection{Input To and Output From a \texttt{ProcessList}} \label{sec:processList-in-out}
Applications often support multiple pixel formats (e.g. 8i, 10i, 16f, 32f, etc.). Often the actual pixel format to be processed may not agree with the \xml{inBitDepth} of the first \xml{ProcessNode} or the \xml{outBitDepth} of the last \xml{ProcessNode}. (Note that the \xml{ProcessList} element itself does not contain global \xml{inBitDepth} or \xml{outBitDepth} attributes.) Therefore, in some cases an application may need to rescale a given \xml{ProcessNode} to be appropriate for the actual image data being processed.

For example, if the last \xml{ProcessNode} in a \xml{ProcessList} is a \xml{LUT1D} with an \xml{outBitDepth} of 12i, it indicates that the LUT \xml{Array} values are scaled relative to 4095. If the application wants to produce floating-point pixel values, it should therefore divide the LUT \xml{Array} values by 4095 before processing the pixels (according to \ref{sec:scaling}). Likewise, if the \xml{outBitDepth} was instead 32f and the application wants to produce \xml{12i} pixel values, it should multiply the LUT \xml{Array} values by 4095. (Note that in this case, since the result of the computations may exceed 4095 and the application wants to produce 12-bit integer output, the application would want to clamp, round, and quantize the value.)


\subsection{Input To and Output From a \texttt{ProcessNode}}
In order to ensure the scaling of parameter values of all \xml{ProcessNode}s in a \xml{ProcessList} are consistent, the \xml{inBitDepth} of each \xml{ProcessNode} must match the \xml{outBitDepth} of the previous \xml{ProcessNode} (if any). 

Please note that an integer \xml{inBitDepth} or \xml{outBitDepth} of a \xml{ProcessNode} does not indicate that any clamping or quantization should be done. These attributes are strictly used to indicate the scaling of parameter and array values. As discussed in \ref{sec:process-precision}, processing precision shall be floating-point.

Furthermore, because the processing precision is intended to be floating-point, the \xml{inBitDepth} and \xml{outBitDepth} only control the scaling of parameter and array values and do not impose range limits. For example, even if the \xml{outBitDepth} of a LUT \xml{Array} is 12i, it does not mean that the \xml{Array} values must be limited to [0,4095] or that they must be integer values. It simply means that in order to rescale to 32f that a scale factor of $1/4095$ should be used (as per \ref{sec:scaling}).

Because processing within a \xml{ProcessList} should be done at floating-point precision, applications may optionally want to rescale the interfaces all \xml{ProcessNode}s ``interior" to a \xml{ProcessList} to be 32f according to \ref{sec:scaling}. As discussed in \ref{sec:processList-in-out}, applications may want to rescale the ``exterior" interfaces of the \xml{ProcessList} based on the type of pixel data being processed.

For some applications, it may be easiest to simply rescale all \xml{ProcessNode}s to 32f input and output bit-depth when parsing the file. That way, the \xml{ProcessList} may be considered a purely 32f set of operations and the implementation therefore does not need to track or deal with bit-depth differences at the \xml{ProcessNode} level.


\subsection{Conversion Between Integer and Normalized Float Scaling} \label{sec:scaling}
As discussed above, the \xml{inBitDepth} or \xml{outBitDepth} of a \xml{ProcessNode} may need to be rescaled in order to accommodate the pixel data type being processed by the application. 

The scale factor associated with the bit-depths 8i, 10i, 12i, and 16i is $2^n-1$, where $n$ is the bit-depth.

The scale factor associated with the bit-depths 16f and 32f is 1.0.

To rescale \xml{Matrix}, \xml{LUT1D}, or \xml{LUT3D} \xml{Array} values when the \xml{outBitDepth} changes, the scale factor is equal to $\frac{\mathrm{newScale}}{\mathrm{oldScale}}$. For example, to convert from 12i to 10i, multiply array values by $1023/4095$.

To rescale \xml{Matrix} \xml{Array} values when the \xml{inBitDepth} changes, the scale factor is equal to $\frac{\mathrm{oldScale}}{\mathrm{newScale}}$. For example, to convert from 32f to 10i, multiply array values by $1/1023$.

To rescale \xml{Range} parameters when the \xml{inBitDepth} changes, the scale factor for \xml{minInValue} and \\ \xml{maxInValue} is $\frac{\mathrm{newScale}}{\mathrm{oldScale}}$. To rescale \xml{Range} parameters when the \xml{outBitDepth} changes, the scale factor for \xml{minOutValue} and \xml{maxOutValue} is $\frac{\mathrm{newScale}}{\mathrm{oldScale}}$.

Please note that in all cases, the conversion shall be only a scale factor. In none of the above cases, should clamping or quantization be applied.

Aside from the specific cases listed above, changes to \xml{inBitDepth} and \xml{outBitDepth} do not affect the parameter or array values of a given \xml{ProcessNode}.

If an application needs to convert between different integer pixel formats or between integer and float (or vice versa) on the way into or out of a \xml{ProcessList}, the same scale factors should be used. Note that when converting from floating-point to integer at the application level that values should be clamped, rounded, and quantized.


\section{Required vs Optional}
The required or optional indicated in parentheses throughout this specification indicate the requirement for an element or attribute to be present for a valid CLF file. In the spirit of a LUT format to be used commonly across different software and hardware, none of the elements or attributes should be considered optional for implementors to support. All elements and attributes, if present, should be recognized and supported by an implementation.

If, due to hardware or software limitations, a particular element or attribute is not able to be supported, a warning should be issued to the user of a LUT that contains one of the offending elements. The focus shall be on the user and maintaining utmost compatibility with the specification so that LUTs can be interchanged seamlessly.

\section{Efficient Processing}
\label{sec:efficient-processing}
The transform engine may merge some \xml{ProcessNode}s in order to obtain better performance. For example, adjacent \xml{Matrix} operators may be combined into a single matrix. However, in general, combining operators in a way that preserves accuracy is difficult and should be avoided. 

Hardware implementations may need to convert all \xml{ProcessNode}s into some other form that is consistent with what the hardware supports. For example, all \xml{ProcessNode}s might need to be combined into a single 3D LUT. Using a grid size of 64 or larger is recommended to preserve as much accuracy as possible. Implementors should be aware that the success of such approximations varies greatly with the nature of the input and output color spaces. For example, if the input color space is scene-linear in nature, it may be necessary to use a ``shaper LUT'' or similar non-linearity before the 3D LUT in order to convert values into a more perceptually uniform representation.

 
\section{Extensions}
It is recommended that implementors of CLF file readers protect against unrecognized elements or attributes that are not defined in this specification. Unrecognized elements that are not children of the \xml{Info} element should either raise an error or at least provide a warning message to the user to indicate that there is an operator present that is not recognized by the reader. Applications that need to add custom metadata should place it under the \xml{Info} element rather than at the top level of the \xml{ProcessList}.

One or more \xml{Description} elements in the \xml{ProcessList} can and should be used for metadata that does not fit into a provided field in the \xml{Info} element and/or is unlikely to be recognized by other applications.