\documentclass[10pt]{AcademyDoc}
\pagestyle{plain}

\usepackage{color}
\definecolor{green}{rgb}{0.1,0.5,0.1}
\definecolor{gray}{rgb}{0.4,0.4,0.4}
\definecolor{darkblue}{rgb}{0.0,0.0,0.6}
\definecolor{cyan}{rgb}{0.0,0.6,0.6}

\lstset{
  basicstyle={\ttfamily\fontsize{8}{9.6}\selectfont},  
  columns=fixed,
  language=XML,
  showstringspaces=false,
  commentstyle=\color{green}\upshape
}

\lstdefinelanguage{XML}
{
  tabsize=2,
  morestring=[b]",
  morestring=[s]{>}{<},
  morecomment=[s]{<?}{?>},
  stringstyle=\color{black},
  identifierstyle=\bfseries\color{darkblue},
  keywordstyle=\color{cyan},
  morekeywords={xmlns, id, name, compCLFversion, inverseOf, inBitDepth, outBitDepth, style, rawHalfs, halfDomain, dim, interpolation, base, logSideSlope, logSideOffset, linSideSlope, linSideOffset, linSideBreak, linearSlope, linearOffset, exponent, offset, channel, 
  type, minOccurs, maxOccurs, use, abstract, substitutionGroup, default, value}% list your attributes here
}

\lstset{language=XML}

\def\xml#1{{\small \texttt{#1}}}
\newcommand{\matr}[1]{\mathbf{#1}}

% Document specific settings/overrides
\renewcommand\lstlistingname{Example}
\lstset{
    tabsize=4,
    basicstyle={\ttfamily\fontsize{8}{9.6}\selectfont},
    frame=single,
    numberbychapter=false,
    captionpos=b
}


% Set Document Details
\doctype{spec} % spec, proc, tb (Specification, Procedure, Technical Bulletin)
\docname{Common LUT Format (CLF) - A Common File Format for Look-Up Tables}
\altdocname{Common LUT Format (CLF)}
\docnumber{S-2014-006}
\committeename{Academy Color Encoding System (ACES) Project Committee}
\docdate{May 19, 2020}
\summary{
This document specifies a human-readable text file format for the interchange of color transformations using an XML schema. The XML format supports Look-Up Tables of several types: 1D LUTs, 3D LUTs, and 3$\times$1D LUTs, as well as additional transformation needs such as matrices, range rescaling, and `shaper LUTs'. The document defines a processing model for color transformations where each transformation is defined by a `Node' that operates upon a stream of image pixels. A node contains the data for a transformation, and a sequence of nodes can be specified in which the output of one transform feeds into the input of another node. The XML representation allows saving in a text file both a chain of multiple nodes or a single node representing a unique transform. The format is extensible and self-contained so the XML file may be used as an archival element.
}


% Document Starts Here
\begin{document}

\maketitle

% This file contains the content for the Notices
\prelimsectionformat	% Change formatting to that of "Notices" section
\chapter{\uppercase{Notices}}
%% Modify below this line %%

\copyright\the\year{} Academy of Motion Picture Arts and Sciences (A.M.P.A.S.). All rights reserved. This document is provided to individuals and organizations for their own internal use, and may be copied or reproduced in its entirety for such use. This document may not be published, distributed, publicly displayed, or transmitted, in whole or in part, without the express written permission of the Academy.

The accuracy, completeness, adequacy, availability or currency of this document is not warranted or guaranteed. Use of information in this document is at your own risk. The Academy expressly disclaims all warranties, including the warranties of merchantability, fitness for a particular purpose and non-infringement.

Copies of this document may be obtained by contacting the Academy at councilinfo@oscars.org.

``Oscars,'' ``Academy Awards,'' and the Oscar statuette are registered trademarks, and the Oscar statuette a copyrighted property, of the Academy of Motion Picture Arts and Sciences.

% This paragraph is optional.  Comment out if you wish to remove it.
This document is distributed to interested parties for review and comment. A.M.P.A.S. reserves the right to change this document without notice, and readers are advised to check with the Council for the latest version of this document.

% This paragraph is optional.  Comment out if you wish to remove it.
The technology described in this document may be the subject of intellectual property rights (including patent, copyright, trademark or similar such rights) of A.M.P.A.S. or others. A.M.P.A.S. declares that it will not enforce any applicable intellectual property rights owned or controlled by it (other than A.M.P.A.S. trademarks) against any person or entity using the intellectual property to comply with this document.

% This paragraph is optional.  Comment out if you wish to remove it.
Attention is drawn to the possibility that some elements of the technology described in this document, or certain applications of the technology may be the subject of intellectual property rights other than those identified above. A.M.P.A.S. shall not be held responsible for identifying any or all such rights. Recipients of this document are invited to submit notification to A.M.P.A.S. of any such intellectual property of which they are aware.

\vspace{10pt}
These notices must be retained in any copies of any part of this document. \newpage

\tableofcontents \newpage

% This file contains the content for the Introduction
\unnumberedformat	    % Change formatting to that of "Introduction" section
\chapter{Introduction} 	% Do not modify section title
%% Modify below this line %%

The Academy Color Encoding Specification (ACES) defines a common color encoding method using half-precision floating point values corresponding to linear exposure values encoded relative to a fixed set of extended-gamut RGB primaries. Many digital-intermediate color grading systems have been engineered assuming image data with primaries similar to the grading display and a logarithmic relationship between relative scene exposures and image code values.

This document describes a 32-bit single precision floating-point logarithm encoding of ACES known as ACEScct.

ACEScct uses values above 1.0 and below 0.0 to encode the entire range of ACES values. ACEScct values should not be clamped except as part of color correction needed to produce a desired artistic intent.

There is no image file container format specified for use with ACEScct as the encoding is intended to be transient and internal to software or hardware systems, and is specifically not intended for interchange or archiving.
% This file contains the content for the Scope
\cleardoublepage
\numberedformat	
\chapter{Scope} 	% Do not modify section title
%% Modify below this line %%

This document describes a 32-bit floating point encoding of ACES for use within color grading systems. 

Equivalent functions may be used for implementation purposes as long as correspondence of grading parameters to this form of log implementation is properly maintained. This document is intended as a guideline to aid developers who are integrating an ACES workflow into a color correction system.
% This section contains the content for the References
\numberedformat
\chapter{References}
The following standards, specifications, articles, presentations, and texts are referenced in this text:
%% Modify below this line %%

IETF RFC 3066:  IETF (Internet Engineering Task Force). RFC 3066: Tags for the Identification of Languages, ed. H. Alvestrand. 2001 IEEE DRAFT Standard P123

Academy S-2014-002, Academy Color Encoding System -- Versioning System

Academy TB-2014-002, Academy Color Encoding System Version 1.0 User Experience Guidelines

ASC Color Decision List (ASC CDL) Transfer Functions and Interchange Syntax. ASC-CDL\_Release1.2. 
Joshua Pines and David Reisner. 2009-05-04.

\newpage

% This file contains the content for a main section
\regularsectionformat
%% Modify below this line %%
\chapter{Specification}
In the following definitions, \textit{italics} represent a changeable placeholder. \textbf{boldface} represents a required string or character.

\section{String Formats}
ACES system components shall use the following versioning string formats where applicable:

\texttt{\textit{Type.}\textbf{a}\textit{MajorVersionNumber.MinorVersionNumber.PatchVersionNumber}}

or

\texttt{\textit{Type.Name.}\textbf{a}\textit{MajorVersionNumber.MinorVersionNumber.PatchVersionNumber}}

where \texttt{Type} is one of the following:

\begin{listize}
    \item \texttt{IDT} -- ACES Input Transform (a.k.a. ``Input Device Transform'')
    \item \texttt{LMT} -- ACES Look Transform (a.k.a. ``Look Modification Transform'')
    \item \texttt{ODT} -- Output Device Transform
    \item \texttt{RRT} -- Reference Rendering Transform
    \item \texttt{RRTODT} -- ACES Output Transform (concatenated RRT and ODT)
    \item \texttt{InvRRT} -- Inverse Reference Rendering Transform
    \item \texttt{InvODT} -- Inverse Output Device Transform
    \item \texttt{InvRRTODT} -- ACES inverse Output Transform (concatenated RRT and ODT)
    \item \texttt{ACESlib} -- ACES library functions for core transforms, e.g., Tonescales
    \item \texttt{ACEScsc} -- ACES color space conversion transforms
    \item \texttt{ACESutil} -- utility functions provided with the ACES release, e.g., Adjust\_Exposure
\end{listize}

ACES system components are assigned a string that serves as a unique identifier for an ACES System Release. This identifier is constructed using a set of tokens as described in this specification so that it will be more human-readable than a typical Universally Unique Identifier (UUID).

If the \texttt{PatchVersionNumber} is zero, it may be omitted from the versioning string for simplification. If both the \texttt{MinorVersionNumber} and the \texttt{PatchVersionNumber} are zero, they may both be omitted from the versioning string for simplification.

\section{Transform Identifiers}
ACES transforms, expressed as CTL files, are assigned a Transform Identifier.  The Transform Identifier shall be included with all Product Partner implementations intended to match that ACES transform. The Transform Identifier shall be contained in the transform files as metadata or as a comment. For Academy-supplied transforms, the Transform Identifier shall be contained in an XML tag of \texttt{<ACEStransformID>} in the file header.

Product Partner implementation transforms may be intended to match the results of a combined series of ACES CTL transforms, e.g., LMT+RRT+ODT.  In that case, all of the relevant ACES Transform Identifiers shall be included in the implementation Transform. The RRT+ODT combination is a unique case that is covered in \autoref{sec:rrtodt} below.

\section{User-Friendly Names}
ACES Transform Identifiers can be complex and therefore not appropriate for presentation to end users for selection purposes. All transforms shall include ``friendly names'' as metadata within the transform file that software applications may access them for presentation in their user interfaces. For Academy-supplied transforms, the user-friendly name shall be contained in an XML tag of \texttt{<ACESuserName>} in the file header. Recommended friendly names are described in a separate document, ``Academy TB-2014-002, ACES Version 1.0 User Experience Guidelines.''

\section{ACES System Release}
The ACES System Release consists of a variety of ACES core components and ACES vendor-supplied components.

The ACES System Release version shall use the following versioning convention: 

\texttt{\textbf{ACESrelease.a}\textit{MajorVersionNumber.MinorVersionNumber.PatchVersionNumber}}

The ACES System Release patch version number shall be incremented with bug fixes. New patch versions shall not require an update to all transforms.

The ACES System Release minor version number shall be incremented with non-substantive changes to the existing ACES core components. Minor version releases may include new ACES Core Transforms (e.g. new ODTs) and/or roll-ups of minor ODT enhancements/additions or bug fixes. New ACES System Release minor versions shall not require an update to all ACES core and vendor-supplied components.

The ACES System Release major version number shall be incremented with substantive changes to the ACES core components. When the ACES System Release major version number is changed it will require all core and vendor-supplied components be updated to confirm compatibility with the new ACES major version.

The ACES System Release version will not be incremented when ACES vendor-supplied components are updated.

\section{ACES Core Components}
ACES core components include: 

\begin{listize}[-]
	\item ACES Core Transforms
	\item ACES Core Libraries and Utilities
	\item ACES Core File Formats
\end{listize}

\subsection{ACES Core Transforms}
The ACES Core Transforms include the following:

\begin{listize}[-]
	\item The Reference Rendering Transform
	\item Academy-supplied Output Device Transforms
	\item Academy-supplied Look Modification Transforms
	\item Color Space Conversion Transforms
\end{listize}

Transforms such as the RRT and ODTs rely on sub-functions and constants included in separate CTL files, i.e. ACESlib. ACESlib files often contain more than one sub-function or constant. If a change is made to the code of a sub-function that affects the output of a calling transform, the version of the calling transform's Transform Identifier shall be incremented (even if the code in the RRT, ODT, etc. itself may not have changed). For simple additions or modifications to an ACESlib file that do not affect the numerical output of a calling function, the calling function version Transform Identifier will not be incremented. 

Any transform updates that do not change the output of that transform shall not require the Transform Identifier to be incremented - e.g. whitespace changes, modifications to code comments, etc.

Because the results of an ODT depend on the RRT, the version of all ODTs shall be incremented whenever the RRT version is incremented.

\subsubsection{Reference Rendering Transform (RRT)}
The RRT major version number shall match the ACES System Version Major Version number. The RRT Transform Identifier shall use the following versioning convention:

\texttt{\textbf{RRT.a}\textit{ACESmajorVersionNumber.RRTminorVersionNumber.RRTpatchVersionNumber}}

Example Transform Identifiers using this format are: 
\begin{listize}
	\item \texttt{RRT.a1.0.0}
	\item \texttt{RRT.a1}
\end{listize}

\subsubsection{Academy-supplied Output Transforms (ODTs)}
An ODT's major version number shall match the ACES System Version Major Version number. ODT Transform Identifiers shall use the following versioning convention:

\begin{sloppypar}
\texttt{\textbf{ODT}\textit{.Namespace.OutputFormat.\-}\textbf{a}\textit{ACESmajorVersionNumber.\-ODTminor\-Version\-Number.ODTpatchVersion\-Number}}
\end{sloppypar}

\texttt{\textit{Namespace}} identifies the creator of the ODT. The \texttt{\textit{Namespace}} \texttt{Academy} is reserved for Academy-supplied ODTs.

\texttt{\textit{OutputFormat}} fully describes the device and/or output data format of the ODT.

Example Transform Identifiers using this format are:
\begin{listize}
	\item \texttt{ODT.Academy.P3D60\_48nits.a1.0.0}
	\item \texttt{ODT.Academy.Rec709\_D60sim\_100nits\_dim.a1.0.0}
\end{listize}

The Academy provides all ODTs in ACES 1.0, although it is anticipated that vendors will provide ACES-compatible ODTs in the future.

\subsubsection{Academy-supplied Look Modification Transforms (LMTs)}
Academy-supplied LMT's major version number shall match the ACES System Version Major Version number. LMT Transform Identifiers shall use the following versioning convention:

\begin{sloppypar}
\texttt{\textbf{LMT}\textit{.Namespace.Name.\-}\textbf{a}\textit{ACESmajorVersionNumber.\-LMTminorVersionNumber.LMT\-patchVersionNumber}}
\end{sloppypar}

\subsubsection{Color Space Conversion Transforms}
Academy-supplied color space conversions' major version number shall match the ACES System Version Major Version number. Color space conversion Transform Identifiers shall use the following versioning convention:

\begin{sloppypar}
\texttt{\textbf{ACEScsc}\textit{.Name.\-}\textbf{a}\textit{ACESmajorVersionNumber.\-ACEScscMinorVersionNumber.\-ACEScsc\-PatchVersionNumber}}
\end{sloppypar}

\subsection{ACES Core Libraries and Utilities}
Academy-supplied core libraries and utilities' major version number shall match the ACES System Version Major Version number. Core library and utilities Transform Identifiers shall use the following versioning convention:

\begin{sloppypar}
\texttt{\textbf{ACESlib}\textit{.Name.\-}\textbf{a}\textit{ACESmajorVersionNumber.ACESlibMinorVersionNumber.ACESlib\-Patch\-Version\-Number}}
\end{sloppypar}

\begin{sloppypar}
\texttt{\textbf{ACESutil}\textit{.Name.\-}\textbf{a}\textit{ACESmajorVersionNumber.ACESutilMinorVersionNumber.ACES\-utilPatchVersionNumber}}
\end{sloppypar}


\subsection{ACES Core File Formats}
The ACES Core File Formats include the following:

\begin{listize}[-]
	\item SMPTE ST 268:2014 (DPX)
	\item SMPTE ST 2065-4:2013 (ACES Image Container File Layout)
	\item ACES Clip-level Metadata File (clip-level metadata sidecar)
	\item Academy-ASC Common LUT Format File (CLF)
\end{listize}

The SMPTE standard file formats are versioned according to SMPTE conventions.

ACES Clip Metadata files and Academy-ASC Common LUT Format files shall contain a metadata field that identifies the ACES System Version Number with which they conform, and the required Transform Identifiers of the transforms referenced therein (see ``Academy TB-2014-009'' and ``Academy S-2014-006'' for additional details).

\section{ACES Vendor-supplied components}
Certain ACES components, such as Input Transforms and concatenated RRT/ODTs, are shipped by ACES Product Partners and therefore are not constrained to ACES System release schedules. Other ACES components, such as ODTs and LMTs, are likely to be vendor-supplied in the future. Nonetheless, the versioning and naming requirements are the same as for ACES core components in that they must be identified as being compatible with a given ACES major system release.

To enable easier reading and parsing of Transform Identifiers, the sub-strings used for \texttt{\textit{NameSpace}} and \texttt{\textit{DeviceName}} should not contain spaces or periods and should also be limited to the ASCII character set.

\subsection{Input Transforms (IDTs)}

\texttt{\textbf{IDT}\textit{.NameSpace.DeviceName.}\textbf{a}\textit{ACESmajorVersionNumber.}\textbf{v}\textit{IDTversionNumber}}

The creator of the IDT shall be identified using the \texttt{\textit{NameSpace}}. When the creator of the transforms is the manufacturer of the camera then the device is not required to repeat the manufacturer name.  If the IDT creator is not the camera manufacturer, then the manufacturer name shall be prepended to the \texttt{\textit{DeviceName}}.

Example IDT Transform Identifiers using this format are: 
\begin{listize}
	\item \texttt{IDT.Sony.F65.a1.v1}
	\item \texttt{IDT.Arri.AlexaEI100T.a1.v2}
	\item \texttt{IDT.Dolby.ArriAlexa.a1.v1}
\end{listize}

\subsection{Look Transforms (LMTs)}

\texttt{\textbf{LMT}\textit{.Namespace.Name.}\textbf{a}\textit{ACESmajorVersionNumber.}\textbf{v}\textit{LMTversionNumber}}

The creator of the LMT shall be identified using the \texttt{\textit{NameSpace}}. The \texttt{\textit{NameSpace}} \texttt{Academy} is reserved for Academy-supplied LMTs. The \texttt{\textit{Name}} shall identify the purpose the LMT serves.

Example LMT Transform Identifiers using this format are: 
\begin{listize}
	\item \texttt{LMT.Academy.ACES\_0\_7\_1.a1.v1} (Academy-supplied v0.7.1 backwards-compatible transform)
	\item \texttt{LMT.ACME.BleachBypass.a1.v1} (ACME Transform, Inc.-supplied bleach bypass LMT compatible with ACES Version 1.0, in the Academy-ASC Common LUT Format)
\end{listize}

\subsection{Output Transforms (ODTs)}

\texttt{\textbf{ODT}\textit{.Namespace.OutputFormat.}\textbf{a}\textit{ACESmajorVersionNumber.}\textbf{v}\textit{ODTversionNumber}}

\texttt{\textit{NameSpace}} shall identify the creator of the ODT. The \texttt{\textit{NameSpace}} \texttt{Academy} is reserved for Academy-supplied ODTs. 

\texttt{\textit{OutputFormat}} fully describes the device and/or output data format of the ODT.

Example Transform Identifiers using this format are:
\begin{listize}
	\item \texttt{ODT.ACME.P3D60ProjectorSomeSpecialCalibration.a1.v1}
	\item \texttt{ODT.ACME.Rec709\_D60sim\_100nits\_dim.a1.v10}
\end{listize}

\subsection{Concatenated Reference Rendering Transform/Output Transforms (RRT/ODTs)} \label{sec:rrtodt}

\texttt{\textbf{RRTODT}\textit{.Namespace.OutputFormat.}\textbf{a}\textit{ACESmajorVersionNumber.}\textbf{v}\textit{ODTversionNumber}}

\texttt{\textit{Namespace}} shall identify the creator of the concatenated RRT/ODT. The \texttt{\textit{Namespace}} \texttt{Academy} is reserved for Academy-supplied concatenated RRT/ODTs.

\texttt{\textit{OutputFormat}} fully describes the device and/or output data format of the concatenated RRT/ODT, and should use the \texttt{\textit{OutputFormat}} associated with the ODT used in the concatenated transform. 

Example Transform Identifiers using this format are: 
\begin{listize}
	\item \texttt{RRTODT.ACME.P3d60Projector.a1.v1}
	\item \texttt{RRTODT.ACME.Rec709\_D60sim\_100nits\_dim.a1.v1}
	\item \texttt{RRTODT.ACME.P3D60ProjectorSomeSpecialCalibration.a1.v1}
	\item \texttt{RRTODT.ACME.Rec709\_d60sim\_8000nits.a1.v1}
\end{listize}

\section{Implementation Version Reporting}
ACES implementations shall report the version of the ACES System in use to at least the Minor Version Number.  Reporting of the Patch Version Number is optional. For more details, refer to ``Academy TB-2014-002.''

\section{ACES Pre-release Versions}
Pre-release versions of ACES, i.e., versions prior to Version 1.0, shall use the following version string format:

\texttt{\textbf{aPR}\textit{majorVersionNumber.MinorVersionNumber.PatchVersionNumber}}

Example:
\begin{listize}
	\item \texttt{ACESrelease.aPR0.7.1}
\end{listize}

The ``\texttt{\textbf{aPR}}'' designation indicates a version of ACES prior to the Version 1.0 release and the use of any transforms with this designation is deprecated.
\newcommand\element[9]{
    \def\tempa{#1}%
    \def\tempb{#2}%
    \def\tempc{#3}%
    \def\tempd{#4}%
    \def\tempe{#5}%
    \def\tempf{#6}%
    \def\tempg{#7}%
    \def\temph{#8}%
    \def\tempi{#9}%
    \elementcontinued
}


\newcommand\elementcontinued[3]{
    \subsection{\textbf{ {\texttt{\tempa}} }}

    \textbf{Description:}
    \begin{adjustwidth}{5mm}{} \vspace{-1.5mm}
        \tempb
    \end{adjustwidth}

    \textbf{Diagram:}
    \begin{figure}[H]
        \includegraphics[width=2.75in]{\tempc}
    \end{figure}

    \textbf{Type:}
    \begin{adjustwidth}{5mm}{} \vspace{-1.5mm}
        \texttt{\tempd}
    \end{adjustwidth}

    \textbf{Required or Optional:}
    \begin{adjustwidth}{5mm}{} \vspace{-1.5mm}
        \tempe
    \end{adjustwidth}

    \textbf{Occurrences:}
    \begin{adjustwidth}{5mm}{} \vspace{-1.5mm}
        Min: \tempf~Max: \tempg
    \end{adjustwidth}

    \textbf{Attributes:}
    \begin{adjustwidth}{5mm}{} \vspace{-1.5mm}
        Required: \texttt{\temph} \\
        Optional: \texttt{\tempi}
    \end{adjustwidth}

    \textbf{Parent:}
    \begin{adjustwidth}{5mm}{} \vspace{-1.5mm}
        #1
    \end{adjustwidth}

    \textbf{Children:}
    \begin{adjustwidth}{5mm}{} \vspace{-1.5mm}
        #2
    \end{adjustwidth}

    \textbf{Example:}
    \begin{adjustwidth}{5mm}{} \vspace{-1.5mm}
        #3
    \end{adjustwidth}
}

% acesMetadataFile
\element{aces:MetadataFile}
        {The top level element of an ACES Metadata File.  This element defines first level child elements.}
        {images/acesMetadataFile_xsd_Element_aces_acesMetadataFile.png}
        {xs:element}
        {Required}
        {1}{1}
        {\texttt{version="1.0", xmlns:aces="urn:ampas:aces:amf:v1.0"}}{\texttt{xmlns, xsi:schemeLocation}}
        {None}
        {\texttt{aces:pipeline, aces:archivedPipeline, aces:clipId, aces:amfInfo}}
        { \lstinline{<aces:MetadataFile} \\
        \lstinline{xmlns:aces="urn:ampas:aces:amf:v1.0"} \\
        \lstinline{xsi:schemaLocation="urn:ampas:aces:amf:v1.0 file:acesMetadataFile.xsd"} \\ \lstinline{xmlns:cdl="urn:ASC:CDL:v1.01"}\\
        \lstinline{xmlns:xsi="http://www.w3.org/2001/XMLSchema-instance"}\\
        \lstinline{version="1.0">}\\
        ... \\
        \lstinline{</aces:MetadataFile>}}

% amfInfo
\element{aces:amfInfo}
        {This element contains all the elements containing information about the AMF itself including date and time information, a description element, and a UUID element.}
        {images/acesMetadataFile_xsd_Element_aces_amfInfo.png}
        {aces:infoType}
        {Required}
        {1}{1}
        {none}{none}
        {\texttt{aces:MetadataFile}}
        {aces:author, aces:dateTime, aces:description, aces:uuid}
        {\lstinline{<aces:amfInfo>} \\
        ... \\
        \lstinline{</aces:amfInfo>}}

% archivedPipeline
 \element{aces:archivedPipeline}
        {This element contains all the elements describing an ACES viewing pipeline archived for historical purposes.}
        {images/acesMetadataFile_xsd_Element_aces_archivedPipeline.png}
        {aces:pipelineType}
        {Optional}
        {0}
        {unbounded}
        {none}{none}
        {\texttt{aces:MetadataFile}}
        {\texttt{aces:inputTransform, aces:lookTransform, aces:outputTransform, \\ aces:pipelineInfo}}
        {\lstinline{<aces:archivedPipeline>} \\
        ... \\
        \lstinline{</aces:archivedPipeline>}}

% clipId
\element{aces:clipId}
        {This optional element contains all the elements describing the location of the media files associated with the AMF.}
        {images/acesMetadataFile_xsd_Element_aces_clipId.png}
        {aces:clipIdType}
        {Optional}{0}{1}
        {none}{none}{\texttt{aces:MetadataFile}}
        {\texttt{aces:clipName, aces:file, aces:sequence, aces:uuid}}
        {\lstinline{<aces:clipId>} \\
        ... \\
        \lstinline{</aces:clipId>}}

% pipeline
\element{aces:pipeline}
        {This element contains all the elements describing the ACES viewing pipeline.}
        {images/acesMetadataFile_xsd_Element_aces_pipeline.png}
        {aces:pipelineType}
        {Required}{1}{1}
        {none}{none}
        {\texttt{aces:MetadataFile}}
        {\texttt{aces:inputTransform, aces:lookTransform, aces:outputTransform, \\ aces:pipelineInfo}}
        {\lstinline{<aces:pipeline>} \\
        ... \\
        \lstinline{</aces:pipeline>}}

% emailAddress
\element{aces:emailAddress}
        {This element used to communicate the AMF author's email address.}
        {images/acesMetadataFile_xsd_Element_aces_emailAddress.png}
        {aces:emailAddressType}
        {Required}{1}{1}
        {none}{none}
        {\texttt{aces:author}}
        {None}
        {\lstinline{<aces:emailAddress>joe@onset.com</aces:emailAddress>}}

% name
\element{aces:name}
        {This element is used to communicate the name of the AMF author.}
        {images/acesMetadataFile_xsd_Element_aces_name.png}
        {\texttt{xs:string}}
        {Required}{1}{1}
        {None}{None}{\texttt{aces:author}}{None}{\lstinline{<aces:name>Joe Onset</aces:name>}}

% aces:fromCdlTransformWorkingSpace
\element{aces:fromCdlWorkingSpace}
        {This element contains all the elements describing the transform used to convert from the working color space in which an ASC-CDL is applied to ACES 2065-1.}
        {images/acesMetadataFile_xsd_Element_aces_fromCdlWorkingSpace.png}
        {\texttt{aces:workingSpaceTransformType}}
        {Required}{1}{1}
        {None}{None}
        {\texttt{aces:cdlWorkingSpace}}
        {aces:description, aces:hash, aces:transformId}
        {\lstinline{<aces:fromCdlWorkingSpace>} \\
        ... \\
        \lstinline{</aces:fromCdlWorkingSpace>}}

% aces:toCdlWorkingSpace
\element{aces:toCdlWorkingSpace}
        {This element contains all the elements describing the transform used to convert from ACES 2065-1 to the working color space in which a ASC-CDL transform is applied.  This transform shall be included when the working color space for the ASC-CDL Transform is not a working color space described in one of the Color Space Conversion transform included in the ACES core transforms.  When the working color space for the ASC-CDL Transform is a working color space described in one of the Color Space Conversion transform included in the ACES core transforms, the \texttt{aces:toCdlWorkingSpace} is optional.}
        {images/acesMetadataFile_xsd_Element_aces_toCdlWorkingSpace.png}
        {\texttt{aces:workingSpaceTransformType}}
        {Optional}{0}{1}
        {None}{None}
        {\texttt{aces:cdlWorkingSpace}}
        {\texttt{aces:description, aces:hash, aces:transformId}}
        {\lstinline{<aces:toCdlWorkingSpace>} \\
        ... \\
        \lstinline{</aces:toCdlWorkingSpace>}}

% clipName
\element{aces:clipName}
        {This element is used to communicate the clip name associated with the media files.}
        {images/acesMetadataFile_xsd_Element_aces_clipName.png}
        {\texttt{xs:string}}
        {Required}{1}{1}
        {None}{None}{\texttt{aces:clipId}}{None}{\lstinline{<aces:clipName>A001C012</aces:clipId>}}

% file
\element{aces:file}
        {This element is used to communicate the name of the media file.  Care should be taken when using the file name as an identifier as file locations and names typically change during production and post-production.}
        {images/acesMetadataFile_xsd_Element_aces_file.png}
        {\texttt{xs:anyURI}}
        {Choice of \texttt{aces:file}, \texttt{aces:sequence} or \texttt{aces:uuid} is required}{1}{1}
        {None}{None}{\texttt{aces:clipId}}{None}
        {\lstinline{<aces:file>file:///foo.mxf</aces:file>}}

% sequence
\element{aces:sequence}
        {This element is used to communicate the file sequence information associated with the media files.  The file sequence includes an index indicated by the \texttt{idx} attribute (e.g. \#) that is used to denote the location of frame numbers within the sequence string.  The \texttt{min} and \texttt{max} attributes are used to indicate the minimum frame number and maximum frame number of the sequence.  For example, if the sequence string is \texttt{movieFrame\#\#\#\#.exr} and attributes of \texttt{aces:sequence} are \texttt{idx='\#'}, \texttt{min='0'} and \texttt{min='1000'} the the media files associated with the AMF would be the frames numbered \texttt{movieFrame0000.exr} through \texttt{movieFrame1000.exr}}
        {images/acesMetadataFile_xsd_Element_aces_sequence.png}
        {\texttt{aces:sequenceType}}
        {Choice of \texttt{aces:file}, \texttt{aces:sequence} or \texttt{aces:uuid} is required}{1}{1}
        {\texttt{idx}, \texttt{min}, \texttt{max}}{None}{\texttt{aces:clipId}}{None}
        {\lstinline{<aces:sequence idx="#" min="1" max="240">A01_C012_AE0306_###.exr</aces:sequence>}}

% aces:clipIdType / aces:uuid
\element{aces:clipIdType / aces:uuid}
		{This element is used to communicate a UUID associated with the media files referred to in the ClipID.}
		{images/acesMetadataFile_xsd_Element_aces_uuid.png}
		{\texttt{dcml:UUIDType}}
		{Choice of \texttt{aces:file}, \texttt{aces:sequence} or \texttt{aces:uuid} is required}{1}{1}
		{None}{None}
		{\texttt{aces:clipId}}{None}
		{\lstinline{<aces:uuid>urn:uuid:797c7cd8-4eb1-4f67-afce-af2b0a1d0285</aces:uuid>}}

% creationDateTime
\element{aces:creationDateTime}
		{This element is used to communicate the creation date and time of an AMF file or an ACES pipeline.}
		{images/acesMetadataFile_xsd_Element_aces_creationDateTime.png}
		{\texttt{xs:dateTime}}
		{Required}{1}{1}
		{None}{None}
		{\texttt{aces:dateTime}}{None}
		{\lstinline{<aces:creationDateTime>2020-11-10T13:20:00Z</aces:creationDateTime>}}

% modificationDateTime
\element{aces:modificationDateTime}
		{This element is used to communicate the most recent modification date and time of an AMF file or an ACES pipeline.}
		{images/acesMetadataFile_xsd_Element_aces_modificationDateTime.png}
		{\texttt{xs:dateTime}}
		{Required}{1}{1}
		{None}{None}
		{\texttt{aces:dateTime}}{None}
		{\lstinline{<aces:modificationDateTime>2020-11-10T13:20:00Z</aces:modificationDateTime>}}

% author
\element{aces:author}
		{This element contains all the elements describing the AMF author information.}
		{images/acesMetadataFile_xsd_Element_aces_author.png}
		{\texttt{xs:sequence}}
		{Optional}{1}{unbounded}
		{None}{None}
		{\texttt{aces:amfInfo}}{\texttt{aces:name, aces:emailAddress}}
		{\lstinline{<aces:author>} \\
        ... \\
        \lstinline{</aces:author>}}

% dateTime
\element{aces:dateTime}
		{This element contains all the elements describing the date and time of the creation and modification of the AMF.}
		{images/acesMetadataFile_xsd_Element_aces_dateTime.png}
		{\texttt{xs:sequence}}
		{Required}{1}{1}
		{None}{None}
		{\texttt{aces:amfInfo}}{\texttt{aces:creationDateTime, modificationDateTime}}
		{\lstinline{<aces:dateTime>} \\
        ... \\
        \lstinline{</aces:dateTime>}}

% description
\element{aces:description}
		{This element is used to communicate description information for an AMF file, an ACES pipeline, or various ACES viewing transforms.}
		{images/acesMetadataFile_xsd_Element_aces_description.png}
		{\texttt{xs:string}}
		{Optional}{0}{1}
		{None}{None}
		{\texttt{aces:amfInfo, aces:piplineInfo, aces:toCdlWorkingSpace,\\ aces:fromCdlWorkingSpace, aces:inputTransform,\\
		  aces:lookTransform, aces:outputTransform, aces:outputDeviceTransform, \\
		  aces:referenceRenderingTransform}}{None}
		{\lstinline{<aces:description>Example Movie</aces:description>}\\
		 \lstinline{<aces:description>Technical Grade</aces:description>}}

% aces:infoType / aces:uuid
\element{aces:infoType / aces:uuid}
		{This element is used to communicate a UUID associated with the AMF or an ACES pipeline.}
		{images/acesMetadataFile_xsd_Element_aces_uuid.png}
		{\texttt{dcml:UUIDType}}
		{Optional}{0}{1}
		{None}{None}
		{\texttt{aces:amfInfo, aces:pipelineInfo}}{None}
		{\lstinline{<aces:uuid>urn:uuid:797c7cd8-4eb1-4f67-afce-af2b0a1d0285</aces:uuid>}}

%  aces:inverseOutputDeviceTransform
\element{aces:inverseOutputDeviceTransform}
        {This element contains all the elements describing the transforms associated with an inverse output device transform used to convert output referred images to ACES.}
        {images/acesMetadataFile_xsd_Element_aces_inverseOutputDeviceTransform.png}
        {\texttt{aces:inverseOutputDeviceTransformType}}
        {Required}{1}{1}
        {None}{None}
        {aces:description, aces:hash, aces:transformId}
        {\lstinline{<aces:inverseOutputDeviceTransform>} \\
        ... \\
        \lstinline{</aces:inverseOutputDeviceTransform>}}

%  aces:inverseOutputTransform
\element{aces:inverseOutputTransform}
        {This element contains all the elements describing the transforms associated with an inverse output  transform used to convert output referred images to ACES.}
        {images/acesMetadataFile_xsd_Element_aces_inverseOutputTransform.png}
        {\texttt{aces:inverseOutputTransformType}}
        {Required}{1}{1}
        {None}{None}
        {aces:description, aces:hash, aces:transformId}
        {\lstinline{<aces:inverseOutputTransform>} \\
        ... \\
        \lstinline{</aces:inverseOutputTransform>}}

%  aces:inverseReferenceRenderingTransform
\element{aces:inverseReferenceRenderingTransform}
        {This element contains all the elements describing the transforms associated with an inverse reference rendering transform used to convert output referred images to ACES.}
        {images/acesMetadataFile_xsd_Element_aces_inverseReferenceRenderingTransform.png}
        {\texttt{aces:inverseReferenceRenderingTransformType}}
        {Required}{1}{1}
        {None}{None}
        {aces:description, aces:hash, aces:transformId}
        {\lstinline{<aces:inverseReferenceRenderingTransform>} \\
        ... \\
        \lstinline{</aces:inverseReferenceRenderingTransform>}}

% aces:inputTransformType / aces:transformId
\element{aces:inputTransformType / aces:transformId}
        {This element is used to communicate the transformID of an ACES Input Transform that transforms images encoded in a color space of a camera native file to ACES 2065-1.  For more information on transformIDs see S-2014-002 Academy Color Encoding System -- Versioning system.  Valid transforms for this element are Input Transforms.  The element is restricted to enforce the use of transformIDs that follow the IDT naming conventions established in the versioning system specification.  As noted in the versioning system specification, manufacturer and user created transforms shall be assigned a transformID according to patterns established in the document.}
        {images/acesMetadataFile_xsd_Element_aces_transformId.png}
        {\texttt{aces:tnInputTransform}}
        {Required}{1}{1}
        {None}{None}
        {\texttt{aces:inputTransform}}{None}
        {\lstinline{<aces:transformId>} \\
        \lstinline{urn:ampas:aces:transformId:v1.5:IDT.Sony.F65.a1.v1} \\
        \lstinline{</aces:transformId>}}

% aces:inverseOutputDeviceTransformType / aces:transformId
\element{aces:inverseOutputDeviceTransformType / aces:transformId}
        {This element is used to communicate the transformID of an ACES Inverse Output Device Transform that transforms images encoded in an output referred color space to OCES.  For more information on transformIDs see S-2014-002 Academy Color Encoding System -- Versioning system.  Valid transforms for this element are Input Transforms.  The element is restricted to enforce the use of transformIDs that follow the InvODT naming conventions established in the versioning system specification.  As noted in the versioning system specification, manufacturer and user created transforms shall be assigned a transformID according to patterns established in the document.}
        {images/acesMetadataFile_xsd_Element_aces_transformId_2.png}
        {\texttt{aces:tnInverseDeviceOutputTransform}}
        {Required}{1}{1}
        {None}{None}
        {\texttt{aces:inputTransform}}{None}
        {\lstinline{<aces:transformId>} \\
        \lstinline{urn:ampas:aces:transformId:v1.5:InvODT.Academy.Rec709_100nits_dim.a1.0.3} \\
        \lstinline{</aces:transformId>}}

% aces:inverseOutputTransformType / aces:transformId
\element{aces:inverseOutputTransformType / aces:transformId}
        {This element is used to communicate the transformID of an ACES Inverse Output Transform that transforms images encoded in an output referred color space to ACES 2065-1.  For more information on transformIDs see S-2014-002 Academy Color Encoding System -- Versioning system.  Valid transforms for this element are Input Transforms.  The element is restricted to enforce the use of transformIDs that follow the InvRRTODT naming conventions established in the versioning system specification.  As noted in the versioning system specification, manufacturer and user created transforms shall be assigned a transformID according to patterns established in the document.}
        {images/acesMetadataFile_xsd_Element_aces_transformId_2.png}
        {\texttt{aces:tnInverseReferenceRenderingTransform}}
        {Required}{1}{1}
        {None}{None}
        {\texttt{aces:inputTransform}}{None}
        {\lstinline{<aces:transformId>}\\
        \lstinline{urn:ampas:aces:transformId:v1.5:InvRRT.a1.0.3}\\
        \lstinline{</aces:transformId>}}

% aces:inverseReferenceRenderingTransformType / aces:transformId
\element{aces:inverseReferenceRenderingTransformType / aces:transformId}
        {This element is used to communicate the transformID of an ACES Inverse Reference Rendering Transform that transforms images encoded in OCES color space to ACES 2065-1.  For more information on transformIDs see S-2014-002 Academy Color Encoding System -- Versioning system.  Valid transforms for this element are Input Transforms.  The element is restricted to enforce the use of transformIDs that follow the InvRRT naming conventions established in the versioning system specification.  As noted in the versioning system specification, manufacturer and user created transforms shall be assigned a transformID according to patterns established in the document.}
        {images/acesMetadataFile_xsd_Element_aces_transformId_3.png}
        {\texttt{aces:tnInverseReferenceRenderingTransform}}
        {Required}{1}{1}
        {None}{None}
        {\texttt{aces:inputTransform}}{None}
        {\lstinline{<aces:transformId>} \\
        \lstinline{urn:ampas:aces:transformId:v1.5:InvRRTODT.Academy.Rec2020_1000nits_15nits_ST2084.a1.1.0}\\
        \lstinline{</aces:transformId>}}

%aces:cdlWorkingSpace
\element{aces:cdlWorkingSpace}
        {This element contains all the elements describing the transforms used to convert to and from the working color space in which a ASC-CDL transform is applied. This element allows for CDLs to be applied in color spaces other than ACES RGB, since CDLs cannot contain ACES transforms themselves. The input and output of the parent \texttt{<lookTransform>} element is still ACES RGB per SMPTE ST.2065-1. }
        {images/acesMetadataFile_xsd_Element_aces_cdlWorkingSpace.png}
        {\texttt{aces:cdlWorkingSpaceType}}
        {Required}{1}{1}
        {None}{None}
        {\texttt{aces:lookTransform}}{\texttt{aces:fromCdlWorkingSpace, aces:toCdlWorkingSpace}}
        {\lstinline{<aces:cdlWorkingSpace>} \\
        ... \\
        \lstinline{</aces:cdlWorkingSpace>}}

% aces:lookTransformType / aces:transformId
\element{aces:lookTransformType / aces:transformId}
        {This element is used to communicate the transformID of an ACES Look Transform.  For more information on transformIDs see S-2014-002 Academy Color Encoding System -- Versioning system.  Valid transforms for this element are Look Transforms (LMT).  The element is restricted to enforce the use of transformIDs that follow the LMT naming conventions established in the versioning system specification.  As noted in the versioning system specification, manufacturer and user created transforms shall be assigned a transformID according to patterns established in the document.}
        {images/acesMetadataFile_xsd_Element_aces_transformId_8.png}
        {\texttt{aces:tnLookTransform}}
        {Choice of \texttt{aces:transformId, cdl:ColorCorrectionRef}, or \texttt{cdl:SOPNode} and \\ \texttt{cdl:SatNode} required}
        {0}{1}
        {None}{None}
        {\texttt{aces:lookTransform}}
        {None}
        {\lstinline{<aces:transformId>}\\
        \lstinline{urn:ampas:aces:transformId:v1.5:LMT.ACME.BleachBypass.a1.v1</aces:transformId>}\\
        \lstinline{</aces:transformId>}
        }

% aces:lookTransformType / aces:uuid
\element{aces:lookTransformType / aces:uuid}
		{This element is used to communicate a UUID associated with externally referenced ASC-CDLs.}
		{images/acesMetadataFile_xsd_Element_aces_uuid.png}
		{\texttt{dcml:UUIDType}}
		{Optional}{0}{1}
		{None}{None}
		{\texttt{aces:lookTransform}}{None}
		{\lstinline{<aces:uuid>urn:uuid:797c7cd8-4eb1-4f67-afce-af2b0a1d0285</aces:uuid>}}

% aces:outputDeviceTransformType / aces:transformId
\element{aces:outputDeviceTransformType / aces:transformId}
        {This element is used to communicate the transformID of the ACES Output Device Transform.  For more information on transformIDs see S-2014-002 Academy Color Encoding System -- Versioning system.  Valid transforms for this element are Output Transforms (ODT).  The element is restricted to enforce the use of transformIDs that follow the ODT naming conventions established in the versioning system specification.  As noted in the versioning system specification, manufacturer and user created transforms shall be assigned a transformID according to patterns established in the document.}
        {images/acesMetadataFile_xsd_Element_aces_transformId_5.png}
        {\texttt{aces:tnOutputDeviceTransform}}
        {Required}{1}{1}
        {None}{None}
        {\texttt{aces:outputDeviceTransform}}{None}
        {\lstinline{<aces:transformId>urn:ampas:aces:transformId:v1.5:ODT.Academy.P3D60_48nits.a1.0.0</aces:transformId>}}

% aces:outputDeviceTransform
\element{aces:outputDeviceTransform}
        {This element contains all the elements containing information about the ACES Output Device Transform for a given ACES viewing pipeline.}
        {images/acesMetadataFile_xsd_Element_aces_outputDeviceTransform.png}
        {aces:outputDeviceTransformType}
        {Choice of \texttt{aces:transformId} or \texttt{aces:outputDeviceTransform} and \\  \texttt{aces:referenceRenderingTransform} required}{1}{1}
        {None}{None}{\texttt{aces:OutputTransform}}
        {\texttt{aces:description, aces:hash, aces:transformId}}
        {\lstinline{<aces:outputDeviceTransform>} \\
        ... \\
        \lstinline{</aces:outputDeviceTransform>}}

% aces:referenceRenderingTransform
\element{aces:referenceRenderingTransform}
        {This element contains all the elements containing information about the ACES Reference Rendering Transform for a given ACES viewing pipeline.}
        {images/acesMetadataFile_xsd_Element_aces_referenceRenderingTransform.png}
        {\texttt{aces:referenceRenderingTransformType}}
        {Choice of \texttt{aces:transformId} or \texttt{aces:outputDeviceTransform} and \\  \texttt{aces:referenceRenderingTransform} required}
        {1}{1}{None}{None}
        {\texttt{aces:OutputTransform}}
        {\texttt{aces:description, aces:hash, aces:transformId}}
        {\lstinline{<aces:referenceRenderingTransform>} \\
        ... \\
        \lstinline{</aces:referenceRenderingTransform>}}

% aces:outputTransformType / aces:transformId
\element{aces:outputTransformType / aces:transformId}
        {This element is used to communicate the transformID of the ACES Output Transform.  For more information on transformIDs see S-2014-002 Academy Color Encoding System -- Versioning system.  Valid transforms for this element are Output Transforms (RRTODT).  The element is restricted to enforce the use of transformIDs that follow the RRTODT naming conventions established in the versioning system specification.  As noted in the versioning system specification, manufacturer and user created transforms shall be assigned a transformID according to patterns established in the document.}
        {images/acesMetadataFile_xsd_Element_aces_transformId_6.png}
        {\texttt{aces:tnOutputTransform}}
        {Required}{1}{1}
        {None}{None}
        {\texttt{aces:outputTransform}}
        {None}
        {\lstinline{<aces:transformId>}\\
        \lstinline{urn:ampas:aces:transformId:v1.5:RRTODT.Academy.Rec2020_1000nits_15nits_HLG.a1.1.0}\\
        \lstinline{</aces:transformId>}}

% aces:systemVersion
\element{aces:systemVersion}
        {This element contains all the elements containing information about the ACES version number associated with the ACES viewing pipeline.}
        {images/acesMetadataFile_xsd_Element_aces_systemVersion.png}
        {\texttt{aces:versionType}}
        {Required}
        {1}{1}
        {None}{None}
        {\texttt{aces:pipelineInfo}}{\texttt{aces:majorVersion, aces:minorVersion, aces:patchVersion}}
        {\lstinline{<aces:systemVersion>} \\
        ... \\
        \lstinline{</aces:systemVersion>}}

% aces:inputTransform
\element{aces:inputTransform}
        {This element contains all the elements containing information about the ACES input transform for a given ACES viewing pipeline.  The required \texttt{applied} attribute is used to indicate if the ACES input transform indicated has been applied to the media files or not.  If \texttt{applied="true"} the media files shall be encoded as according to SMPTE ST 2065-1. If \texttt{applied="false"} the media files may be transcoded to ACES using the transform indicated in the child element \texttt{transformId}.}
        {images/acesMetadataFile_xsd_Element_aces_inputTransform.png}
        {\texttt{aces:inputTransformType}}
        {Optional}{0}{1}
        {\texttt{applied (xs:boolean)}}{None}
        {\texttt{aces:pipeline}, \texttt{aces:archivedPipeline}}
        {\texttt{aces:description, aces:hash, aces:transformId}}
        {\lstinline{<aces:inputTransform>} \\
        ... \\
        \lstinline{</aces:inputTransform>}}

% aces:lookTransform
 \element{aces:lookTransform}
        {This element contains a look transform (LMT) for a given ACES viewing pipeline.  If the AMF includes multiple \texttt{<lookTransform>} elements, they shall be applied in the order in which they are written in the AMF (top to bottom).

        The required \texttt{applied} attribute is used to indicate if the ACES look transform has been applied to the media files or not.  If \texttt{applied="true"}, the media files shall have the look transform "baked" into the image data (but is still included for diagnostic purposes). If \texttt{applied="false"}, the media files shall not have the look transform "baked" into the image data.

        The input values and output values of ACES Look Transforms are ACES 2065-1. This is to avoid linking the Look Transforms to project specific working spaces. Look Transforms may convert ACES 2065-1 to a more appropriate working space for internal look application. Care should be taken when building Look Transforms as 3D LUTs, given Look Transforms input and output values are linear. In practice, smart implementations may modify the Look Transform to avoid unnecessary conversions within the context of an ACES pipeline as long as the results match those specified by the transforms in the AMF.

        ASC-CDL does not have a mechanism to convert to a non-linear working space appropriate for the application of ASC-CDL values. For this reason, the \texttt{<aces:cdlTransformWorkingSpace>} element can be used to indicate the working space via transformsIDs in which ASC-CDL values are to be applied.}
        {images/acesMetadataFile_xsd_Element_aces_lookTransform.png}
        {\texttt{aces:lookTransformType}}
        {Optional}{0}{unbounded}
        {\texttt{applied (xs:boolean)}}{None}
        {\texttt{aces:pipeline}, \texttt{aces:archivedPipeline}}
        {\texttt{aces:description, aces:hash, aces:CdlWorkingSpace, aces:transformId,\\  cdl:ColorCorrectionRef, aces:uuid, cdl:SOPNode, cdl:SatNode}}
        {\lstinline{<aces:lookTransform>} \\
        ... \\
        \lstinline{</aces:lookTransform>}}

% aces:outputTransform
 \element{aces:outputTransform}
        {This element contains all the elements containing information about the ACES output transform for a given ACES viewing pipeline.}
        {images/acesMetadataFile_xsd_Element_aces_outputTransform.png}
        {\texttt{aces:outputTransformType}}
        {Required}{1}{1}
        {None}{None}
        {\texttt{aces:pipeline}, \texttt{aces:archivedPipeline}}
        {\texttt{aces:description, aces:hash, aces:outputDeviceTransform, \\
        aces:referenceRenderingTransform, aces:transformId}}
        {\lstinline{<aces:outputTransform>} \\
        ... \\
        \lstinline{</aces:outputTransform>}}

% aces:pipelineInfo
\element{aces:pipelineInfo}
        {This element contains all the elements containing metadata information such as description, author, date and time, etc. for a given ACES viewing pipeline. }
        {images/acesMetadataFile_xsd_Element_aces_pipelineInfo.png}
        {\texttt{aces:piplineInfoType}}
        {Required}{1}{1}
        {None}{None}
        {\texttt{aces:pipeline, aces:archivedPipeline}}
        {\texttt{aces:author, aces:dateTime, aces:description, aces:systemVersion,\\ aces:uuid}}
        {\lstinline{<aces:pipelineInfo>} \\
        ... \\
        \lstinline{</aces:pipelineInfo>}}

% aces:referenceRenderingTransform / aces:transformId
\element{aces:referenceRenderingTransform / aces:transformId}
        {This element is used to communicate the transformID of the ACES Reference Rendering Transform.  For more information on transformIDs see S-2014-002 Academy Color Encoding System -- Versioning system.  Valid transforms for this element are Reference Rendering Transform (RRT).  The element is restricted to enforce the use of transformIDs that follow the RRT naming conventions established in the versioning system specification.}
        {images/acesMetadataFile_xsd_Element_aces_transformId_4.png}
        {\texttt{aces:tnReferenceRenderingTransform}}
        {Required}{1}{1}
        {None}{None}
        {\texttt{aces:outputTransform}}
        {None}
        {\lstinline{<aces:transformId>}\\
        \lstinline{urn:ampas:aces:transformId:v1.5:urn:ampas:aces:transformId:v1.5:RRT.a1.0.0}\\
        \lstinline{</aces:transformId>}}

% aces:hash
\element{aces:hash}
        {This element is used to communicate the cryptographic hash for a transform referenced by the AMF.}
        {images/acesMetadataFile_xsd_Element_aces_hash.png}
        {\texttt{aces:hashType}}
        {Optional}{0}{1}
        {\texttt{algorithm (restricted xs:anyURI)}}{None}
        {\texttt{aces:inputTransform, aces:lookTransform, aces:outputDeviceTransform,\\ aces:outputTransform, aces:referenceRenderingTransform}}{None}
        {
        \lstinline{<aces:hash algorithm="http://www.w3.org/2001/04/xmlenc#sha256">c81af4fb4a22ee}\\      \lstinline{0353308e4582708951df4682bf73f838c24bf44e585fc3bb61</aces:hash>}
        }

% aces:majorVersion
\element{aces:majorVersion}
        {This element contains information on the ACES system major version number associated with an ACES viewing pipeline.  If the system reading the AFM has not implemented the major version specified the system shall indicate that the major version of the system and AMF do not match and produce an error.}
        {images/acesMetadataFile_xsd_Element_aces_majorVersion.png}
        {\texttt{aces:singleDigitType}}
        {Required}{1}{1}
        {None}{None}
        {/texttt{aces:systemVersion}}{None}
        {\lstinline{<aces:majorVersion>1</aces:majorVersion>}}

% aces:minorVersion
\element{aces:minorVersion}
        {This element contains information on the ACES system minor version number associated with an ACES viewing pipeline.  If the system reading the AFM has not implemented the minor version specified the system shall indicate that the minor version of the system and AMF do not match and produce an error.}
        {images/acesMetadataFile_xsd_Element_aces_minorVersion.png}
        {\texttt{aces:singleDigitType}}
        {Required}{1}{1}
        {None}{None}
        {/texttt{aces:systemVersion}}{None}
        {\lstinline{<aces:minorVersion>2</aces:minorVersion>}}

% aces:minorVersion
\element{aces:patchVersion}
        {This element contains information on the ACES system patch version number associated with an ACES viewing pipeline.  If the system reading the AFM has not implemented the patch version specified the system shall indicate that the patch version of the system and AMF do not match with a warning and fall back to the most recent patch version available.}
        {images/acesMetadataFile_xsd_Element_aces_patchVersion.png}
        {\texttt{aces:singleDigitType}}
        {Required}{1}{1}
        {None}{None}
        {/texttt{aces:systemVersion}}{None}
        {\lstinline{<aces:patchVersion>2</aces:patchVersion>}}

% aces:workingSpaceTransformType / aces:transformId
\element{aces:workingSpaceTransformType / aces:transformId}
        {This element is used to communicate the transformID of the ACES Color Space Conversion Transform used to convert to or from the Look Transform working space.  For more information on transformIDs see S-2014-002 Academy Color Encoding System -- Versioning system.  Valid transforms for this element are Reference Rendering Transform (ACEScsc).  The element is restricted to enforce the use of transformIDs that follow the ACEScsc naming conventions established in the versioning system specification.}
        {images/acesMetadataFile_xsd_Element_aces_transformId_7.png}
        {\texttt{aces:tnColorSpaceConversionTransform}}
        {Required}{1}{1}
        {None}{None}
        {\texttt{aces:toCdlWorkingSpace, aces:fromCdlWorkingSpace}}
        {None}
        {\lstinline{<aces:transformId>}\\
        \lstinline{urn:ampas:aces:transformId:v1.5:ACEScsc.Academy.ACEScct_to_ACES.a1.0.3}\\
        \lstinline{</aces:transformId>}}

% cdl:SOPNode
\element{cdl:SOPNode}
        {This element is imported from the ASC-CDL schema (\texttt{ASC-CDL\_schema\_v1.01.xsd}).  It defines a Slope, Offset, Power node.  \texttt{<cdl:SOPNode>} may be substituted with \texttt{<cdl:ASC\_SOP>}.  See the ASC-CDL documentation for more information on its usage.}
        {images/ASC-CDL_schema_v1_01_xsd_Element_cdl_SOPNode.png}
        {\texttt{cdl:SOPNodeType}}
        {Choice of \texttt{aces:transformId, cdl:ColorCorrectionRef}, or \texttt{cdl:SOPNode} and \\ \texttt{cdl:SatNode} required}
        {1}{1}
        {None}{None}
        {\texttt{aces:lookTransform}}
        {\texttt{cdl:Description, cdl:Offset, cdl:Power, cdl:Slope}}
        {\lstinline{<cdl:ASC_SOP>}\\
                    \lstinline{<cdl:Slope>2.0 2.0 2.0</cdl:Slope>}\\
                    \lstinline{<cdl:Offset>0.1 0.1 0.1</cdl:Offset>}\\
                    \lstinline{<cdl:Power>1 1 1</cdl:Power>}\\
                    \lstinline{</cdl:ASC_SOP>}}

% cdl:SATNode
\element{cdl:SATNode}
        {This element is imported from the ASC-CDL schema (\texttt{ASC-CDL\_schema\_v1.01.xsd}).  It defines a saturation node.  \texttt{<cdl:SATNode>} may be substituted with \texttt{<cdl:ASC\_SAT>}.  See the ASC-CDL documentation for more information on its usage.}
        {images/ASC-CDL_schema_v1_01_xsd_Element_cdl_SatNode.png}
        {\texttt{cdl:SATNodeType}}
        {Choice of \texttt{aces:transformId, cdl:ColorCorrectionRef}, or \texttt{cdl:SOPNode} and \\ \texttt{cdl:SatNode} required}
        {1}{1}
        {None}{None}
        {\texttt{aces:lookTransform}}
        {\texttt{cdl:Description, cdl:Saturation}}
        {\lstinline{<cdl:ASC_SAT>}\\
                    \lstinline{<cdl:Saturation>1.0</cdl:Saturation>}\\
                    \lstinline{</cdl:ASC_SAT>}}

% cdl:ColorCorrectionRef
\element{cdl:ColorCorrectionRef}
        {This element is imported from the ASC-CDL schema (\texttt{ASC-CDL\_schema\_v1.01.xsd}).  It defines a Color Correction Reference node for referencing ASC-CDL values that exist in transport containers other than the AMF.  \texttt{<cdl:ColorCorrectionRef>} may be substituted with \texttt{<cdl:ASC\_CC\_XML>}.  It is recommended the \texttt{cdl:InputDescription} and \texttt{cdl:ViewingDescription} nodes not be used as this information is included in other locations within the AMF.  See the ASC-CDL documentation for more information on the usage of this node.}
        {images/ASC-CDL_schema_v1_01_xsd_Element_cdl_ColorCorrectionRef.png}
        {\texttt{cdl:ColorCorrectionRefType}}
        {Choice of \texttt{aces:transformId, cdl:ColorCorrectionRef}, or \texttt{cdl:SOPNode} and \\ \texttt{cdl:SatNode} required}
        {1}{1}{\texttt{ref} (xs:anyURI)}{None}
        {\texttt{aces:lookTransform}}
        {\texttt{cdl:Description, cdl:InputDescription, cdl:ViewingDescription}}
        {\lstinline{<cdl:ColorCorrectionRef ref="file:///foo.edl>}\\
                    \lstinline{<cdl:Description>Technical Grade</cdl:Description>}\\
                    \lstinline{</cdl:ColorCorrectionRef>}}

% This file contains the content for a main section
\regularsectionformat
%% Modify below this line %%
\chapter{Implementation Notes}
\label{sec:implementation}

% -- Bit Depth Behavior --
\section{Bit Depth}

\subsection{Processing Precision} \label{sec:process-precision}
All processing shall be performed using 32-bit floating-point values. The values of the \xml{inBitDepth} and 
\xml{outBitDepth} attributes shall not affect the quantization of color values. 

\note{For some hardware devices, 32-bit float processing might not be possible. In such instances, processing should be performed at the highest precision available. Because CLF permits complex series of discrete operations, CLF LUT files are unlikely to run on hardware devices without some form of pre-processing. Any pre-processing to prepare a CLF for more limited hardware applications should adhere to the processing precision requirements.}

\subsection{Input To and Output From a \texttt{ProcessList}} \label{sec:processList-in-out}
Applications often support multiple pixel formats (e.g. 8i, 10i, 16f, 32f, etc.). Often the actual pixel format to be processed may not agree with the \xml{inBitDepth} of the first \xml{ProcessNode} or the \xml{outBitDepth} of the last \xml{ProcessNode}. (Note that the \xml{ProcessList} element itself does not contain global \xml{inBitDepth} or \xml{outBitDepth} attributes.) Therefore, in some cases an application may need to rescale a given \xml{ProcessNode} to be appropriate for the actual image data being processed.

For example, if the last \xml{ProcessNode} in a \xml{ProcessList} is a \xml{LUT1D} with an \xml{outBitDepth} of 12i, it indicates that the LUT \xml{Array} values are scaled relative to 4095. If the application wants to produce floating-point pixel values, it should therefore divide the LUT \xml{Array} values by 4095 before processing the pixels (according to \ref{sec:scaling}). Likewise, if the \xml{outBitDepth} was instead 32f and the application wants to produce \xml{12i} pixel values, it should multiply the LUT \xml{Array} values by 4095. (Note that in this case, since the result of the computations may exceed 4095 and the application wants to produce 12-bit integer output, the application would want to clamp, round, and quantize the value.)


\subsection{Input To and Output From a \texttt{ProcessNode}}
In order to ensure the scaling of parameter values of all \xml{ProcessNode}s in a \xml{ProcessList} are consistent, the \xml{inBitDepth} of each \xml{ProcessNode} must match the \xml{outBitDepth} of the previous \xml{ProcessNode} (if any). 

Please note that an integer \xml{inBitDepth} or \xml{outBitDepth} of a \xml{ProcessNode} does not indicate that any clamping or quantization should be done. These attributes are strictly used to indicate the scaling of parameter and array values. As discussed in \ref{sec:process-precision}, processing precision shall be floating-point.

Furthermore, because the processing precision is intended to be floating-point, the \xml{inBitDepth} and \xml{outBitDepth} only control the scaling of parameter and array values and do not impose range limits. For example, even if the \xml{outBitDepth} of a LUT \xml{Array} is 12i, it does not mean that the \xml{Array} values must be limited to [0,4095] or that they must be integer values. It simply means that in order to rescale to 32f that a scale factor of $1/4095$ should be used (as per \ref{sec:scaling}).

Because processing within a \xml{ProcessList} should be done at floating-point precision, applications may optionally want to rescale the interfaces all \xml{ProcessNode}s ``interior" to a \xml{ProcessList} to be 32f according to \ref{sec:scaling}. As discussed in \ref{sec:processList-in-out}, applications may want to rescale the ``exterior" interfaces of the \xml{ProcessList} based on the type of pixel data being processed.

For some applications, it may be easiest to simply rescale all \xml{ProcessNode}s to 32f input and output bit-depth when parsing the file. That way, the \xml{ProcessList} may be considered a purely 32f set of operations and the implementation therefore does not need to track or deal with bit-depth differences at the \xml{ProcessNode} level.


\subsection{Conversion Between Integer and Normalized Float Scaling} \label{sec:scaling}
As discussed above, the \xml{inBitDepth} or \xml{outBitDepth} of a \xml{ProcessNode} may need to be rescaled in order to accommodate the pixel data type being processed by the application. 

The scale factor associated with the bit-depths 8i, 10i, 12i, and 16i is $2^n-1$, where $n$ is the bit-depth.

The scale factor associated with the bit-depths 16f and 32f is 1.0.

To rescale \xml{Matrix}, \xml{LUT1D}, or \xml{LUT3D} \xml{Array} values when the \xml{outBitDepth} changes, the scale factor is equal to $\frac{\mathrm{newScale}}{\mathrm{oldScale}}$. For example, to convert from 12i to 10i, multiply array values by $1023/4095$.

To rescale \xml{Matrix} \xml{Array} values when the \xml{inBitDepth} changes, the scale factor is equal to $\frac{\mathrm{oldScale}}{\mathrm{newScale}}$. For example, to convert from 32f to 10i, multiply array values by $1/1023$.

To rescale \xml{Range} parameters when the \xml{inBitDepth} changes, the scale factor for \xml{minInValue} and \\ \xml{maxInValue} is $\frac{\mathrm{newScale}}{\mathrm{oldScale}}$. To rescale \xml{Range} parameters when the \xml{outBitDepth} changes, the scale factor for \xml{minOutValue} and \xml{maxOutValue} is $\frac{\mathrm{newScale}}{\mathrm{oldScale}}$.

Please note that in all cases, the conversion shall be only a scale factor. In none of the above cases, should clamping or quantization be applied.

Aside from the specific cases listed above, changes to \xml{inBitDepth} and \xml{outBitDepth} do not affect the parameter or array values of a given \xml{ProcessNode}.

If an application needs to convert between different integer pixel formats or between integer and float (or vice versa) on the way into or out of a \xml{ProcessList}, the same scale factors should be used. Note that when converting from floating-point to integer at the application level that values should be clamped, rounded, and quantized.


\section{Required vs Optional}
The required or optional indicated in parentheses throughout this specification indicate the requirement for an element or attribute to be present for a valid CLF file. In the spirit of a LUT format to be used commonly across different software and hardware, none of the elements or attributes should be considered optional for implementors to support. All elements and attributes, if present, should be recognized and supported by an implementation.

If, due to hardware or software limitations, a particular element or attribute is not able to be supported, a warning should be issued to the user of a LUT that contains one of the offending elements. The focus shall be on the user and maintaining utmost compatibility with the specification so that LUTs can be interchanged seamlessly.

\section{Efficient Processing}
\label{sec:efficient-processing}
The transform engine may merge some \xml{ProcessNode}s in order to obtain better performance. For example, adjacent \xml{Matrix} operators may be combined into a single matrix. However, in general, combining operators in a way that preserves accuracy is difficult and should be avoided. 

Hardware implementations may need to convert all \xml{ProcessNode}s into some other form that is consistent with what the hardware supports. For example, all \xml{ProcessNode}s might need to be combined into a single 3D LUT. Using a grid size of 64 or larger is recommended to preserve as much accuracy as possible. Implementors should be aware that the success of such approximations varies greatly with the nature of the input and output color spaces. For example, if the input color space is scene-linear in nature, it may be necessary to use a ``shaper LUT'' or similar non-linearity before the 3D LUT in order to convert values into a more perceptually uniform representation.

 
\section{Extensions}
It is recommended that implementors of CLF file readers protect against unrecognized elements or attributes that are not defined in this specification. Unrecognized elements that are not children of the \xml{Info} element should either raise an error or at least provide a warning message to the user to indicate that there is an operator present that is not recognized by the reader. Applications that need to add custom metadata should place it under the \xml{Info} element rather than at the top level of the \xml{ProcessList}.

One or more \xml{Description} elements in the \xml{ProcessList} can and should be used for metadata that does not fit into a provided field in the \xml{Info} element and/or is unlikely to be recognized by other applications.
% This file contains the content for a main section
\regularsectionformat
%% Modify below this line %%
\chapter{Examples}
\label{sec:examples}

This section illustrates some of the typical forms of the LUT format. It should be noted that these are not real examples.

The simplest form is an XML file containing a single node:

\begin{lstlisting}[caption=Example of a ``LUT1D'']
<?xml version="1.0" encoding="UTF-8"?>
<ProcessList xmlns="urn:NATAS:ASC:LUT:v1.2" id="ex1" name="example 1 transform">
	<Description> Turn 4 grey levels into 4 inverted codes using a 1D </Description>
	<LUT1D id="lut-23" name="4valueLut" inBitDepth="12i" outBitDepth="12i">
		<Description> 1D LUT </Description>
		<Array dim="4 1">
			3
			2
			1
			0
		</Array>
	</LUT1D>
</ProcessList>
\end{lstlisting}

The \texttt{LUT1D} ProcessNode could be replaced with a 3D LUT (\autoref{ex:3dlut}) or a matrix (\autoref{ex:matrix}).

\begin{lstlisting}[caption=Example of a \texttt{LUT3D},label=ex:3dlut]
<LUT3D id="lut-24" name="green look" interpolation="trilinear" 
	inBitDepth="12i" outBitDepth="16f">
	<Description> 3D LUT </Description>
	<Array dim="2 2 2 3">
		0.0 0.0 0.0
		0.0 0.0 1.0
		0.0 1.0 0.0
		0.0 1.0 1.0
		1.0 0.0 0.0
		1.0 0.0 1.0
		1.0 1.0 0.0
		1.0 1.0 1.0
	</Array>
</LUT3D>

\end{lstlisting}

\begin{lstlisting}[caption=Example of a \texttt{Matrix},label=ex:matrix]
<Matrix id="lut-25" name="colorspace conversion" inBitDepth="10i" outBitDepth="10i" >
	<Description> 3x4 Matrix , 4th column is offset </Description>
	<Array dim="3 4 3">
		1.2  	0.0  	0.0   	0.002
		0.0 	1.03 	0.001 	-0.005
		0.004 	-0.007 	1.004  	0.0
	</Array>
</Matrix>

\end{lstlisting}

``Shaper LUTs'' require a bit more of an explanation.  This is once again an illustration of the technique using an \texttt{IndexMap} and not a real world example. These can also be implemented with \texttt{1DLUT}s.

\begin{lstlisting}[caption=Example of a partially enumerated ``Shaper LUT'',label=ex:shaperlut]
<LUT1D id="lut-25" name="shaper LUT" inBitDepth="10i" outBitDepth="16f">
	<Description> 1D LUT with shaper </Description>
	<IndexMap dim=4>0@0 10@100 20@250 30@360 40@440 445@445 
					700@600 800@700 900@850 950@1023</IndexMap>
	<Array dim="1024 1">
		0.00
		0.32
		0.50
		<1020 entries omitted>
		1.0
	</Array>
</LUT1D>
\end{lstlisting}

\begin{lstlisting}[caption=Example of an ASC CDL node,label=ex:asccdl]
<ASC_CDL id="cc01234" inBitDepth="16f" outBitDepth="16f" style="Fwd">
	<Description>scene 1 exterior look</Description>
	<SOPNode>
		<Slope>1.000000  1.000000  0.900000</Slope>
		<Offset>-0.030000  -0.020000  0.000000</Offset>
		<Power>1.2500000  1.000000  1.000000</Power>
	</SOPNode>
	<SatNode>
		<Saturation>1.700000</Saturation>
	</SatNode>
</ASC_CDL>

\end{lstlisting}

A full example of an XML file (\autoref{ex:fullexample}) shows three nodes in a ProcessList.

\begin{lstlisting}[caption=Full example of an XML LUT file,label=ex:fullexample]
<?xml version="1.0" encoding="UTF-8"?>
<ProcessList xmlns="urn:NATAS:ASC:LUT:v1.2" id="luts-23+24+25" name="lut chain 34">
	<Description> Turn 4 grey levels into 4 codes for a monitor using a 3by1D LUT 
			into 3D LUT into 3x1D LUT </Description>
	<OutputDescriptor> Sony BVM CRT </OutputDescriptor>
	<LUT1D id="lut-23" name="input lut" inBitDepth="12i" outBitDepth="12i">
		<Description> 3by1D LUT </Description>
		<Array dim="4 3">
			1 1 1
			1 1 1
			2 2 2 
			2 2 2
		</Array>
	</LUT1D>
	<LUT3D id="lut-24" name="green look output rendering" interpolation="trilinear"
				inBitDepth="12i" outBitDepth="16f">
		<Description> 3D LUT </Description>
		<Array dim="4 4 4 3">
			0.0 0.0 0.0
			0.0 0.0 1.0
			0.0 1.0 0.0
			0.0 1.0 1.0
			1.0 0.0 0.0
			1.0 0.0 1.0
			1.0 1.0 0.0
			1.0 1.0 1.0
			[ed:  ...abridged:   64 total entries...]
			1.0 1.0 1.0
		</Array>
	</LUT3D>
	<LUT1D id="lut-25" name="output conversion" inBitDepth="16f" outBitDepth="12i">
		<Description> 3x1D LUT </Description>
		<IndexMap dim=2>0.0@0 3.0@65504.0</IndexMap>
		<Array dim="4 3">
			0 0 0 
			1 1 1
			2 2 2 
			3 3 3
		</Array>
	</LUT1D>
</ProcessList>

\end{lstlisting}


\begin{appendices}
    \appendixchapter{XML Schema}{n}
\label{appendixA}

\lstset{frame=none}
\begin{lstlisting}
<?xml version="1.0" encoding="UTF-8"?>

<xs:schema targetNamespace="urn:AMPAS:CLF:v3.0" 
    xmlns:xs="http://www.w3.org/2001/XMLSchema"
    xmlns:clf="urn:AMPAS:CLF:v3.0" 
    elementFormDefault="qualified"
    attributeFormDefault="unqualified">
    
    <!--  Process List definition  -->
    <xs:element name="ProcessList" type="clf:ProcessListType"/>
    
    <xs:complexType name="ProcessListType">
        <xs:sequence>
            <xs:element name="Description" type="xs:string" minOccurs="0" 
                maxOccurs="unbounded"/>
            <xs:element name="InputDescriptor" type="xs:string" minOccurs="0" 
                maxOccurs="1"/>
            <xs:element name="OutputDescriptor" type="xs:string" minOccurs="0" 
                maxOccurs="1"/>
            <xs:element ref="clf:Info" minOccurs="0" maxOccurs="1"/>
            <xs:element ref="clf:ProcessNode" minOccurs="1" 
                maxOccurs="unbounded"/>
        </xs:sequence>
        <xs:attribute name="id" type="xs:anyURI" use="required"/>
        <xs:attribute name="compCLFversion" type="xs:string" use="required"/>
        <xs:attribute name="name" type="xs:string" use="optional"/>
        <xs:attribute name="inverseOf" type="xs:string" use="optional"/>
    </xs:complexType>
    
    <!--  Info element definition  -->
    <xs:element name="Info" type="clf:InfoType"/>
    
    <xs:complexType name="InfoType">
        <xs:sequence>
            <xs:element name="AppRelease" type="xs:string" minOccurs="0" 
                maxOccurs="1"/>
            <xs:element name="Copyright" type="xs:string" minOccurs="0" 
                maxOccurs="1"/>
            <xs:element name="Revision" type="xs:string" minOccurs="0" 
                maxOccurs="1"/>            
            <xs:element name="ACEStransformID" type="xs:string" minOccurs="0" 
                maxOccurs="1"/>
            <xs:element name="ACESuserName" type="xs:string" minOccurs="0" 
                maxOccurs="1"/>
            <xs:element ref="clf:CalibrationInfo" minOccurs="0" maxOccurs="1"/>
        </xs:sequence>
    </xs:complexType>
    
    <!--  CalibrationInfo element definition  -->
    <xs:element name="CalibrationInfo" type="clf:CalibrationInfoType"/>
    
    <xs:complexType name="CalibrationInfoType">
        <xs:sequence>
            <xs:element name="DisplayDeviceSerialNum" type="xs:string" 
                minOccurs="0" maxOccurs="1"/>
            <xs:element name="DisplayDeviceHostName" type="xs:string" 
                minOccurs="0" maxOccurs="1"/>
            <xs:element name="OperatorName" type="xs:string" minOccurs="0" 
                maxOccurs="1"/>
            <xs:element name="CalibrationDateTime" type="xs:string" 
                minOccurs="0" maxOccurs="1"/>
            <xs:element name="MeasurementProbe" type="xs:string" minOccurs="0" 
                maxOccurs="1"/>
            <xs:element name="CalibrationSoftwareName" type="xs:string" 
                minOccurs="0" maxOccurs="1"/>
            <xs:element name="CalibrationSoftwareVersion" type="xs:string" 
                minOccurs="0" maxOccurs="1"/>
        </xs:sequence>
    </xs:complexType>
    
    <!--  ProcessNode definition  -->
    <xs:element name="ProcessNode" type="clf:ProcessNodeType"/>
    
    <xs:complexType name="ProcessNodeType" abstract="true">
        <xs:sequence>
            <xs:element name="Description" type="xs:string" minOccurs="0" 
                maxOccurs="unbounded"/>
        </xs:sequence>
        <xs:attribute name="id" type="xs:anyURI" use="optional"/>
        <xs:attribute name="name" type="xs:string" use="optional"/>
        <xs:attribute name="inBitDepth" type="clf:bitDepthType" use="required"/>
        <xs:attribute name="outBitDepth" type="clf:bitDepthType" use="required"/>
    </xs:complexType>
    
    <!--  ProcessNode: LUT1D definition  -->
    <xs:element name="LUT1D" type="clf:LUT1DType" substitutionGroup="clf:ProcessNode"/>
    
    <xs:complexType name="LUT1DType">
        <xs:complexContent>
            <xs:extension base="clf:ProcessNodeType">
                <xs:sequence>
                    <xs:element name="Array" type="clf:ArrayType" minOccurs="1" 
                        maxOccurs="1"/>
                </xs:sequence>
                <xs:attribute name="interpolation" type="xs:string" use="optional"
                    fixed="linear"/>
                <xs:attribute name="rawHalfs" type="xs:string" use="optional"/>
                <xs:attribute name="halfDomain" type="xs:string" 
                    use="optional"/>
            </xs:extension>
        </xs:complexContent>
    </xs:complexType>
    
    <!--  ProcessNode: LUT3D definition  -->
    <xs:element name="LUT3D" type="clf:LUT3DType" substitutionGroup="clf:ProcessNode"/>
    
    <xs:complexType name="LUT3DType">
        <xs:complexContent>
            <xs:extension base="clf:ProcessNodeType">
                <xs:sequence>
                    <xs:element name="Array" type="clf:ArrayType" minOccurs="1" 
                        maxOccurs="1"/>
                </xs:sequence>
                <xs:attribute name="interpolation" type="xs:string" 
                    use="optional"/>
            </xs:extension>
        </xs:complexContent>
    </xs:complexType>
    
    <!--  ProcessNode: Exponent definition  -->
    <xs:element name="Exponent" type="clf:ExponentType" substitutionGroup="clf:ProcessNode"/>
    
    <xs:complexType name="ExponentType">
        <xs:complexContent>
            <xs:extension base="clf:ProcessNodeType">
                <xs:sequence>
                    <xs:element name="ExponentParamsType" type="clf:ExponentParamsType" 
                        minOccurs="0" maxOccurs="3"/>
                </xs:sequence>
            </xs:extension>
        </xs:complexContent>
    </xs:complexType>
    
    <xs:complexType name="ExponentParamsType">
        <xs:attribute name="exponent" type="xs:float" use="optional" default="1.0"/>
        <xs:attribute name="offset" type="xs:float" use="optional" default="0.0"/>
        <xs:attribute name="channel" type="clf:channelType" use="optional"/>
    </xs:complexType>

    <!--  ProcessNode: Log definition  -->
    <xs:element name="Log" type="clf:LogType" substitutionGroup="clf:ProcessNode"/>
    
    <xs:complexType name="LogType">
        <xs:complexContent>
            <xs:extension base="clf:ProcessNodeType">
                <xs:sequence>
                    <xs:element name="LogParamsType" type="clf:LogParamsType" minOccurs="0" 
                        maxOccurs="1"/>
                </xs:sequence>                
                <xs:attribute name="style" type="xs:string" use="required"/>
            </xs:extension>
        </xs:complexContent>        
    </xs:complexType>

    <xs:complexType name="LogParamsType">
        <xs:attribute name="base" type="xs:float" use="optional" default="10"/>
        <xs:attribute name="logSideSlope" type="xs:float" use="optional" default="1.0"/>
        <xs:attribute name="logSideOffset" type="xs:float" use="optional" default="0.0"/>
        <xs:attribute name="linSideSlope" type="xs:float" use="optional" default="1.0"/>
        <xs:attribute name="linSideOffset" type="xs:float" use="optional" default="0.0"/>
        <xs:attribute name="linSideBreak" type="xs:float" use="optional" default="0.0"/>
        <xs:attribute name="linearSlope" type="xs:float" use="optional"/>
        <xs:attribute name="linearOffset" type="xs:float" use="optional"/>
        <xs:attribute name="channel" type="clf:channelType" use="optional"/>
    </xs:complexType>
    
    <!--  ProcessNode: ASC-CDL definition  -->
    <xs:element name="ASC_CDL" type="clf:ASC_CDLType" substitutionGroup="clf:ProcessNode"/>
    
    <xs:complexType name="ASC_CDLType">
        <xs:complexContent>
            <xs:extension base="clf:ProcessNodeType">
                <xs:sequence>
                    <xs:element name="SOPNodeType" type="clf:SOPNodeType" minOccurs="0" 
                        maxOccurs="1"/>
                    <xs:element name="SatNodeType" type="clf:SatNodeType" minOccurs="0" 
                        maxOccurs="1"/>
                </xs:sequence>                
                <xs:attribute name="style" type="xs:string" use="required"/>
            </xs:extension>
        </xs:complexContent>        
    </xs:complexType>
    
    <xs:complexType name="SOPNodeType">
        <xs:sequence>
            <xs:element name="Slope" type="clf:floatListType" minOccurs="0" maxOccurs="1" 
                default="1.0 1.0 1.0"/>
            <xs:element name="Offset" type="clf:floatListType" minOccurs="0" maxOccurs="1" 
                default="0.0 0.0 0.0"/>
            <xs:element name="Power" type="clf:floatListType" minOccurs="0" maxOccurs="1" 
                default="1.0 1.0 1.0"/>
        </xs:sequence>
    </xs:complexType>

    <xs:complexType name="SatNodeType">
        <xs:sequence>
            <xs:element name="Saturation" type="xs:float" minOccurs="0" maxOccurs="1" 
                default="1.0"/>
        </xs:sequence>
    </xs:complexType>
    
    <!--  ProcessNode: Range definition  -->
    <xs:element name="Range" type="clf:RangeType" substitutionGroup="clf:ProcessNode"/>
    
    <xs:complexType name="RangeType">
        <xs:complexContent>
            <xs:extension base="clf:ProcessNodeType">
                <xs:sequence>
                    <xs:element name="minValueIn" type="xs:float" minOccurs="0" 
                        maxOccurs="1"/>
                    <xs:element name="maxValueIn" type="xs:float" minOccurs="0" 
                        maxOccurs="1"/>
                    <xs:element name="minValueOut" type="xs:float" minOccurs="0"
                        maxOccurs="1"/>
                    <xs:element name="maxValueOut" type="xs:float" minOccurs="0"
                        maxOccurs="1"/>
                </xs:sequence>
                <xs:attribute name="style" type="xs:string" use="optional" 
                    default="Clamp"/>
            </xs:extension>
        </xs:complexContent>
    </xs:complexType>
    
    <!--  ProcessNode: Matrix definition  -->
    <xs:element name="Matrix" type="clf:MatrixType" 
        substitutionGroup="clf:ProcessNode"/>
    
    <xs:complexType name="MatrixType">
        <xs:complexContent>
            <xs:extension base="clf:ProcessNodeType">
                <xs:sequence>
                    <xs:element name="Array" type="clf:ArrayType" minOccurs="1" 
                        maxOccurs="1"/>
                </xs:sequence>
            </xs:extension>
        </xs:complexContent>
    </xs:complexType>
    
    <xs:complexType name="ArrayType">
        <xs:simpleContent>
            <xs:extension base="clf:floatListType">
                <xs:attribute name="dim" type="clf:dimType" use="optional"/>
            </xs:extension>
        </xs:simpleContent>
    </xs:complexType>
        
    <xs:simpleType name="floatListType">
        <xs:list itemType="xs:float"/>
    </xs:simpleType>
    
    <xs:simpleType name="dimType">
        <xs:restriction base="clf:positiveIntegerListType">
            <xs:minLength value="1"/>
        </xs:restriction>
    </xs:simpleType>
    
    <xs:simpleType name="positiveIntegerListType">
        <xs:list itemType="xs:positiveInteger"/>
    </xs:simpleType>
    
    <xs:simpleType name="bitDepthType">
        <xs:restriction base="xs:string">
            <xs:pattern value="[0-9]+[fi]"/>
        </xs:restriction>
    </xs:simpleType>
    
    <xs:simpleType name="channelType">
        <xs:restriction base="xs:string">
            <xs:pattern value="[RGB]"/>
        </xs:restriction>
    </xs:simpleType>
    
</xs:schema>
\end{lstlisting}


    \appendixchapter{Interpolation}{n}
\label{appendix:interpolation}

When an input value falls between sampled positions in a LUT, the output value must be calculated as a proportion of the distance along some function that connects the nearest surrounding values in the LUT. There are many different types of interpolation possible, but only three types of interpolation are currently specified for use with the Common LUT Format (CLF). 

The first type -- linear interpolation -- is specified for use with a \xml{LUT1D} node. The other two -- trilinear and tetrahedral interpolation -- are specified for use with a \xml{LUT3D} node.

\section{Linear Interpolation}
With a table of the sampled input values in $inValue[i]$ where $i$ ranges from $0$ to $(n-1)$, and a table of the corresponding output values in $outValue[j]$ where $j$ is equal to $i$,

\begin{center}
\begin{tabularx}{3in}{ccXcc}
    index $i$ & inValue && index $j$ & outValue \\ \hline
    0 & 0 && 0 & 0 \\
    $\vdots$ & $\vdots$ && $\vdots$ & $\vdots$ \\
    $n-1$ & 1 && $n-1$ & 1000 \\
\end{tabularx}
\end{center}

the $output$ resulting from $input$ can be calculated after finding the nearest $inValue[i] < input$. 

When $inValue[i] = input$, the result is evaluated directly.

\begin{center}
$output = \dfrac{input-inValue[i]}{inValue[i+1]-inValue[i]} \times (outValue[j+1]-outValue[j])+outValue[j]$ 
\end{center}

\section{Trilinear Interpolation}
Trilinear interpolation implements linear interpolation in three-dimensions by successively interpolating each direction. 

\begin{figure}[htbp]
\begin{center}
    \includegraphics[width=2in]{images/interp-pointInMesh.png}
    \includegraphics[width=2in]{images/interp-pointInCubelet.png}
\caption{Illustration of a sampled point located within a basic 3D LUT mesh grid (left) and the same point but with only the vertices surrounding the sampled point (right).}
\label{fig:lut-sampling}
\end{center}
\end{figure}

\begin{figure}[htbp]
\begin{center}
    \includegraphics[width=3.5in]{images/interp-labeledCubelet.png}
\caption{Labeling the mesh points surrounding the sampled point $(r,g,b)$.}
\label{fig:mesh-points}
\end{center}
\end{figure}

\note{The convention used for notation is uppercase variables for mesh points and lowercase variables for points on the grid.}

Consider a sampled point as depicted in Figure \ref{fig:mesh-points}. Let $V(r,g,b)$ represent the value at the point with coordinate $(r,g,b)$. The distance between each node per color coordinate shows the proportion of each mesh point's color coordinate values that contribute to the sampled point.

\begin{equation}
    \Delta_r = \frac{r - R_0}{R_1 - R_0} \hspace{0.25in}
    \Delta_g = \frac{g - G_0}{G_1 - G_0} \hspace{0.25in}
    \Delta_b = \frac{b - B_0}{B_1 - B_0}
\end{equation}

The general expression for trilinear interpolation can be expressed as:

\begin{equation}
    V(r,g,b) = c_0 + c_1\Delta_b + c_2\Delta_r + c_3\Delta_g + c_4\Delta_b\Delta_r + c_5\Delta_r\Delta_g + c_6\Delta_g\Delta_b + c_7\Delta_r\Delta_g\Delta_b
\end{equation}
\tabto{0.5in}where: \\
\begin{equation*}
\begin{aligned}
    c_0 &= V(R_0, G_0, B_0) \\
    c_1 &= V(R_0, G_0, B_1) - V(R_0, G_0, B_0) \\
    c_2 &= V(1_0, G_0, B_0) - V(R_0, G_0, B_0) \\
    c_3 &= V(R_0, G_1, B_0) - V(R_0, G_0, B_0) \\
    c_4 &= V(R_1, G_1, B_1) - V(R_1, G_0, B_0) - V(R_0, G_0, B_1) + V(R_0, G_0, B_0) \\
    c_5 &= V(R_1, G_1, B_0) - V(R_0, G_1, B_0) - V(R_1, G_0, B_0) + V(R_0, G_0, B_0) \\
    c_6 &= V(R_0, G_1, B_1) - V(R_1, G_1, B_0) - V(R_0, G_0, B_1) + V(R_0, G_0, B_0) \\
    c_7 &= V(R_1, G_1, B_1) - V(R_1, G_1, B_0) - V(R_0, G_1, B_1) - V(R_1, G_0, B_1) \\ 
        &+ V(R_0, G_0, B_1) + V(R_0, G_1, B_0) + V(R_1, G_0, B_0) - V(R_0, G_0, B_0) 
\end{aligned}    
\end{equation*}

Expressed in matrix form:
\begin{gather}
    \matr{C} = [c_0 \enspace c_1 \enspace c_2 \enspace c_3 \enspace c_4 \enspace c_5 \enspace c_6 \enspace c_7]^T \\
    \matr{\Delta} = [1 \quad \Delta_b \quad \Delta_r \quad \Delta_g \quad \Delta_b\Delta_r \quad \Delta_r\Delta_g \quad \Delta_g\Delta_b \quad \Delta_r\Delta_g\Delta_b]^T \\
    V(r,g,b) = \matr{C}^T \matr{\Delta}
\end{gather}

\begin{equation} \label{eq:linInterpMatrixForm}
    \begin{bmatrix}
    c_0 \\ c_1 \\ c_2 \\ c_3 \\ c_4 \\ c_5 \\ c_6 \\ c_7
    \end{bmatrix} =
    \begin{bmatrix}
    1 & 0 & 0 & 0 & 0 & 0 & 0 & 0 \\
    -1 & 0 & 0 & 0 & 1 & 0 & 0 & 0 \\
    -1 & 0 & 1 & 0 & 0 & 0 & 0 & 0 \\
    -1 & 1 & 0 & 0 & 0 & 0 & 0 & 0 \\
    1 & 0 & -1 & 0 & -1 & 0 & 1 & 0 \\
    1 & -1 & -1 & 1 & 0 & 0 & 0 & 0 \\
    1 & -1 & 0 & 0 & -1 & 1 & 0 & 0 \\
    -1 & 1 & 1 & -1 & 1 & -1 & -1 & 1
    \end{bmatrix}
    \begin{bmatrix}
    V(R_0, G_0, B_0) \\
    V(R_0, G_1, B_0) \\
    V(R_1, G_0, B_0) \\
    V(R_1, G_1, B_0) \\
    V(R_0, G_0, B_1) \\
    V(R_0, G_1, B_1) \\
    V(R_1, G_0, B_1) \\
    V(R_1, G_1, B_1)
    \end{bmatrix}    
\end{equation}

The expression in Equation \ref{eq:linInterpMatrixForm} can be written as: $\matr{C} = \matr{A}\matr{V}$.

Trilinear interpolation shall be done according to $V(r,g,b) = \matr{C}^T \matr{\Delta} = \matr{V}^T \matr{A}^T \matr{\Delta}$.

\note{ The term $\matr{V}^T \matr{A}^T$ does not depend on the variable $(r,g,b)$ and thus can be computed in advance for optimization. Each sub-cube can have the values of the vector $\matr{C}$ already stored in memory. Therefore the algorithm can be summarized as: 
\begin{enumerate}
    \item Find the sub-cube containing the point \((r,g,b)\)
    \item Select the vector \(\matr{C}\) corresponding to that sub-cube
    \item Compute \(\Delta_r\), \(\Delta_g\), \(\Delta_b\)
    \item Return \(V(r,g,b) = \matr{C}^T \matr{\Delta}\)
\end{enumerate}
}

\section{Tetrahedral Interpolation}
Tetrahedral interpolation subdivides the cubelet defined by the vertices surrounding a sampled point into six tetrahedra by segmenting along the main (and usually neutral) diagonal (Figure \ref{fig:tetrahedra}). 

\begin{figure}[htbp]
\begin{center}
    \includegraphics[width=6in]{images/interp-tetrahedrons.png}
\caption{Illustration of the six subdivided tetrahedra.}
\label{fig:tetrahedra}
\end{center}
\end{figure}

To find the tetrahedron containing the point \((r,g,b)\):
\begin{itemize}
    \item if \(\Delta_b > \Delta_r > \Delta_g\), then use the first tetrahedron, \(t1\)
    \item if \(\Delta_b > \Delta_g > \Delta_r\), then use the first tetrahedron, \(t2\)
    \item if \(\Delta_g > \Delta_b > \Delta_r\), then use the first tetrahedron, \(t3\)
    \item if \(\Delta_r > \Delta_b > \Delta_g\), then use the first tetrahedron, \(t4\)
    \item if \(\Delta_r > \Delta_g > \Delta_b\), then use the first tetrahedron, \(t5\)
    \item else, use the sixth tetrahedron, \(t6\)
\end{itemize}

The matrix notation is:

\begin{equation}
    \matr{V} = \begin{bmatrix}
    V(R_0, G_0, B_0) \\
    V(R_0, G_1, B_0) \\
    V(R_1, G_0, B_0) \\
    V(R_1, G_1, B_0) \\
    V(R_0, G_0, B_1) \\
    V(R_0, G_1, B_1) \\
    V(R_1, G_0, B_1) \\
    V(R_1, G_1, B_1)
    \end{bmatrix}\\ \\
\end{equation}
\begin{equation}
    \matr{\Delta_t} = [1 \enspace \Delta_b \enspace \Delta_r \enspace \Delta_g]^T
\end{equation}

\begin{equation}
\begin{aligned} \label{eq:tetrahedral-matrices}
    \matr{T}_1 = \begin{bmatrix}
    1 & 0 & 0 & 0 & 0 & 0 & 0 & 0 \\
    -1 & 0 & 0 & 0 & 1 & 0 & 0 & 0 \\
    0 & 0 & 0 & 0 & -1 & 0 & 1 & 0 \\
    0 & 0 & 0 & 0 & 0 & 0 & -1 & 1
    \end{bmatrix} \hspace{0.5in}
    \matr{T}_2 = \begin{bmatrix}
    1 & 0 & 0 & 0 & 0 & 0 & 0 & 0 \\
    -1 & 0 & 0 & 0 & 1 & 0 & 0 & 0 \\
    0 & 0 & 0 & 0 & 0 & -1 & 0 & 1 \\
    0 & 0 & 0 & 0 & -1 & 1 & 0 & 0
    \end{bmatrix} \\
    \matr{T}_3 = \begin{bmatrix}
    1 & 0 & 0 & 0 & 0 & 0 & 0 & 0 \\
    0 & -1 & 0 & 0 & 0 & 1 & 0 & 0 \\
    0 & 0 & 0 & 0 & 0 & -1 & 0 & 1 \\
    -1 & 1 & 0 & 0 & 0 & 0 & 0 & 0
    \end{bmatrix} \hspace{0.5in}
    \matr{T}_4 = \begin{bmatrix}
    1 & 0 & 0 & 0 & 0 & 0 & 0 & 0 \\
    0 & 0 & -1 & 0 & 0 & 0 & 1 & 0 \\
    -1 & 0 & 1 & 0 & 0 & 0 & 0 & 0 \\
    0 & 0 & 0 & 0 & 0 & 0 & -1 & 1
    \end{bmatrix} \\
    \matr{T}_5 = \begin{bmatrix}
    1 & 0 & 0 & 0 & 0 & 0 & 0 & 0 \\
    0 & 0 & 0 & -1 & 0 & 0 & 0 & 1 \\
    -1 & 0 & 1 & 0 & 0 & 0 & 0 & 0 \\
    0 & 0 & -1 & 1 & 0 & 0 & 0 & 0
    \end{bmatrix} \hspace{0.5in}
    \matr{T}_6 = \begin{bmatrix}
    1 & 0 & 0 & 0 & 0 & 0 & 0 & 0 \\
    0 & 0 & 0 & -1 & 0 & 0 & 0 & 1 \\
    0 & -1 & 0 & 1 & 0 & 0 & 0 & 0 \\
    -1 & 1 & 0 & 0 & 0 & 0 & 0 & 0
    \end{bmatrix}
\end{aligned}
\end{equation}

Trilinear interpolation shall be done according to:
\begin{gather}
    V(r,g,b)_{t1} = \matr{\Delta}^T_t \matr{T}_1 \matr{V}\\
    V(r,g,b)_{t2} = \matr{\Delta}^T_t \matr{T}_2 \matr{V}\\
    V(r,g,b)_{t3} = \matr{\Delta}^T_t \matr{T}_3 \matr{V}\\
    V(r,g,b)_{t4} = \matr{\Delta}^T_t \matr{T}_4 \matr{V}\\
    V(r,g,b)_{t5} = \matr{\Delta}^T_t \matr{T}_5 \matr{V}\\
    V(r,g,b)_{t6} = \matr{\Delta}^T_t \matr{T}_6 \matr{V}
\end{gather} 

\note{ The vectors \(\matr{T}_i \matr{V}\) for \(i = 1,2,3,4,5,6\) does not depend on the variable \((r,g,b)\) and thus can be computed in advance for optimization.}
    
    \appendixchapter{Cineon-style Log Parameters}{i}
\label{appendix:cineon-log}

When using a \xml{Log} node, it might be desirable to conform an existing logarithmic function that uses Cineon style parameters to the parameters used by CLF. A translation from Cineon-style parameters to those used by CLF's \xml{LogParams} element is quite straightforward using the following steps.

Traditionally, \var{refWhite} and \var{refBlack} are provided as 10-bit quantities, and if they indeed are, first normalize them to floating point by dividing by 1023.
\begin{align}
    \var{refWhite} = \frac{\var{refWhite}_{10i}}{1023.0} \\
    \var{refBlack} = \frac{\var{refBlack}_{10i}}{1023.0}        
\end{align}
\tabto{0.5in} Where:
\tabto{0.75in} subscript \emph{10i} indicates a 10-bit quantity

The density range is assumed to be:
\begin{equation}
    \var{range} = 0.002 \times 1023.0
\end{equation}

Then solve the following quantities:
\begin{align}
    \var{multFactor} =& \frac{\var{range}}{\var{gamma}} \\
    \var{gain} =& \frac{\var{highlight} - \var{shadow}}{1.0 - 10^{( MIN( \var{multFactor} \times (\var{refBlack}-\var{refWhite}), -0.0001)}} \\
    \var{offset} =& \var{gain} - (\var{highlight} - \var{shadow}) \\
\end{align}
\tabto{0.5in} Where:
\tabto{0.75in} $MIN(x,y)$ returns $x$ if $x<y$, otherwise returns $y$

The parameters for the \xml{LogParams} element are then:
\begin{align}
    \xml{base} =& 10.0 \\
    \xml{logSlope} =& \frac{1}{\var{multFactor}} \\
    \xml{logOffset} =& \var{refWhite} \\
    \xml{linSlope} =& \frac{1}{\var{gain}} \\
    \xml{linOffset} =& \frac{\var{offset}-\var{shadow}}{\var{gain}}
\end{align}    
    \appendixchapter{Changes between v2.0 and v3.0}{i}
\label{appendix-changes}

\vspace{20pt}
\begin{itemize}
    \item Add \xml{Log} ProcessNode
    \item Add \xml{Exponent} ProcessNode
    \item Revise formulas for defining use of \xml{Range} ProcessNode to clamp at the low or high end.
    \item \xml{IndexMap}s removed. Use a \xml{halfDomain} LUT to achieve reshaping of input to a LUT.
    \item Move \xml{ACEStransform} elements to \xml{Info} element of \xml{ProcessList} in main spec
    \item Changed syntax for \xml{dim} attribute of \xml{Array} when contained in a \xml{Matrix}. Two integers are now used to define the dimensions of the matrix instead of the previous three values which defined the dimensions of the matrix and the number of color components.
    \item Update schema to correct errors and add new elements
\end{itemize}
\end{appendices}

\end{document}