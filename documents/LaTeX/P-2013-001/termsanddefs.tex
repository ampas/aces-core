% This section contains the content for the Terms and Definitions
\numberedformat
\chapter{Terms and Definitions}
The following terms and definitions are used in this document.
%% Modify below this line %%

\term{Academy Color Encoding Specification (ACES)}
RGB color encoding for exchange of image data that have not been color rendered, between and throughout production and postproduction, within the Academy Color Encoding System. ACES is specified in SMPTE ST 2065-1.

\term{ACES RGB relative exposure values}
Relative responses to light of the ACES Reference Image Capture Device, determined by the integrated spectral responsivities of its color channels and the spectral radiances of scene stimuli.

\term{ACES unity neutral}
A triplet of ACES RGB relative exposure values all of which have unity magnitude.

\term{chromatic adaptation}
A process by which the visual mechanism adjusts in response to the radiant energy to which the eyes are exposed.

\term{chromaticity}
A property of a color stimulus defined by the ratios of each tristimulus value of the color stimulus to their sum.

\term{color rendering}
The mapping of image data representing the color-space coordinates of the elements of a scene to output-referred image data representing the color-space coordinates of the elements of a reproduction. Color rendering generally consists of one or more of the following: compensating for differences in the input and output viewing conditions, tone scale and gamut mapping to map the scene colors onto the dynamic range and color gamut of the reproduction, and applying preference adjustments.

\term{color stimulus}
Radiant energy such as that produced by an illumination source, by the reflection of light from a reflective object, or by the transmission of light through a transmissive object, or a combination of these.

\term{focal-plane-referred}
A representation of a captured scene that includes any flare light introduced by the camera’s optical system.

\term{Input Device Transform (IDT)}
A signal-processing transform that maps an image capture system’s representation of an image to ACES RGB relative exposure values.

\term{memory color}
A color sensation derived from memory rather than the immediate perception of a color stimulus.

\term{radiometric linearity}
An attribute of a representation of measured energy in which a change in the amount of measured energy is accompanied by an equal change in the representation of that energy, e.g. a doubling of measured energy is matched by a doubling of the quantity representing that energy.

\term{Reference Input Capture Device (RICD)}
A hypothetical camera, which records an image of a scene directly as ACES RGB relative exposure values.

\term{re-illumination}
An alteration of colors of a captured scene simulating the reflectance of objects in a scene illuminated by an illumination source other than the one under which the scene was captured.

\term{scene adopted white}
A spectral radiance distribution as seen by an image capture or measurement device that is converted to color signals that are considered to be perfectly achromatic and to have an observer adaptive luminance factor of unity; i.e. color signals that are considered to correspond to a perfect white diffuser.

\term{spectral responsivity}
The response of a detection system as a function of wavelength.

\term{spectral sensitivity}
The response of a detector to monochromatic stimuli of equal radiant power.

\term{white balance}
The process of adjusting the RGB signals of an electronic camera system such that equal signals are produced for an object in the scene that is desired to be achromatic