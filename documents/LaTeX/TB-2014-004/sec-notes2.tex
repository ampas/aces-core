% This file contains the content for a main section
\numberedformat
%% Modify below this line %%
\chapter{Notes on Appendix A}

Those referring to S-2008-001 -- Annex B (contained in this document's Appendix \ref{appendixA}) may encounter some difficulty in trying to reproduce the tabular list of ACES RGB relative exposure values that the RICD would produce for an 18\% neutral reflecting diffuser, a perfect reflecting diffuser and a ColorChecker\textregistered chart, all illuminated by a CIE Standard Illuminant D60 source.

As written, the equations in S-2008-001 -- Annex B do not factor in the RICD camera flare that amounts to 0.5\% of the capture values of a perfect reflecting diffuser. Without the augmentation of flare, followed by the scale factor, values computed with the provided equations will be slightly off from those provided in the table.
 
Therefore, the full computation steps necessary to calculate ACES RICD relative exposure values that match those provided in the S-2008-001 Annex B table are clarified as follows:

\begin{quote}
Calculation of the values in the table listing stimulus and ACES RGB used the following elements:
	
\begin{itemize}
	\item the published CIE Standard Illuminant D60 spectral power distribution
	\item the area-normalized RICD spectral sensitivities from Annex C
	\item red, green and blue scale factors $k_r$, $k_g$ and $k_b$ which white-balance those area-normalized RICD spectral sensitivities for CIE Standard Illuminant D60
	\item hypothetical 18\% and 100\% neutral reflectors
	\item the measured reflectances of the patches on a particular ColorChecker chart
	\item camera flare amounting to 0.5\% of the capture values of a perfect reflecting diffuser	
\end{itemize}

The scale factors applied to the RICD spectral sensitivity curves are each the reciprocal of the scalar product of that curve and the illuminant. These scale factors $k_r$, $k_g$ and $k_b$ are calculated as follows:


\begin{center}
$k_r=\dfrac{1}{(I\cdot{S_r})},\quad k_g=\dfrac{1}{(I\cdot{S_g})},\quad k_b=\dfrac{1}{(I\cdot{S_b})},$	
\end{center}

where $I$ represents the illuminant and $S_r$, $S_g$ and $S_b$ represent the area-normalized spectral sensitivities from Annex C.

Once these scale factors have been determined, the ACES relative exposure values $E_r$, $E_g$ and $E_b$ are calculated as

\begin{floatequ}
\begin{gather}
E_r=S\left(\displaystyle\sum_\lambda IRS_{r_\lambda}+0.005\right), \quad\quad E_g=S\left(\displaystyle\sum_\lambda IRS_{g_\lambda}+0.005\right),\\
E_b=S\left(\displaystyle\sum_\lambda IRS_{b_\lambda}+0.005\right)
\end{gather} 
\end{floatequ}
\vspace{-20pt}

where (as before) $I$ represents the illuminant and $S_r$, $S_g$ and $S_b$ represent the area-normalized spectral sensitivities from Annex C, where $R$ represents the spectral reflectance of the object being captured, and where $S=\frac{0.18}{(0.18+0.005)}$, the flare factor from S-2008-001 4.1.1.

\end{quote}