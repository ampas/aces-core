% This file contains the content for the Introduction
\unnumberedformat	    % Change formatting to that of "Introduction" section
\chapter{Introduction} 	% Do not modify section title
%% Modify below this line %%

A goal of ACES 1.0 is to enable widespread adoption by encouraging consistent implementations in production and post-production tools throughout the complete film and television product ecosystem spanning capture to archiving. This is a very diverse set of tools, each used by professionals with different sets of skills. Furthermore, each manufacturer has established their own set of conventions for how to structure their user experience to best serve their market. Clearly, it is neither feasible nor appropriate for these guidelines to specify in minute detail every aspect of a user interface (e.g. ``all products must use a set of vertical drop-down menus labeled in 10-point Helvetica'').

That said, the feedback from users on the first wave of products implementing the pre-release versions of ACES has been clear in the need for guidelines. One common comment is that the implementations are so different, figuring out how to configure ACES in one product is of little help when configuring the next. For example, naming conventions are different for no apparent reason.

Another common concern is that the system is too reliant on acronyms and uses unfamiliar concepts (e.g., what is a ``reference rendering transform''?). Although some of these acronyms have become familiar within the inner circle of ACES product partners and early adopters, it must be acknowledged that the tolerance for these terms is much lower amongst the general population of industry professionals (e.g. how would one explain what an RRT is to an editor, CG animator, or anyone else without some color science background).

As the ACES project transitions from technical development to wider industry deployment and the release of Version 1.0, it is appropriate that we take a fresh look at how to portray the system to an audience that includes end-users in addition to engineers and color scientists. Although the technical terms and acronyms will continue to be used within the engineering community, these guidelines introduce a new set of terms intended to be simpler and more familiar to a wider set of users.

\note{This document provides naming conventions in English. However, it is recognized that for many products it will be necessary to translate these names into other languages to localize for various global markets.}