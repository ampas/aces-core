\documentclass[10pt]{academydoc}
\pagestyle{plain}

% Set Document Details
\doctype{tb} % spec, proc, tb (Specification, Procedure, Technical Bulletin)
\docname{Academy Color Encoding System (ACES) Clip-level Metadata File Format Definition and Usage}
\altdocname{ACES Clip-level Metadata File Format Definition and Usage}
% Sets the document name used in header - usually an abbreviated document title
\docnumber{TB-2014-009}
\committeename{Academy Color Encoding System (ACES) Project Committee}
\versionnumber{1.0.1}
\docdate{April 24, 2015}
\summary{
The ACES Clip-level Metadata File (``ACESclip'') is a `sidecar' XML file intended to assist in configuring ACES viewing pipelines and to enable portability of ACES transforms in production. This document specifies use cases for ACESclip files, application support requirements, and the data model and XML tags needed for implementation.
}

% Document Starts Here
\begin{document}

\maketitle

% This file contains the content for the Notices
\prelimsectionformat	% Change formatting to that of "Notices" section
\chapter{\uppercase{Notices}}
%% Modify below this line %%

\copyright\the\year{} Academy of Motion Picture Arts and Sciences (A.M.P.A.S.). All rights reserved. This document is provided to individuals and organizations for their own internal use, and may be copied or reproduced in its entirety for such use. This document may not be published, distributed, publicly displayed, or transmitted, in whole or in part, without the express written permission of the Academy.

The accuracy, completeness, adequacy, availability or currency of this document is not warranted or guaranteed. Use of information in this document is at your own risk. The Academy expressly disclaims all warranties, including the warranties of merchantability, fitness for a particular purpose and non-infringement.

Copies of this document may be obtained by contacting the Academy at councilinfo@oscars.org.

``Oscars,'' ``Academy Awards,'' and the Oscar statuette are registered trademarks, and the Oscar statuette a copyrighted property, of the Academy of Motion Picture Arts and Sciences.

% This paragraph is optional.  Comment out if you wish to remove it.
This document is distributed to interested parties for review and comment. A.M.P.A.S. reserves the right to change this document without notice, and readers are advised to check with the Council for the latest version of this document.

% This paragraph is optional.  Comment out if you wish to remove it.
The technology described in this document may be the subject of intellectual property rights (including patent, copyright, trademark or similar such rights) of A.M.P.A.S. or others. A.M.P.A.S. declares that it will not enforce any applicable intellectual property rights owned or controlled by it (other than A.M.P.A.S. trademarks) against any person or entity using the intellectual property to comply with this document.

% This paragraph is optional.  Comment out if you wish to remove it.
Attention is drawn to the possibility that some elements of the technology described in this document, or certain applications of the technology may be the subject of intellectual property rights other than those identified above. A.M.P.A.S. shall not be held responsible for identifying any or all such rights. Recipients of this document are invited to submit notification to A.M.P.A.S. of any such intellectual property of which they are aware.

\vspace{10pt}
These notices must be retained in any copies of any part of this document. \newpage
% This file contains the content for the Revision History and 
\prelimsectionformat	% Change formatting to that of "Notices" section
\chapter{Revision History}
%% Modify below this line %%

\begin{tabularx}{\linewidth}{|l|l|X|}
    \hline
    Version & Date       & Description \\ \hline
    1.0     & 12/19/2014 & Initial Version
    \\ \hline
    1.0.1   & 04/24/2015 & Formatting and typo fixes \\ \hline
    &   &   \\ \hline
    &   &   \\ \hline
    &   &   \\ \hline
    &   &   \\ \hline
\end{tabularx}

\vspace{0.25in} % <-- DO NOT REMOVE
\chapter{Related A.M.P.A.S. Documents} % <-- DO NOT REMOVE
\begin{tabularx}{\linewidth}{|l|X|}
    \hline
    Document Name & Description \\ \hline
    & \\ \hline
    & \\ \hline
    & \\ \hline
    & \\ \hline
    & \\ \hline
\end{tabularx} \newpage

\tableofcontents \newpage

% This file contains the content for the Scope
\cleardoublepage
\numberedformat	
\chapter{Scope} 	% Do not modify section title
%% Modify below this line %%

This document describes a 32-bit floating point encoding of ACES for use within color grading systems. 

Equivalent functions may be used for implementation purposes as long as correspondence of grading parameters to this form of log implementation is properly maintained. This document is intended as a guideline to aid developers who are integrating an ACES workflow into a color correction system.
% This section contains the content for the References
\numberedformat
\chapter{References}
The following standards, specifications, articles, presentations, and texts are referenced in this text:
%% Modify below this line %%

IETF RFC 3066:  IETF (Internet Engineering Task Force). RFC 3066: Tags for the Identification of Languages, ed. H. Alvestrand. 2001 IEEE DRAFT Standard P123

Academy S-2014-002, Academy Color Encoding System -- Versioning System

Academy TB-2014-002, Academy Color Encoding System Version 1.0 User Experience Guidelines

ASC Color Decision List (ASC CDL) Transfer Functions and Interchange Syntax. ASC-CDL\_Release1.2. 
Joshua Pines and David Reisner. 2009-05-04.
% This section contains the content for the Terms and Definitions
\numberedformat
\chapter{Terms and Definitions}
The following terms and definitions are used in this document.
%% Modify below this line %%

\term{ACESclip}
Collection of image files color-managed using the Academy Color Encoding System (ACES).

\term{ACESclip file, ACES Clip-level Metadata File}
Metadata “sidecar” XML-based file that contains information describing an ACESclip.

\term{ACES Encodings}
Color encoding specifications specified as part of the Academy Color Encoding System, e.g., ACES2065-1, ACEScc, etc.

\term{ACES File Formats}
Digital data containers specified as part of the Academy Color Encoding System, e.g., ACESclip files, ACES Image Container (SMPTE ST2065-4), etc.

\term{ACES Product Partners}
Companies that integrate ACES concepts and components into their products and/or services.

\term{ACES System}
Complete set of components that comprise the Academy Color Encoding System.

\term{ACES System Release}
Published ACES System.

\term{ACES Transforms}
Color transformations specified as part of the Academy Color Encoding System, e.g., Reference Rendering Transform (RRT), Output Device Transforms (ODT), etc.

\term{CTL files}
Files containing Color Transformation Language code. CTL files are the primary documentation for ACES transforms.

\term{Implementation Transforms}
ACES System transforms implemented by ACES Product Partners, likely as a Color Look-up Table or as GPU or CPU code.

\term{Transform Identifiers}
Tags that identify specific ACES Transforms.

% This file contains the content for a main section
\regularsectionformat	% Change formatting to that of a main, numbered section
%% Modify below this line %%
\chapter{Use Cases}
ACES Metadata Files (AMFs) are intended to contain the minimum required metadata for transferring information about ACES viewing pipelines during production, post-production, and archival.

Typical use cases for AMF files are the application of ``show LUT'' LMTs in cameras and on-set systems, the capture of shot-to-shot looks generated on-set using ASC-CDL, and communication of both to dailies, editorial, VFX, and post-production mastering facilities.

AMF supports the transfer of looks by embedding ASC-CDL values within the AMF file or by referencing sidecar look files containing LMTs, such as CLF (Common LUT Format) files.  

\section{Look Development}
The development of a creative look before the commencement of production is common. Production uses this look to produce a pre-adjusted reference for on-set monitoring. The creative look may be a package of files containing a viewing transform (also known as a “Show LUT”), CDL grades, or more. There are no consistent standards specifying how to produce them, and exchanging them is complex due to a lack of metadata.

AMF contains the ability to completely specify the application of a creative look. This automates the exchange of these files and the recreation of the look when applying the AMF. In an ACES workflow, one specifies the creative look as one or more Look Modification Transforms (LMT). AMF can include references to any number of these transforms, and maintains their order of operations.

The input and output of an LMT is always a triplet of ACES RGB relative exposure values, as defined in SMPTE ST 2065-1. This will likely need a robust transform, such as CLF, that can handle linear input data and output data.

AMF offers an unambiguous description of the full ACES viewing pipeline for on-set look management software to load and display images as intended.

\section{On Set}
Before production begins, an AMF may be created and shared with production as a ``look template'' for use during on-set monitoring or look management.

Cameras with AMF support can load or generate AMFs to configure or communicate the viewing pipeline of images viewed out of the camera's live video signal.

On-set color grading software can load or generate AMFs, allowing the communication of the color adjustments created on set.

\section{Dailies}
Dailies can apply AMFs from production to the camera files to reproduce the same images seen on set. There is no single method of exchange between production and dailies. AMFs should be agnostic to the given exchange method.

It is possible, or even likely, that one will update AMFs in the dailies stage. For example, a dailies colorist may choose to balance shots at this stage and update the look. Another example could be that dailies uses a different ODT than the one used in on-set monitoring.

This specification does not define how one should transport AMFs between stages. Existing exchange formats may reference them, or image files themselves may embed them. One may also transport AMFs independently of any other files.

\section{VFX}
The exchange of shots for VFX work requires perfect translation of each shot’s viewing pipeline, or ‘color recipe’. If the images cannot be accurately reproduced from VFX plates, effects will be created with an incorrect reference.

AMF provides a complete and unambiguous translation of ACES viewing pipelines. If they travel with VFX plates, they can describe how to view each plate along with any associated looks.

VFX software should have the ability to read AMF to configure its internal viewing pipeline. Or, AMF will inform the configuration of third party color management software, such as OpenColorIO.

\section{Finishing}
In finishing, the on-set or dailies viewing reference can be automatically recreated upon reading an AMF. This stage typically uses a higher quality display, which may warrant the use of a different ODT than one specified in an ingested AMF.

AMF can seamlessly provide the colorist a starting point that is consistent with the creative intent of the filmmakers on-set. This removes any necessity to recreate a starting look from scratch.

\section{Archival}
AMF enables the ability to establish a complete ACES archive, and effectively serves as a snapshot of creative intent for preservation and remastering purposes. All components required to recreate the look of an ACES archive are meaningfully described and preserved within the AMF.

One possible method for this could be the utilization of SMPTE standards such as ST.2067-50 (IMF App \#5) -- commonly referred to as ``ACES IMF'' -- and SMPTE RDD 47 (ISXD) -- a virtual track file containing XML data -- in order to form a complete and flexible ACES archival package.

Another method could be to use SMPTE ST.2067-9 (Sidecar Composition Map) which would allow linking of a single AMF to a CPL (Composition Playlist) in the case where there is a single AMF for an entire playlist.

% This file contains the content for a main section
\regularsectionformat	% Change formatting to that of a main, numbered section
%% Modify below this line %%
\chapter{Application and Use of ACESclip Files}

\section{ACESclip Filename and Correspondence with Images}
Transforms are identified with the CTL reference transform filename as defined in the ACES System Versioning Specification. Linking of the metadata about a transform to an actual instance of a transform is supported using ACES TransformIDs and XML id attributes.

ACESclips are named using the format ``production\_naming\_convention.ACESclip.xml''.

The production naming convention may be used to associate an ACESclip file with an image sequence, but this document does not specify an exact file naming convention.

\section{Saving State of IDT Conversion and Initial Grade}
Applications record the IDT used for converting camera-native data to ACES encodings, and include any pre-grade ASC CDL values that were used for the image sequences referenced by an ACESclip file.

\section{Conversion of Camera Files Using IDTs and Pre-grades}
For images not yet in the ACES file format, applications use the metadata for the IDT and ASC CDL pre-grade to view images using the ACES viewing pipeline. For images already in the ACES file format, the IDT conversion may be ignored, and only the pre-grade is applied prior to the ACES viewing pipeline.  

\section{Default Configuration of ACES Viewing Pipeline}
ACES content must be viewed as intended at any stage of production. Specific viewing pipelines may require different elements, so the exact viewing configuration used by a user making creative decisions must be recorded prior to shipment to another user. The ACES image sequence shall be displayed in an application with this ``last used'' viewing configuration, but a user may override the configured settings.

The \textit{aces:Config} xml tag is used to set the viewing pipeline to match the viewing conditions recorded in the ACESclip file. The ODT for the current viewing display may be used instead of the \textit{aces:Config} ODT, but the user should be warned if they are not of the same class of display, e.g. Rec.709 used previously and an HDR display is the current display.

\section{User Management of the Viewing Pipeline}
Users may override an application’s ACES viewing pipeline at any time.  Applications must manage the conversion between various ACES-compatible images and the user-selected working spaces.

When ASC CDL metadata is used, conversions to and from the ACEScc working space are saved in the \textit{aces:Config} metadata (wherein those particular CDL values must be applied).

\section{Saving the State of the ACES Viewing Pipeline}
The last state of the ACES Viewing Pipeline used to view an image sequence referenced by an ACESclip file must be recorded in the ACESclip file when a clip is closed or exported unless the user overrides this and does not want the clip to be changed.

\section{Creating an Archive Metadata Link}
When an ACES image sequence is created, placing identification traceable to source media in the \textit{aces:clipID} field is recommended.

\section{Reading and Writing LMTs}
The \textit{aces:TransformLibrary} XML element is used to transfer the actual transforms for LMTs to other users and facilities since these often may be custom LUTs. Applications shall read and write the XML structures containing CLF files.  An LMT combined with an RRT and ODT can be provided as well as a transform that simply contains the LMT. A stand-alone LMT must be merged with the other transforms in the basic ACES viewing pipeline for a user to look at the image.

\section{Reading and Writing ACESclip Files}
Applications shall support reading and writing of all XML elements described in this document. Recognition of extensions to the ACESclip specification developed by third parties is optional. However, if extensions are present, applications shall preserve them without change.

The ACESclip file may contain any one or all of the top-level XML structures (\textit{aces:clipID}, \textit{aces:Config}, \textit{aces:TransformLibrary}). For any particular XML file, these are listed as optional. However, production requirements determine which structures must be present in an ACESclip file.
% This file contains the content for a main section
\regularsectionformat	% Change formatting to that of a main, numbered section
%% Modify below this line %%

% Define a local macro - FieldName, description, comment
\newcommand{\xmlfield}[3]{
	\TabPositions{2em,1.75in,2.5in,2.6in}
	\tab\texttt{#1} \tab#2 \tab// #3	 \par
}

\chapter{Data Model}

This section describes the data intended for use within the ACES Metadata file.

All top level structures shall be tagged as being within the \texttt{aces} namespace with urn \\ \texttt{urn:acesMetadata:acesMetadataFile:v1.0}

\section{UML Diagram}
\begin{figure}[H]
  \centering
  \includegraphics[height=7.5in]{./images/amf_uml.pdf}
\end{figure}
\newpage

\section{Types}

The following types are defined for use within the AMF XML file and are validated with the XSD schema included in Appendix A.  The types are used as the basis to form the elements listed in section X in the schema.

\newcommand{\simpleType}[3]{
    \subsubsection{\textbf{ {\texttt{#1}} }}
    
    \textbf{Description:} 
    \begin{adjustwidth}{5mm}{} \vspace{-1.5mm}
        #2
    \end{adjustwidth}
    
    \textbf{Base Type:} 
    \begin{adjustwidth}{5mm}{} \vspace{-1.5mm}
        #3
    \end{adjustwidth}
    
    \textbf{Restrictions:} \vspace{-1.5mm}
}

\newcommand{\complexType}[4]{
    \subsubsection{\textbf{ {\texttt{#1}} }}
    
    \textbf{Description:}
    \begin{adjustwidth}{5mm}{} \vspace{-1.5mm}
        #2
    \end{adjustwidth}
    
    \textbf{Base Type:}
    \begin{adjustwidth}{5mm}{} \vspace{-1.5mm}
        #3
    \end{adjustwidth}
    
    \textbf{Children:}
    \begin{adjustwidth}{5mm}{} \vspace{-1.5mm}
        \texttt{#4}
    \end{adjustwidth}
}

\newcommand{\complexTypeNCWA}[4]{
    \subsubsection{\textbf{ {\texttt{#1}} }}
    
    \textbf{Description:}
    \begin{adjustwidth}{5mm}{} \vspace{-1.5mm}
        #2
    \end{adjustwidth}
    
    \textbf{Base Type:}
    \begin{adjustwidth}{5mm}{} \vspace{-1.5mm}
        #3
    \end{adjustwidth}
    
    \textbf{Attributes:}
    \begin{adjustwidth}{5mm}{} \vspace{-1.5mm}
        \texttt{#4}
    \end{adjustwidth}
}

\newcommand{\complexTypeCWA}[5]{
    \subsubsection{\textbf{ {\texttt{#1}} }}
    
    \textbf{Description:}
    \begin{adjustwidth}{5mm}{} \vspace{-1.5mm}
        #2
    \end{adjustwidth}
    
    \textbf{Base Type:}
    \begin{adjustwidth}{5mm}{} \vspace{-1.5mm}
        #3
    \end{adjustwidth}
    
    \textbf{Children:}
    \begin{adjustwidth}{5mm}{} \vspace{-1.5mm}
        \texttt{#4}
    \end{adjustwidth}
    
    \textbf{Attributes:}
    \begin{adjustwidth}{5mm}{} \vspace{-1.5mm}
        \texttt{#5}
    \end{adjustwidth}
}


\subsection{Simple Types}

% emailAddressType
\simpleType{aces:emailAddressType}{Type defining a restricted string conforming to an email address}{Restriction of \texttt{xs:string}}
            
            \begin{adjustwidth}{5mm}{}            
                \lstinline{xs:pattern value="[^@]+@[^\.]+\..+"}
            \end{adjustwidth}

% hashAlgoType
\simpleType{aces:hashAlgoType}
            {Type defining valid hash algorithms that can be used to validate specified transforms.  The allowed algorithms are specified by the enumerated URIs in this type.}
            {restriction of \texttt{xs:anyURI}}
                
            \begin{adjustwidth}{5mm}{}            
                \lstinline{xs:enumeration value="http://www.w3.org/2001/04/xmlenc\#sha256"} \\
                \lstinline{xs:enumeration value="http://www.w3.org/2000/09/xmldsig\#sha1"} \\
                \lstinline{xs:enumeration value="http://www.w3.org/2001/04/xmldsig-more\#md5"}
            \end{adjustwidth}

% singleDigitType
\simpleType{aces:singleDigitType}
            {Type defining a single digit integer}
            {restriction of \texttt{xs:integer}}
                
            \begin{adjustwidth}{5mm}{}            
            \lstinline{xs:totalDigits value="1"} \\
            \lstinline{xs:totalDigits value="0"} \\
            \lstinline{xs:totalDigits value="9"}
        \end{adjustwidth}
        
% tnColorSpaceConversionTransform
\simpleType{aces:tnColorSpaceConversionTransform}
            {Type defining valid transformID strings for Color Space Conversion transforms}
            {Restriction of \texttt{xs:string}}
            
            \begin{adjustwidth}{5mm}{}            
                \lstinline{xs:pattern value="urn:ampas:aces:transformId:v1.5:(ACEScsc\.\S+\.\S+\.a\d+\.v\d+|ACEScsc\.Academy\.\S+\.a\d+\.\d+\.\d+)"}
            \end{adjustwidth}

% tnInputTransform
\simpleType{aces:tnInputTransform}
            {Type defining valid transformID strings for InputTransform transforms}
            {Restriction of \texttt{xs:string}}

            \begin{adjustwidth}{5mm}{}            
            \lstinline{xs:pattern value=urn:ampas:aces:transformId:v1.5:IDT\.\S+\.\S+\.a\d+\.v\d+}
            \end{adjustwidth}
            
% tnInverseOutputDeviceTransform
\simpleType{aces:tnInverseOutputDeviceTransform}
            {Type defining valid transformID strings for InverseOutputDeviceTransform transforms}
            {Restriction of \texttt{xs:string}}

            \begin{adjustwidth}{5mm}{}            
            \lstinline{xs:pattern value=urn:ampas:aces:transformId:v1.5:(InvODT\.\S+\.\S+\.a\d+\.v\d+|InvODT\.Academy\.\S+\.a\d+\.\d+\.\d+)}
            \end{adjustwidth}
            
% tnInverseOutputTransform
\simpleType{aces:tnInverseOutputTransform}
            {Type defining valid transformID strings for InverseOutputTransform transforms}
            {Restriction of \texttt{xs:string}}

            \begin{adjustwidth}{5mm}{}            
            \lstinline{xs:pattern value=urn:ampas:aces:transformId:v1.5:(InvRRTODT\.\S+\.\S+\.a\d+\.v\d+|InvRRTODT\.Academy\.\S+\.a\d+\.\d+\.\d+)}
            \end{adjustwidth}
            
% tnInverseReferenceRenderingTransform
\simpleType{aces:tnInverseReferenceRenderingTransform}
            {Type defining valid transformID strings for tnInverseReferenceRenderingTransform transforms}
            {Restriction of \texttt{xs:string}}

            \begin{adjustwidth}{5mm}{}            
            \lstinline{xs:pattern value=urn:ampas:aces:transformId:v1.5:InvRRT\.a\d+\.\d+\.\d+}
            \end{adjustwidth}

% tnLookTransform
\simpleType{aces:tnLookTransform}
            {Type defining valid transformID strings for lookTransform transforms}
            {Restriction of \texttt{xs:string}}

            \begin{adjustwidth}{5mm}{}            
            \lstinline{xs:pattern value="urn:ampas:aces:transformId:v1.5:(LMT\.\S+\.\S+\.a\d+\.v\d+|LMT\.Academy\.\S+\.a\d+\.\d+\.\d+)"}
            \end{adjustwidth}

% tnOutputDeviceTransform
\simpleType{aces:tnOutputDeviceTransform}
            {Type defining valid transformID strings for OutputDeviceTransform transforms}
            {Restriction of \texttt{xs:string}}
            
            \begin{adjustwidth}{5mm}{}            
            \lstinline{xs:pattern value="urn:ampas:aces:transformId:v1.5:(ODT\.\S+\.\S+\.a\d+\.v\d+|ODT\.Academy\.\S+\.a\d+\.\d+\.\d+)"}
            \end{adjustwidth}

% tnOutputTransform
\simpleType{aces:tnOutputTransform}
            {Type defining valid transformID strings for OutputTransform transforms}
            {Restriction of \texttt{xs:string}}

            \begin{adjustwidth}{5mm}{}            
            \lstinline{xs:pattern value="urn:ampas:aces:transformId:v1.5:(RRTODT\.\S+\.\S+\.a\d+\.v\d+|RRTODT\.Academy\.\S+\.a\d+\.\d+\.\d+)"}
            \end{adjustwidth}

% tnReferenceRenderingTransform
\simpleType{aces:tnReferenceRenderingTransform}
            {Type defining valid transformID strings for OutputTransform transforms}
            {Restriction of \texttt{xs:string}}
            
            \begin{adjustwidth}{5mm}{}            
            \lstinline{xs:pattern value="urn:ampas:aces:transformId:v1.5:RRT\.a\d+\.\d+\.\d+"}
            \end{adjustwidth}

\subsection{Complex Types}

% authorType
\complexType{aces:authorType}
            {Type defining a sequence of elements used to communicate information about the author of the AMF file}
            {\texttt{xs:sequence}}
            {aces:emailAddress, aces:name}
            
% cdlWorkingSpaceType
\complexType{aces:cdlWorkingSpaceType}
            {Type defining the elements to communicate information about the transforms to and from the ASC-CDL Working Space}
            {\texttt{xs:sequence}}
            {aces:fromCdlWorkingSpace, aces:toCdlWorkingSpace}
            
% clipIdType
\complexType{aces:clipIdType}
            {Type defining elements used to communicate information about the essence associated with the AMF}
            {\texttt{xs:sequence}}
            {aces:clipName, aces:file, aces:sequence, aces:uuid}
            
% dateTimeType
\complexType{aces:dateTimeType}
            {Type defining the elements to communicate information about the creation and modification date and time associated with various AMF elements.}
            {\texttt{xs:sequence}}
            {aces:creationDateTime, aces:modificationDateTime}

% hashType
\complexTypeNCWA{aces:hashType}
            {Type defining the element to communicate information about a cryptographic file hash associated with file referenced by the AMF.}
            {extension of \texttt{xs:base64Binary}}
            {algorithm}
            
% infoType
\complexType{aces:infoType}
            {Type defining the elements to communicate description, date and time, and UUID information.}
            {\texttt{xs:sequence}}
            {aces:dateTime, aces:description, aces:uuid}
            
% inputTransformType
\complexTypeCWA{aces:inputTransformType}
                {Type defining the elements to communicate information about an ACES Input Transform associated with an ACES viewing pipeline.}
                {extension of \texttt{aces:transformType}}
                {aces:description, aces:hash, aces:transformId}
                {applied}
                
% inverseOutputDeviceTransformType
\complexType{aces:inverseOutputDeviceTransformType}
                {Type defining the elements to communicate information about an ACES Inverse Output Device Transform.  This type is used to define an inverse ACES pipeline to specify how output referred image data should be converted to ACES.}
                {extension of \texttt{aces:transformType}}
                {aces:description, aces:hash, aces:transformId}
                
% inverseOutputTransformType
\complexType{aces:inverseOutputTransformType}
                {Type defining the elements to communicate information about an ACES Inverse Output Transform. This type is used to define an inverse ACES pipeline to specify how output referred image data should be converted to ACES.}
                {extension of texttt{aces:transformType}}
                {aces:description, aces:hash, aces:transformId}

% inverseReferenceRenderingTransformType
\complexType{aces:inverseReferenceRenderingTransformType}
                {Type defining the elements to communicate information about an ACES Inverse Reference Rendering Transform.  This type is used to define an inverse ACES pipeline to specify how output referred image data should be converted to ACES.}
                {extension of \texttt{aces:transformType}}
                {aces:description, aces:hash, aces:transformId}

% lookTransformType
\complexTypeCWA{aces:lookTransformType}
                {Type defining the elements to communicate information about an ACES Look Transform associated with an ACES viewing pipeline.}
                {extension of \texttt{aces:transformType}}
                {aces:description, aces:hash, aces:lookTransformWorkingSpace,\\ aces:transformId, aces:uuid, cdl:ColorCorrectionRef, cdl:SOPNode, cdl:SatNode}
                {applied}
                
% outputDeviceTransformType
\complexTypeCWA{aces:outputDeviceTransformType}
                {Type defining the elements to communicate information about an ACES Output Device Transform associated with an ACES viewing pipeline.}
                {extension of \texttt{aces:transformType}}
                {aces:description, aces:hash, aces:transformId}
                {applied}

% outputTransformType
\complexTypeCWA{aces:outputTransformType}
                {Type defining the elements to communicate information about an ACES Output Transform associated with an ACES viewing pipeline.}
                {extension of texttt{aces:transformType}}
                {aces:description, aces:hash, aces:outputDeviceTransform,\\ aces:referenceRenderingTransform, aces:transformId}
                {applied}

% pipelineInfoType
\complexType{aces:pipelineInfoType}
            {Type defining the elements to communicate description, author, date and time, UUID information, and ACES version information.}
            {extension of \texttt{aces:infoType}}
            {aces:author, aces:dateTime, aces:description, aces:systemVersion, aces:uuid}
            
% pipelineType
\complexType{aces:pipelineType}
            {Type defining a sequence of elements used to communicate an ACES viewing pipeline}
            {xs:sequence}
            {aces:pipelineInfo, aces:inputTransform, aces:lookTransforms,\\ aces:outputTransform}
 
% referenceRenderingTransformType
\complexType{aces:referenceRenderingTransformType}
            {Type defining elements used to communicate the ACES Reference Rendering Transform associated with an ACES viewing pipeline.}
            {extension of \texttt{aces:transformType}}
            {aces:description, aces:hash, aces:transformId}

% sequenceType
\complexTypeNCWA{aces:sequenceType}
                {Type defining elements used to communicate information about a file sequence associated with an AMF.}
                {extension of \texttt{xs:string}}
                {idx, min, max}

% transformType
\complexType{aces:transformType}
            {Type defining elements used to communicate information about ACES transforms.  This type is used as the basis for other complex types.}
            {\texttt{xs:sequence}}
            {aces:description, aces:hash}

% versionType
\complexType{aces:versionType}
            {Type defining elements used to communicate ACES system version information.}
            {\texttt{xs:sequence}}
            {aces:majorVersion, aces:minorVersion, aces:patchVersion}

% workingSpaceTransformType
\complexType{aces:workingSpaceTransformType}
            {Type defining elements used to communicate the Color Space conversion transform used to convert between the working color space associated with a particular look transform and ACES 2065-1.}
            {extension of \texttt{aces:transformType}}
            {aces:description, aces:hash, aces:transformId}



\newpage
\section{Elements}
%\settocdepth{section}

The following elements are defined for use with the AMF XML file and are validated with the XSD schema included in Appendix A. 

\newcommand\element[9]{
    \def\tempa{#1}%
    \def\tempb{#2}%
    \def\tempc{#3}%
    \def\tempd{#4}%
    \def\tempe{#5}%
    \def\tempf{#6}%
    \def\tempg{#7}%
    \def\temph{#8}%
    \def\tempi{#9}%
    \elementcontinued
}


\newcommand\elementcontinued[3]{
    \subsection{\textbf{ {\texttt{\tempa}} }}

    \textbf{Description:}
    \begin{adjustwidth}{5mm}{} \vspace{-1.5mm}
        \tempb
    \end{adjustwidth}

    \textbf{Diagram:}
    \begin{figure}[H]
        \includegraphics[width=2.75in]{\tempc}
    \end{figure}

    \textbf{Type:}
    \begin{adjustwidth}{5mm}{} \vspace{-1.5mm}
        \texttt{\tempd}
    \end{adjustwidth}

    \textbf{Required or Optional:}
    \begin{adjustwidth}{5mm}{} \vspace{-1.5mm}
        \tempe
    \end{adjustwidth}

    \textbf{Occurrences:}
    \begin{adjustwidth}{5mm}{} \vspace{-1.5mm}
        Min: \tempf~Max: \tempg
    \end{adjustwidth}

    \textbf{Attributes:}
    \begin{adjustwidth}{5mm}{} \vspace{-1.5mm}
        Required: \texttt{\temph} \\
        Optional: \texttt{\tempi}
    \end{adjustwidth}

    \textbf{Parent:}
    \begin{adjustwidth}{5mm}{} \vspace{-1.5mm}
        #1
    \end{adjustwidth}

    \textbf{Children:}
    \begin{adjustwidth}{5mm}{} \vspace{-1.5mm}
        #2
    \end{adjustwidth}

    \textbf{Example:}
    \begin{adjustwidth}{5mm}{} \vspace{-1.5mm}
        #3
    \end{adjustwidth}
}

% acesMetadataFile
\element{aces:MetadataFile}
        {The top level element of an ACES Metadata File.  This element defines first level child elements.}
        {images/acesMetadataFile_xsd_Element_aces_acesMetadataFile.png}
        {xs:element}
        {Required}
        {1}{1}
        {\texttt{version="1.0", xmlns:aces="urn:ampas:aces:amf:v1.0"}}{\texttt{xmlns, xsi:schemeLocation}}
        {None}
        {\texttt{aces:pipeline, aces:archivedPipeline, aces:clipId, aces:amfInfo}}
        { \lstinline{<aces:MetadataFile} \\
        \lstinline{xmlns:aces="urn:ampas:aces:amf:v1.0"} \\
        \lstinline{xsi:schemaLocation="urn:ampas:aces:amf:v1.0 file:acesMetadataFile.xsd"} \\ \lstinline{xmlns:cdl="urn:ASC:CDL:v1.01"}\\
        \lstinline{xmlns:xsi="http://www.w3.org/2001/XMLSchema-instance"}\\
        \lstinline{version="1.0">}\\
        ... \\
        \lstinline{</aces:MetadataFile>}}

% amfInfo
\element{aces:amfInfo}
        {This element contains all the elements containing information about the AMF itself including date and time information, a description element, and a UUID element.}
        {images/acesMetadataFile_xsd_Element_aces_amfInfo.png}
        {aces:infoType}
        {Required}
        {1}{1}
        {none}{none}
        {\texttt{aces:MetadataFile}}
        {aces:author, aces:dateTime, aces:description, aces:uuid}
        {\lstinline{<aces:amfInfo>} \\
        ... \\
        \lstinline{</aces:amfInfo>}}

% archivedPipeline
 \element{aces:archivedPipeline}
        {This element contains all the elements describing an ACES viewing pipeline archived for historical purposes.}
        {images/acesMetadataFile_xsd_Element_aces_archivedPipeline.png}
        {aces:pipelineType}
        {Optional}
        {0}
        {unbounded}
        {none}{none}
        {\texttt{aces:MetadataFile}}
        {\texttt{aces:inputTransform, aces:lookTransform, aces:outputTransform, \\ aces:pipelineInfo}}
        {\lstinline{<aces:archivedPipeline>} \\
        ... \\
        \lstinline{</aces:archivedPipeline>}}

% clipId
\element{aces:clipId}
        {This optional element contains all the elements describing the location of the media files associated with the AMF.}
        {images/acesMetadataFile_xsd_Element_aces_clipId.png}
        {aces:clipIdType}
        {Optional}{0}{1}
        {none}{none}{\texttt{aces:MetadataFile}}
        {\texttt{aces:clipName, aces:file, aces:sequence, aces:uuid}}
        {\lstinline{<aces:clipId>} \\
        ... \\
        \lstinline{</aces:clipId>}}

% pipeline
\element{aces:pipeline}
        {This element contains all the elements describing the ACES viewing pipeline.}
        {images/acesMetadataFile_xsd_Element_aces_pipeline.png}
        {aces:pipelineType}
        {Required}{1}{1}
        {none}{none}
        {\texttt{aces:MetadataFile}}
        {\texttt{aces:inputTransform, aces:lookTransform, aces:outputTransform, \\ aces:pipelineInfo}}
        {\lstinline{<aces:pipeline>} \\
        ... \\
        \lstinline{</aces:pipeline>}}

% emailAddress
\element{aces:emailAddress}
        {This element used to communicate the AMF author's email address.}
        {images/acesMetadataFile_xsd_Element_aces_emailAddress.png}
        {aces:emailAddressType}
        {Required}{1}{1}
        {none}{none}
        {\texttt{aces:author}}
        {None}
        {\lstinline{<aces:emailAddress>joe@onset.com</aces:emailAddress>}}

% name
\element{aces:name}
        {This element is used to communicate the name of the AMF author.}
        {images/acesMetadataFile_xsd_Element_aces_name.png}
        {\texttt{xs:string}}
        {Required}{1}{1}
        {None}{None}{\texttt{aces:author}}{None}{\lstinline{<aces:name>Joe Onset</aces:name>}}

% aces:fromCdlTransformWorkingSpace
\element{aces:fromCdlWorkingSpace}
        {This element contains all the elements describing the transform used to convert from the working color space in which an ASC-CDL is applied to ACES 2065-1.}
        {images/acesMetadataFile_xsd_Element_aces_fromCdlWorkingSpace.png}
        {\texttt{aces:workingSpaceTransformType}}
        {Required}{1}{1}
        {None}{None}
        {\texttt{aces:cdlWorkingSpace}}
        {aces:description, aces:hash, aces:transformId}
        {\lstinline{<aces:fromCdlWorkingSpace>} \\
        ... \\
        \lstinline{</aces:fromCdlWorkingSpace>}}

% aces:toCdlWorkingSpace
\element{aces:toCdlWorkingSpace}
        {This element contains all the elements describing the transform used to convert from ACES 2065-1 to the working color space in which a ASC-CDL transform is applied.  This transform shall be included when the working color space for the ASC-CDL Transform is not a working color space described in one of the Color Space Conversion transform included in the ACES core transforms.  When the working color space for the ASC-CDL Transform is a working color space described in one of the Color Space Conversion transform included in the ACES core transforms, the \texttt{aces:toCdlWorkingSpace} is optional.}
        {images/acesMetadataFile_xsd_Element_aces_toCdlWorkingSpace.png}
        {\texttt{aces:workingSpaceTransformType}}
        {Optional}{0}{1}
        {None}{None}
        {\texttt{aces:cdlWorkingSpace}}
        {\texttt{aces:description, aces:hash, aces:transformId}}
        {\lstinline{<aces:toCdlWorkingSpace>} \\
        ... \\
        \lstinline{</aces:toCdlWorkingSpace>}}

% clipName
\element{aces:clipName}
        {This element is used to communicate the clip name associated with the media files.}
        {images/acesMetadataFile_xsd_Element_aces_clipName.png}
        {\texttt{xs:string}}
        {Required}{1}{1}
        {None}{None}{\texttt{aces:clipId}}{None}{\lstinline{<aces:clipName>A001C012</aces:clipId>}}

% file
\element{aces:file}
        {This element is used to communicate the name of the media file.  Care should be taken when using the file name as an identifier as file locations and names typically change during production and post-production.}
        {images/acesMetadataFile_xsd_Element_aces_file.png}
        {\texttt{xs:anyURI}}
        {Choice of \texttt{aces:file}, \texttt{aces:sequence} or \texttt{aces:uuid} is required}{1}{1}
        {None}{None}{\texttt{aces:clipId}}{None}
        {\lstinline{<aces:file>file:///foo.mxf</aces:file>}}

% sequence
\element{aces:sequence}
        {This element is used to communicate the file sequence information associated with the media files.  The file sequence includes an index indicated by the \texttt{idx} attribute (e.g. \#) that is used to denote the location of frame numbers within the sequence string.  The \texttt{min} and \texttt{max} attributes are used to indicate the minimum frame number and maximum frame number of the sequence.  For example, if the sequence string is \texttt{movieFrame\#\#\#\#.exr} and attributes of \texttt{aces:sequence} are \texttt{idx='\#'}, \texttt{min='0'} and \texttt{min='1000'} the the media files associated with the AMF would be the frames numbered \texttt{movieFrame0000.exr} through \texttt{movieFrame1000.exr}}
        {images/acesMetadataFile_xsd_Element_aces_sequence.png}
        {\texttt{aces:sequenceType}}
        {Choice of \texttt{aces:file}, \texttt{aces:sequence} or \texttt{aces:uuid} is required}{1}{1}
        {\texttt{idx}, \texttt{min}, \texttt{max}}{None}{\texttt{aces:clipId}}{None}
        {\lstinline{<aces:sequence idx="#" min="1" max="240">A01_C012_AE0306_###.exr</aces:sequence>}}

% aces:clipIdType / aces:uuid
\element{aces:clipIdType / aces:uuid}
		{This element is used to communicate a UUID associated with the media files referred to in the ClipID.}
		{images/acesMetadataFile_xsd_Element_aces_uuid.png}
		{\texttt{dcml:UUIDType}}
		{Choice of \texttt{aces:file}, \texttt{aces:sequence} or \texttt{aces:uuid} is required}{1}{1}
		{None}{None}
		{\texttt{aces:clipId}}{None}
		{\lstinline{<aces:uuid>urn:uuid:797c7cd8-4eb1-4f67-afce-af2b0a1d0285</aces:uuid>}}

% creationDateTime
\element{aces:creationDateTime}
		{This element is used to communicate the creation date and time of an AMF file or an ACES pipeline.}
		{images/acesMetadataFile_xsd_Element_aces_creationDateTime.png}
		{\texttt{xs:dateTime}}
		{Required}{1}{1}
		{None}{None}
		{\texttt{aces:dateTime}}{None}
		{\lstinline{<aces:creationDateTime>2020-11-10T13:20:00Z</aces:creationDateTime>}}

% modificationDateTime
\element{aces:modificationDateTime}
		{This element is used to communicate the most recent modification date and time of an AMF file or an ACES pipeline.}
		{images/acesMetadataFile_xsd_Element_aces_modificationDateTime.png}
		{\texttt{xs:dateTime}}
		{Required}{1}{1}
		{None}{None}
		{\texttt{aces:dateTime}}{None}
		{\lstinline{<aces:modificationDateTime>2020-11-10T13:20:00Z</aces:modificationDateTime>}}

% author
\element{aces:author}
		{This element contains all the elements describing the AMF author information.}
		{images/acesMetadataFile_xsd_Element_aces_author.png}
		{\texttt{xs:sequence}}
		{Optional}{1}{unbounded}
		{None}{None}
		{\texttt{aces:amfInfo}}{\texttt{aces:name, aces:emailAddress}}
		{\lstinline{<aces:author>} \\
        ... \\
        \lstinline{</aces:author>}}

% dateTime
\element{aces:dateTime}
		{This element contains all the elements describing the date and time of the creation and modification of the AMF.}
		{images/acesMetadataFile_xsd_Element_aces_dateTime.png}
		{\texttt{xs:sequence}}
		{Required}{1}{1}
		{None}{None}
		{\texttt{aces:amfInfo}}{\texttt{aces:creationDateTime, modificationDateTime}}
		{\lstinline{<aces:dateTime>} \\
        ... \\
        \lstinline{</aces:dateTime>}}

% description
\element{aces:description}
		{This element is used to communicate description information for an AMF file, an ACES pipeline, or various ACES viewing transforms.}
		{images/acesMetadataFile_xsd_Element_aces_description.png}
		{\texttt{xs:string}}
		{Optional}{0}{1}
		{None}{None}
		{\texttt{aces:amfInfo, aces:piplineInfo, aces:toCdlWorkingSpace,\\ aces:fromCdlWorkingSpace, aces:inputTransform,\\
		  aces:lookTransform, aces:outputTransform, aces:outputDeviceTransform, \\
		  aces:referenceRenderingTransform}}{None}
		{\lstinline{<aces:description>Example Movie</aces:description>}\\
		 \lstinline{<aces:description>Technical Grade</aces:description>}}

% aces:infoType / aces:uuid
\element{aces:infoType / aces:uuid}
		{This element is used to communicate a UUID associated with the AMF or an ACES pipeline.}
		{images/acesMetadataFile_xsd_Element_aces_uuid.png}
		{\texttt{dcml:UUIDType}}
		{Optional}{0}{1}
		{None}{None}
		{\texttt{aces:amfInfo, aces:pipelineInfo}}{None}
		{\lstinline{<aces:uuid>urn:uuid:797c7cd8-4eb1-4f67-afce-af2b0a1d0285</aces:uuid>}}

%  aces:inverseOutputDeviceTransform
\element{aces:inverseOutputDeviceTransform}
        {This element contains all the elements describing the transforms associated with an inverse output device transform used to convert output referred images to ACES.}
        {images/acesMetadataFile_xsd_Element_aces_inverseOutputDeviceTransform.png}
        {\texttt{aces:inverseOutputDeviceTransformType}}
        {Required}{1}{1}
        {None}{None}
        {aces:description, aces:hash, aces:transformId}
        {\lstinline{<aces:inverseOutputDeviceTransform>} \\
        ... \\
        \lstinline{</aces:inverseOutputDeviceTransform>}}

%  aces:inverseOutputTransform
\element{aces:inverseOutputTransform}
        {This element contains all the elements describing the transforms associated with an inverse output  transform used to convert output referred images to ACES.}
        {images/acesMetadataFile_xsd_Element_aces_inverseOutputTransform.png}
        {\texttt{aces:inverseOutputTransformType}}
        {Required}{1}{1}
        {None}{None}
        {aces:description, aces:hash, aces:transformId}
        {\lstinline{<aces:inverseOutputTransform>} \\
        ... \\
        \lstinline{</aces:inverseOutputTransform>}}

%  aces:inverseReferenceRenderingTransform
\element{aces:inverseReferenceRenderingTransform}
        {This element contains all the elements describing the transforms associated with an inverse reference rendering transform used to convert output referred images to ACES.}
        {images/acesMetadataFile_xsd_Element_aces_inverseReferenceRenderingTransform.png}
        {\texttt{aces:inverseReferenceRenderingTransformType}}
        {Required}{1}{1}
        {None}{None}
        {aces:description, aces:hash, aces:transformId}
        {\lstinline{<aces:inverseReferenceRenderingTransform>} \\
        ... \\
        \lstinline{</aces:inverseReferenceRenderingTransform>}}

% aces:inputTransformType / aces:transformId
\element{aces:inputTransformType / aces:transformId}
        {This element is used to communicate the transformID of an ACES Input Transform that transforms images encoded in a color space of a camera native file to ACES 2065-1.  For more information on transformIDs see S-2014-002 Academy Color Encoding System -- Versioning system.  Valid transforms for this element are Input Transforms.  The element is restricted to enforce the use of transformIDs that follow the IDT naming conventions established in the versioning system specification.  As noted in the versioning system specification, manufacturer and user created transforms shall be assigned a transformID according to patterns established in the document.}
        {images/acesMetadataFile_xsd_Element_aces_transformId.png}
        {\texttt{aces:tnInputTransform}}
        {Required}{1}{1}
        {None}{None}
        {\texttt{aces:inputTransform}}{None}
        {\lstinline{<aces:transformId>} \\
        \lstinline{urn:ampas:aces:transformId:v1.5:IDT.Sony.F65.a1.v1} \\
        \lstinline{</aces:transformId>}}

% aces:inverseOutputDeviceTransformType / aces:transformId
\element{aces:inverseOutputDeviceTransformType / aces:transformId}
        {This element is used to communicate the transformID of an ACES Inverse Output Device Transform that transforms images encoded in an output referred color space to OCES.  For more information on transformIDs see S-2014-002 Academy Color Encoding System -- Versioning system.  Valid transforms for this element are Input Transforms.  The element is restricted to enforce the use of transformIDs that follow the InvODT naming conventions established in the versioning system specification.  As noted in the versioning system specification, manufacturer and user created transforms shall be assigned a transformID according to patterns established in the document.}
        {images/acesMetadataFile_xsd_Element_aces_transformId_2.png}
        {\texttt{aces:tnInverseDeviceOutputTransform}}
        {Required}{1}{1}
        {None}{None}
        {\texttt{aces:inputTransform}}{None}
        {\lstinline{<aces:transformId>} \\
        \lstinline{urn:ampas:aces:transformId:v1.5:InvODT.Academy.Rec709_100nits_dim.a1.0.3} \\
        \lstinline{</aces:transformId>}}

% aces:inverseOutputTransformType / aces:transformId
\element{aces:inverseOutputTransformType / aces:transformId}
        {This element is used to communicate the transformID of an ACES Inverse Output Transform that transforms images encoded in an output referred color space to ACES 2065-1.  For more information on transformIDs see S-2014-002 Academy Color Encoding System -- Versioning system.  Valid transforms for this element are Input Transforms.  The element is restricted to enforce the use of transformIDs that follow the InvRRTODT naming conventions established in the versioning system specification.  As noted in the versioning system specification, manufacturer and user created transforms shall be assigned a transformID according to patterns established in the document.}
        {images/acesMetadataFile_xsd_Element_aces_transformId_2.png}
        {\texttt{aces:tnInverseReferenceRenderingTransform}}
        {Required}{1}{1}
        {None}{None}
        {\texttt{aces:inputTransform}}{None}
        {\lstinline{<aces:transformId>}\\
        \lstinline{urn:ampas:aces:transformId:v1.5:InvRRT.a1.0.3}\\
        \lstinline{</aces:transformId>}}

% aces:inverseReferenceRenderingTransformType / aces:transformId
\element{aces:inverseReferenceRenderingTransformType / aces:transformId}
        {This element is used to communicate the transformID of an ACES Inverse Reference Rendering Transform that transforms images encoded in OCES color space to ACES 2065-1.  For more information on transformIDs see S-2014-002 Academy Color Encoding System -- Versioning system.  Valid transforms for this element are Input Transforms.  The element is restricted to enforce the use of transformIDs that follow the InvRRT naming conventions established in the versioning system specification.  As noted in the versioning system specification, manufacturer and user created transforms shall be assigned a transformID according to patterns established in the document.}
        {images/acesMetadataFile_xsd_Element_aces_transformId_3.png}
        {\texttt{aces:tnInverseReferenceRenderingTransform}}
        {Required}{1}{1}
        {None}{None}
        {\texttt{aces:inputTransform}}{None}
        {\lstinline{<aces:transformId>} \\
        \lstinline{urn:ampas:aces:transformId:v1.5:InvRRTODT.Academy.Rec2020_1000nits_15nits_ST2084.a1.1.0}\\
        \lstinline{</aces:transformId>}}

%aces:cdlWorkingSpace
\element{aces:cdlWorkingSpace}
        {This element contains all the elements describing the transforms used to convert to and from the working color space in which a ASC-CDL transform is applied. This element allows for CDLs to be applied in color spaces other than ACES RGB, since CDLs cannot contain ACES transforms themselves. The input and output of the parent \texttt{<lookTransform>} element is still ACES RGB per SMPTE ST.2065-1. }
        {images/acesMetadataFile_xsd_Element_aces_cdlWorkingSpace.png}
        {\texttt{aces:cdlWorkingSpaceType}}
        {Required}{1}{1}
        {None}{None}
        {\texttt{aces:lookTransform}}{\texttt{aces:fromCdlWorkingSpace, aces:toCdlWorkingSpace}}
        {\lstinline{<aces:cdlWorkingSpace>} \\
        ... \\
        \lstinline{</aces:cdlWorkingSpace>}}

% aces:lookTransformType / aces:transformId
\element{aces:lookTransformType / aces:transformId}
        {This element is used to communicate the transformID of an ACES Look Transform.  For more information on transformIDs see S-2014-002 Academy Color Encoding System -- Versioning system.  Valid transforms for this element are Look Transforms (LMT).  The element is restricted to enforce the use of transformIDs that follow the LMT naming conventions established in the versioning system specification.  As noted in the versioning system specification, manufacturer and user created transforms shall be assigned a transformID according to patterns established in the document.}
        {images/acesMetadataFile_xsd_Element_aces_transformId_8.png}
        {\texttt{aces:tnLookTransform}}
        {Choice of \texttt{aces:transformId, cdl:ColorCorrectionRef}, or \texttt{cdl:SOPNode} and \\ \texttt{cdl:SatNode} required}
        {0}{1}
        {None}{None}
        {\texttt{aces:lookTransform}}
        {None}
        {\lstinline{<aces:transformId>}\\
        \lstinline{urn:ampas:aces:transformId:v1.5:LMT.ACME.BleachBypass.a1.v1</aces:transformId>}\\
        \lstinline{</aces:transformId>}
        }

% aces:lookTransformType / aces:uuid
\element{aces:lookTransformType / aces:uuid}
		{This element is used to communicate a UUID associated with externally referenced ASC-CDLs.}
		{images/acesMetadataFile_xsd_Element_aces_uuid.png}
		{\texttt{dcml:UUIDType}}
		{Optional}{0}{1}
		{None}{None}
		{\texttt{aces:lookTransform}}{None}
		{\lstinline{<aces:uuid>urn:uuid:797c7cd8-4eb1-4f67-afce-af2b0a1d0285</aces:uuid>}}

% aces:outputDeviceTransformType / aces:transformId
\element{aces:outputDeviceTransformType / aces:transformId}
        {This element is used to communicate the transformID of the ACES Output Device Transform.  For more information on transformIDs see S-2014-002 Academy Color Encoding System -- Versioning system.  Valid transforms for this element are Output Transforms (ODT).  The element is restricted to enforce the use of transformIDs that follow the ODT naming conventions established in the versioning system specification.  As noted in the versioning system specification, manufacturer and user created transforms shall be assigned a transformID according to patterns established in the document.}
        {images/acesMetadataFile_xsd_Element_aces_transformId_5.png}
        {\texttt{aces:tnOutputDeviceTransform}}
        {Required}{1}{1}
        {None}{None}
        {\texttt{aces:outputDeviceTransform}}{None}
        {\lstinline{<aces:transformId>urn:ampas:aces:transformId:v1.5:ODT.Academy.P3D60_48nits.a1.0.0</aces:transformId>}}

% aces:outputDeviceTransform
\element{aces:outputDeviceTransform}
        {This element contains all the elements containing information about the ACES Output Device Transform for a given ACES viewing pipeline.}
        {images/acesMetadataFile_xsd_Element_aces_outputDeviceTransform.png}
        {aces:outputDeviceTransformType}
        {Choice of \texttt{aces:transformId} or \texttt{aces:outputDeviceTransform} and \\  \texttt{aces:referenceRenderingTransform} required}{1}{1}
        {None}{None}{\texttt{aces:OutputTransform}}
        {\texttt{aces:description, aces:hash, aces:transformId}}
        {\lstinline{<aces:outputDeviceTransform>} \\
        ... \\
        \lstinline{</aces:outputDeviceTransform>}}

% aces:referenceRenderingTransform
\element{aces:referenceRenderingTransform}
        {This element contains all the elements containing information about the ACES Reference Rendering Transform for a given ACES viewing pipeline.}
        {images/acesMetadataFile_xsd_Element_aces_referenceRenderingTransform.png}
        {\texttt{aces:referenceRenderingTransformType}}
        {Choice of \texttt{aces:transformId} or \texttt{aces:outputDeviceTransform} and \\  \texttt{aces:referenceRenderingTransform} required}
        {1}{1}{None}{None}
        {\texttt{aces:OutputTransform}}
        {\texttt{aces:description, aces:hash, aces:transformId}}
        {\lstinline{<aces:referenceRenderingTransform>} \\
        ... \\
        \lstinline{</aces:referenceRenderingTransform>}}

% aces:outputTransformType / aces:transformId
\element{aces:outputTransformType / aces:transformId}
        {This element is used to communicate the transformID of the ACES Output Transform.  For more information on transformIDs see S-2014-002 Academy Color Encoding System -- Versioning system.  Valid transforms for this element are Output Transforms (RRTODT).  The element is restricted to enforce the use of transformIDs that follow the RRTODT naming conventions established in the versioning system specification.  As noted in the versioning system specification, manufacturer and user created transforms shall be assigned a transformID according to patterns established in the document.}
        {images/acesMetadataFile_xsd_Element_aces_transformId_6.png}
        {\texttt{aces:tnOutputTransform}}
        {Required}{1}{1}
        {None}{None}
        {\texttt{aces:outputTransform}}
        {None}
        {\lstinline{<aces:transformId>}\\
        \lstinline{urn:ampas:aces:transformId:v1.5:RRTODT.Academy.Rec2020_1000nits_15nits_HLG.a1.1.0}\\
        \lstinline{</aces:transformId>}}

% aces:systemVersion
\element{aces:systemVersion}
        {This element contains all the elements containing information about the ACES version number associated with the ACES viewing pipeline.}
        {images/acesMetadataFile_xsd_Element_aces_systemVersion.png}
        {\texttt{aces:versionType}}
        {Required}
        {1}{1}
        {None}{None}
        {\texttt{aces:pipelineInfo}}{\texttt{aces:majorVersion, aces:minorVersion, aces:patchVersion}}
        {\lstinline{<aces:systemVersion>} \\
        ... \\
        \lstinline{</aces:systemVersion>}}

% aces:inputTransform
\element{aces:inputTransform}
        {This element contains all the elements containing information about the ACES input transform for a given ACES viewing pipeline.  The required \texttt{applied} attribute is used to indicate if the ACES input transform indicated has been applied to the media files or not.  If \texttt{applied="true"} the media files shall be encoded as according to SMPTE ST 2065-1. If \texttt{applied="false"} the media files may be transcoded to ACES using the transform indicated in the child element \texttt{transformId}.}
        {images/acesMetadataFile_xsd_Element_aces_inputTransform.png}
        {\texttt{aces:inputTransformType}}
        {Optional}{0}{1}
        {\texttt{applied (xs:boolean)}}{None}
        {\texttt{aces:pipeline}, \texttt{aces:archivedPipeline}}
        {\texttt{aces:description, aces:hash, aces:transformId}}
        {\lstinline{<aces:inputTransform>} \\
        ... \\
        \lstinline{</aces:inputTransform>}}

% aces:lookTransform
 \element{aces:lookTransform}
        {This element contains a look transform (LMT) for a given ACES viewing pipeline.  If the AMF includes multiple \texttt{<lookTransform>} elements, they shall be applied in the order in which they are written in the AMF (top to bottom).

        The required \texttt{applied} attribute is used to indicate if the ACES look transform has been applied to the media files or not.  If \texttt{applied="true"}, the media files shall have the look transform "baked" into the image data (but is still included for diagnostic purposes). If \texttt{applied="false"}, the media files shall not have the look transform "baked" into the image data.

        The input values and output values of ACES Look Transforms are ACES 2065-1. This is to avoid linking the Look Transforms to project specific working spaces. Look Transforms may convert ACES 2065-1 to a more appropriate working space for internal look application. Care should be taken when building Look Transforms as 3D LUTs, given Look Transforms input and output values are linear. In practice, smart implementations may modify the Look Transform to avoid unnecessary conversions within the context of an ACES pipeline as long as the results match those specified by the transforms in the AMF.

        ASC-CDL does not have a mechanism to convert to a non-linear working space appropriate for the application of ASC-CDL values. For this reason, the \texttt{<aces:cdlTransformWorkingSpace>} element can be used to indicate the working space via transformsIDs in which ASC-CDL values are to be applied.}
        {images/acesMetadataFile_xsd_Element_aces_lookTransform.png}
        {\texttt{aces:lookTransformType}}
        {Optional}{0}{unbounded}
        {\texttt{applied (xs:boolean)}}{None}
        {\texttt{aces:pipeline}, \texttt{aces:archivedPipeline}}
        {\texttt{aces:description, aces:hash, aces:CdlWorkingSpace, aces:transformId,\\  cdl:ColorCorrectionRef, aces:uuid, cdl:SOPNode, cdl:SatNode}}
        {\lstinline{<aces:lookTransform>} \\
        ... \\
        \lstinline{</aces:lookTransform>}}

% aces:outputTransform
 \element{aces:outputTransform}
        {This element contains all the elements containing information about the ACES output transform for a given ACES viewing pipeline.}
        {images/acesMetadataFile_xsd_Element_aces_outputTransform.png}
        {\texttt{aces:outputTransformType}}
        {Required}{1}{1}
        {None}{None}
        {\texttt{aces:pipeline}, \texttt{aces:archivedPipeline}}
        {\texttt{aces:description, aces:hash, aces:outputDeviceTransform, \\
        aces:referenceRenderingTransform, aces:transformId}}
        {\lstinline{<aces:outputTransform>} \\
        ... \\
        \lstinline{</aces:outputTransform>}}

% aces:pipelineInfo
\element{aces:pipelineInfo}
        {This element contains all the elements containing metadata information such as description, author, date and time, etc. for a given ACES viewing pipeline. }
        {images/acesMetadataFile_xsd_Element_aces_pipelineInfo.png}
        {\texttt{aces:piplineInfoType}}
        {Required}{1}{1}
        {None}{None}
        {\texttt{aces:pipeline, aces:archivedPipeline}}
        {\texttt{aces:author, aces:dateTime, aces:description, aces:systemVersion,\\ aces:uuid}}
        {\lstinline{<aces:pipelineInfo>} \\
        ... \\
        \lstinline{</aces:pipelineInfo>}}

% aces:referenceRenderingTransform / aces:transformId
\element{aces:referenceRenderingTransform / aces:transformId}
        {This element is used to communicate the transformID of the ACES Reference Rendering Transform.  For more information on transformIDs see S-2014-002 Academy Color Encoding System -- Versioning system.  Valid transforms for this element are Reference Rendering Transform (RRT).  The element is restricted to enforce the use of transformIDs that follow the RRT naming conventions established in the versioning system specification.}
        {images/acesMetadataFile_xsd_Element_aces_transformId_4.png}
        {\texttt{aces:tnReferenceRenderingTransform}}
        {Required}{1}{1}
        {None}{None}
        {\texttt{aces:outputTransform}}
        {None}
        {\lstinline{<aces:transformId>}\\
        \lstinline{urn:ampas:aces:transformId:v1.5:urn:ampas:aces:transformId:v1.5:RRT.a1.0.0}\\
        \lstinline{</aces:transformId>}}

% aces:hash
\element{aces:hash}
        {This element is used to communicate the cryptographic hash for a transform referenced by the AMF.}
        {images/acesMetadataFile_xsd_Element_aces_hash.png}
        {\texttt{aces:hashType}}
        {Optional}{0}{1}
        {\texttt{algorithm (restricted xs:anyURI)}}{None}
        {\texttt{aces:inputTransform, aces:lookTransform, aces:outputDeviceTransform,\\ aces:outputTransform, aces:referenceRenderingTransform}}{None}
        {
        \lstinline{<aces:hash algorithm="http://www.w3.org/2001/04/xmlenc#sha256">c81af4fb4a22ee}\\      \lstinline{0353308e4582708951df4682bf73f838c24bf44e585fc3bb61</aces:hash>}
        }

% aces:majorVersion
\element{aces:majorVersion}
        {This element contains information on the ACES system major version number associated with an ACES viewing pipeline.  If the system reading the AFM has not implemented the major version specified the system shall indicate that the major version of the system and AMF do not match and produce an error.}
        {images/acesMetadataFile_xsd_Element_aces_majorVersion.png}
        {\texttt{aces:singleDigitType}}
        {Required}{1}{1}
        {None}{None}
        {/texttt{aces:systemVersion}}{None}
        {\lstinline{<aces:majorVersion>1</aces:majorVersion>}}

% aces:minorVersion
\element{aces:minorVersion}
        {This element contains information on the ACES system minor version number associated with an ACES viewing pipeline.  If the system reading the AFM has not implemented the minor version specified the system shall indicate that the minor version of the system and AMF do not match and produce an error.}
        {images/acesMetadataFile_xsd_Element_aces_minorVersion.png}
        {\texttt{aces:singleDigitType}}
        {Required}{1}{1}
        {None}{None}
        {/texttt{aces:systemVersion}}{None}
        {\lstinline{<aces:minorVersion>2</aces:minorVersion>}}

% aces:minorVersion
\element{aces:patchVersion}
        {This element contains information on the ACES system patch version number associated with an ACES viewing pipeline.  If the system reading the AFM has not implemented the patch version specified the system shall indicate that the patch version of the system and AMF do not match with a warning and fall back to the most recent patch version available.}
        {images/acesMetadataFile_xsd_Element_aces_patchVersion.png}
        {\texttt{aces:singleDigitType}}
        {Required}{1}{1}
        {None}{None}
        {/texttt{aces:systemVersion}}{None}
        {\lstinline{<aces:patchVersion>2</aces:patchVersion>}}

% aces:workingSpaceTransformType / aces:transformId
\element{aces:workingSpaceTransformType / aces:transformId}
        {This element is used to communicate the transformID of the ACES Color Space Conversion Transform used to convert to or from the Look Transform working space.  For more information on transformIDs see S-2014-002 Academy Color Encoding System -- Versioning system.  Valid transforms for this element are Reference Rendering Transform (ACEScsc).  The element is restricted to enforce the use of transformIDs that follow the ACEScsc naming conventions established in the versioning system specification.}
        {images/acesMetadataFile_xsd_Element_aces_transformId_7.png}
        {\texttt{aces:tnColorSpaceConversionTransform}}
        {Required}{1}{1}
        {None}{None}
        {\texttt{aces:toCdlWorkingSpace, aces:fromCdlWorkingSpace}}
        {None}
        {\lstinline{<aces:transformId>}\\
        \lstinline{urn:ampas:aces:transformId:v1.5:ACEScsc.Academy.ACEScct_to_ACES.a1.0.3}\\
        \lstinline{</aces:transformId>}}

% cdl:SOPNode
\element{cdl:SOPNode}
        {This element is imported from the ASC-CDL schema (\texttt{ASC-CDL\_schema\_v1.01.xsd}).  It defines a Slope, Offset, Power node.  \texttt{<cdl:SOPNode>} may be substituted with \texttt{<cdl:ASC\_SOP>}.  See the ASC-CDL documentation for more information on its usage.}
        {images/ASC-CDL_schema_v1_01_xsd_Element_cdl_SOPNode.png}
        {\texttt{cdl:SOPNodeType}}
        {Choice of \texttt{aces:transformId, cdl:ColorCorrectionRef}, or \texttt{cdl:SOPNode} and \\ \texttt{cdl:SatNode} required}
        {1}{1}
        {None}{None}
        {\texttt{aces:lookTransform}}
        {\texttt{cdl:Description, cdl:Offset, cdl:Power, cdl:Slope}}
        {\lstinline{<cdl:ASC_SOP>}\\
                    \lstinline{<cdl:Slope>2.0 2.0 2.0</cdl:Slope>}\\
                    \lstinline{<cdl:Offset>0.1 0.1 0.1</cdl:Offset>}\\
                    \lstinline{<cdl:Power>1 1 1</cdl:Power>}\\
                    \lstinline{</cdl:ASC_SOP>}}

% cdl:SATNode
\element{cdl:SATNode}
        {This element is imported from the ASC-CDL schema (\texttt{ASC-CDL\_schema\_v1.01.xsd}).  It defines a saturation node.  \texttt{<cdl:SATNode>} may be substituted with \texttt{<cdl:ASC\_SAT>}.  See the ASC-CDL documentation for more information on its usage.}
        {images/ASC-CDL_schema_v1_01_xsd_Element_cdl_SatNode.png}
        {\texttt{cdl:SATNodeType}}
        {Choice of \texttt{aces:transformId, cdl:ColorCorrectionRef}, or \texttt{cdl:SOPNode} and \\ \texttt{cdl:SatNode} required}
        {1}{1}
        {None}{None}
        {\texttt{aces:lookTransform}}
        {\texttt{cdl:Description, cdl:Saturation}}
        {\lstinline{<cdl:ASC_SAT>}\\
                    \lstinline{<cdl:Saturation>1.0</cdl:Saturation>}\\
                    \lstinline{</cdl:ASC_SAT>}}

% cdl:ColorCorrectionRef
\element{cdl:ColorCorrectionRef}
        {This element is imported from the ASC-CDL schema (\texttt{ASC-CDL\_schema\_v1.01.xsd}).  It defines a Color Correction Reference node for referencing ASC-CDL values that exist in transport containers other than the AMF.  \texttt{<cdl:ColorCorrectionRef>} may be substituted with \texttt{<cdl:ASC\_CC\_XML>}.  It is recommended the \texttt{cdl:InputDescription} and \texttt{cdl:ViewingDescription} nodes not be used as this information is included in other locations within the AMF.  See the ASC-CDL documentation for more information on the usage of this node.}
        {images/ASC-CDL_schema_v1_01_xsd_Element_cdl_ColorCorrectionRef.png}
        {\texttt{cdl:ColorCorrectionRefType}}
        {Choice of \texttt{aces:transformId, cdl:ColorCorrectionRef}, or \texttt{cdl:SOPNode} and \\ \texttt{cdl:SatNode} required}
        {1}{1}{\texttt{ref} (xs:anyURI)}{None}
        {\texttt{aces:lookTransform}}
        {\texttt{cdl:Description, cdl:InputDescription, cdl:ViewingDescription}}
        {\lstinline{<cdl:ColorCorrectionRef ref="file:///foo.edl>}\\
                    \lstinline{<cdl:Description>Technical Grade</cdl:Description>}\\
                    \lstinline{</cdl:ColorCorrectionRef>}}




\newpage
% This file contains the content for a main section
\regularsectionformat	% Change formatting to that of a main, numbered section
%% Modify below this line %%
\chapter{External References}

The ACESclip file may be externally referenced in an EDL file to assign different ACES pipeline configurations to different segments of a timeline.    Use the following comment field in an EDL for this purpose:

001 Clipname  V             %___________  __________ ___________ ___________

*ACES\_ClipXML:   myshow\_LMTnight\_A0001.xml

The file name convention is user-defined.

Applications importing an EDL with such a comment field should set its viewing pipeline based on the \textit{aces:Config} value for a particular marked section of the timeline.


\begin{appendices}
    \appendixchapter{S-2008-002 -- Academy Density Exchange Encoding (ADX) and the Spectral Responsivities Defining Academy Printing Density (APD)}{i}
\label{appendixA}

\vspace{24pt}
\begin{center}
\uppercase{Document begins on next page}
\end{center}
\end{appendices}

\end{document}