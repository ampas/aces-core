\documentclass[10pt]{academydoc}
\pagestyle{plain}

% Set Document Details
\doctype{tb} % spec, proc, tb (Specification, Procedure, Technical Bulletin)
\docname{Academy Color Encoding System (ACES) Clip-level Metadata File Format Definition and Usage}
\altdocname{ACES Clip-level Metadata File Format Definition and Usage}
% Sets the document name used in header - usually an abbreviated document title
\docnumber{TB-2014-009}
\committeename{Academy Color Encoding System (ACES) Project Committee}
\versionnumber{1.0.1}
\docdate{April 24, 2015}
\summary{
The ACES Clip-level Metadata File (``ACESclip'') is a `sidecar' XML file intended to assist in configuring ACES viewing pipelines and to enable portability of ACES transforms in production. This document specifies use cases for ACESclip files, application support requirements, and the data model and XML tags needed for implementation.
}

% Document Starts Here
\begin{document}

\maketitle

% This file contains the content for the Notices
\prelimsectionformat	% Change formatting to that of "Notices" section
\chapter{\uppercase{Notices}}
%% Modify below this line %%

\copyright\the\year{} Academy of Motion Picture Arts and Sciences (A.M.P.A.S.). All rights reserved. This document is provided to individuals and organizations for their own internal use, and may be copied or reproduced in its entirety for such use. This document may not be published, distributed, publicly displayed, or transmitted, in whole or in part, without the express written permission of the Academy.

The accuracy, completeness, adequacy, availability or currency of this document is not warranted or guaranteed. Use of information in this document is at your own risk. The Academy expressly disclaims all warranties, including the warranties of merchantability, fitness for a particular purpose and non-infringement.

Copies of this document may be obtained by contacting the Academy at councilinfo@oscars.org.

``Oscars,'' ``Academy Awards,'' and the Oscar statuette are registered trademarks, and the Oscar statuette a copyrighted property, of the Academy of Motion Picture Arts and Sciences.

% This paragraph is optional.  Comment out if you wish to remove it.
This document is distributed to interested parties for review and comment. A.M.P.A.S. reserves the right to change this document without notice, and readers are advised to check with the Council for the latest version of this document.

% This paragraph is optional.  Comment out if you wish to remove it.
The technology described in this document may be the subject of intellectual property rights (including patent, copyright, trademark or similar such rights) of A.M.P.A.S. or others. A.M.P.A.S. declares that it will not enforce any applicable intellectual property rights owned or controlled by it (other than A.M.P.A.S. trademarks) against any person or entity using the intellectual property to comply with this document.

% This paragraph is optional.  Comment out if you wish to remove it.
Attention is drawn to the possibility that some elements of the technology described in this document, or certain applications of the technology may be the subject of intellectual property rights other than those identified above. A.M.P.A.S. shall not be held responsible for identifying any or all such rights. Recipients of this document are invited to submit notification to A.M.P.A.S. of any such intellectual property of which they are aware.

\vspace{10pt}
These notices must be retained in any copies of any part of this document. \newpage
% This file contains the content for the Revision History and 
\prelimsectionformat	% Change formatting to that of "Notices" section
\chapter{Revision History}
%% Modify below this line %%

\begin{tabularx}{\linewidth}{|l|l|X|}
    \hline
    Version & Date & Description \\ \hline
    1.0     & 05/10/2013 & Initial Version      \\ \hline
    1.1     & 08/02/2013 & Modify ACESproxy to handle negative ACES values \\ \hline
    2.0     & 12/19/2014 & Modify ACESproxy primaries, constrain to legal range \\ \hline
    2.0.1   & 04/24/2015 & Formatting and typo fixes \\ \hline
            &      &             \\ \hline
\end{tabularx}

\vspace{0.25in} % <-- DO NOT REMOVE
\chapter{Related Academy Documents} % <-- DO NOT REMOVE
\begin{tabularx}{\linewidth}{|l|X|}
    \hline
    Document Name & Description \\ \hline
    S-2008-001  & Academy Color Encoding Specification (ACES) \\ \hline
    S-2014-003  & ACEScc -- A Logarithmic Encoding of ACES Data for use within Color Grading Systems \\ \hline
    S-2014-004  & ACEScg -- A Working Space for CGI Render and Compositing \\ \hline
    & \\ \hline
    & \\ \hline
\end{tabularx} \newpage

\tableofcontents \newpage

% This file contains the content for the Scope
\cleardoublepage
\numberedformat	
\chapter{Scope} 	% Do not modify section title
%% Modify below this line %%

This document specifies 10-bit and 12-bit integer encodings of ACES for use with imaging systems that produce look metadata such as ASC CDL, and with transport systems such as HD-SDI. The color encoding provided in this format represents ACES relative exposure values as RGB triplets in a logarithmic encoding, and does not define the interfaces or signals that may carry this encoding.
% This section contains the content for the References
\numberedformat
\chapter{References}
The following standards, specifications, articles, presentations, and texts are referenced in this text:
%% Modify below this line %%

Academy S-2013-001, ACESproxy -- An Integer Log Encoding of ACES Data

SMPTE ST 2065-1:2012, Academy Color Encoding Specification (ACES)

SMPTE RP 177:1993, Derivation of Basic Television Color Equations

% This section contains the content for the Terms and Definitions
\numberedformat
\chapter{Terms and Definitions}
The following terms and definitions are used in this document.
%% Modify below this line %%

\term{Academy Color Encoding Specification (ACES)}
RGB color encoding for exchange of image data that have not been color rendered, between and throughout production and postproduction, within the Academy Color Encoding System. ACES is specified in SMPTE ST 2065-1.

\term{American Society of Cinematographers Color Decision List (ASC CDL)}
A set of file formats for the exchange of basic primary color grading information between equipment and software from different manufacturers. ASC CDL provides for Slope, Offset and Power operations applied to each of the red, green and blue channels and for an overall Saturation operation affecting all three.


% This file contains the content for a main section
\regularsectionformat	% Change formatting to that of a main, numbered section
%% Modify below this line %%
\chapter{Use Cases}

Image sequences are formed at several stages of production:

\begin{itemize}
	\item On-set from digital motion picture cameras, on-set dailies systems and on-set look management systems
	\item From film scanners and telecines
	\item In visual effects and animation production
	\item In production and post-production from editorial and color correction systems	
\end{itemize}

ACES image sequences are collections of related image files that have been converted to the ACES Image Container format SMPTE ST 2065-4:2013  (a.k.a. ``OpenEXR'').

In the context of ACES color-managed clips, a sequence may be a single frame, a collection of sequential frames gathered in a directory, or a ``packaged'' set of frames gathered in a file such as an MXF file. Image files do not need to be encoded as ACES image sequences to be ACES color-managed; they may be camera-native file formats or other encodings if they have associated Input Device Transforms (IDTs) so they may be displayed using an ACES viewing pipeline.

\section{Correspondence of ACESclip Files with Camera Image Sequences}
Image file sequences generated by a digital motion picture camera and recorded by a digital recorder are generally written in one of two ways:

\begin{itemize}
	\item as a collection of individual image files to a file directory, generally one directory for each shot or take
	\item as packaged sequence files using wrappers such as MXF, with one or more sequence files per file directory
\end{itemize}

An ACESclip file is generated on-set for each collection of individual image files or packaged sequence. Each ACESclip file contains metadata that describes the essential ACES transforms required to properly configure the ACES viewing pipeline for the image files it references:

\begin{itemize}
	\item The IDT used to convert camera-native image files to ACES2065-1 encoding
	\item If a Look Management System was used, the ASC-CDL values used for that sequence and the ACES Output Transform used to view the referenced sequence
	\item The LMT or LMTs for that sequence, if used	
\end{itemize}

ACESclip files are located in the same file directory as the image file collections or sequences that they describe, and they are associated with image file collections or sequences via matched filenames, e.g., ACESclip.MySequence.xml is associated with MySequence001.DPX through MySequence.100.DPX, where the numbers 001 and 100 are the range of frame counts for a 100 frame sequence.

Multiple ACESclip files, image collections and sequences in a single directory are possible by using this associative file naming approach.

Recommendations on naming conventions are outside of the scope of this document.
ALE and EDL files generated on-set may reference ACESclip files as an additional method of association.

\section{Use of ACESclip File in Visual Effects and Animation}
ACESclip files for image sequences generated by using computerized tools are handled in the same manner as for sequences generated on-set: an ACESclip file is created for each image sequence and populated with the required metadata that describes how that sequence was viewed when it was created.  This enables transmission of viewing pipeline information to a subsequent artist or facility so the image sequence may be viewed correctly.

For delivery of ACES image sequences to visual effects and 3D conversion facilities, it is recommended that image sequences be split into individual shots, and that a single ACESclip be present for each shot.

\section{Use of ACESclip File in Post-production}
ACES image sequences that arrive at the DI suite with an ACESclip file have all of the information necessary for an ACES-compatible color correction system to automatically configure itself to correctly display the sequence.

\section{Use of ACESclip File in Editorial}
Individual ACESclip files may be referred to in an EDL note field to enable application of different LMTs to different parts of an edited sequence.  For this reason, it is possible that more than one ACESclip file may be in a directory.

\section{Use of ACESclip File for Production Color Management}
The color transforms created in a production may be transferred between users and departments using the ACESclip file together with LUTs in the CLF format, and/or with ASC CDL metadata.

\section{Use of ACESclip File for Clip and Archive Management}
ACESclip files that incorporate a ClipID to reference an image sequence are easily re-attached to their image files should they become separated (it is common for related files to become accidentally separated during production). Using the ClipID throughout production also provides additional and useful information to archivists about originating source media.
% This file contains the content for a main section
\regularsectionformat	% Change formatting to that of a main, numbered section
%% Modify below this line %%
\chapter{Application and Use of ACESclip Files}

\section{ACESclip Filename and Correspondence with Images}
Transforms are identified with the CTL reference transform filename as defined in the ACES System Versioning Specification. Linking of the metadata about a transform to an actual instance of a transform is supported using ACES TransformIDs and XML id attributes.

ACESclips are named using the format ``production\_naming\_convention.ACESclip.xml''.

The production naming convention may be used to associate an ACESclip file with an image sequence, but this document does not specify an exact file naming convention.

\section{Saving State of IDT Conversion and Initial Grade}
Applications record the IDT used for converting camera-native data to ACES encodings, and include any pre-grade ASC CDL values that were used for the image sequences referenced by an ACESclip file.

\section{Conversion of Camera Files Using IDTs and Pre-grades}
For images not yet in the ACES file format, applications use the metadata for the IDT and ASC CDL pre-grade to view images using the ACES viewing pipeline. For images already in the ACES file format, the IDT conversion may be ignored, and only the pre-grade is applied prior to the ACES viewing pipeline.  

\section{Default Configuration of ACES Viewing Pipeline}
ACES content must be viewed as intended at any stage of production. Specific viewing pipelines may require different elements, so the exact viewing configuration used by a user making creative decisions must be recorded prior to shipment to another user. The ACES image sequence shall be displayed in an application with this ``last used'' viewing configuration, but a user may override the configured settings.

The \textit{aces:Config} xml tag is used to set the viewing pipeline to match the viewing conditions recorded in the ACESclip file. The ODT for the current viewing display may be used instead of the \textit{aces:Config} ODT, but the user should be warned if they are not of the same class of display, e.g. Rec.709 used previously and an HDR display is the current display.

\section{User Management of the Viewing Pipeline}
Users may override an application’s ACES viewing pipeline at any time.  Applications must manage the conversion between various ACES-compatible images and the user-selected working spaces.

When ASC CDL metadata is used, conversions to and from the ACEScc working space are saved in the \textit{aces:Config} metadata (wherein those particular CDL values must be applied).

\section{Saving the State of the ACES Viewing Pipeline}
The last state of the ACES Viewing Pipeline used to view an image sequence referenced by an ACESclip file must be recorded in the ACESclip file when a clip is closed or exported unless the user overrides this and does not want the clip to be changed.

\section{Creating an Archive Metadata Link}
When an ACES image sequence is created, placing identification traceable to source media in the \textit{aces:clipID} field is recommended.

\section{Reading and Writing LMTs}
The \textit{aces:TransformLibrary} XML element is used to transfer the actual transforms for LMTs to other users and facilities since these often may be custom LUTs. Applications shall read and write the XML structures containing CLF files.  An LMT combined with an RRT and ODT can be provided as well as a transform that simply contains the LMT. A stand-alone LMT must be merged with the other transforms in the basic ACES viewing pipeline for a user to look at the image.

\section{Reading and Writing ACESclip Files}
Applications shall support reading and writing of all XML elements described in this document. Recognition of extensions to the ACESclip specification developed by third parties is optional. However, if extensions are present, applications shall preserve them without change.

The ACESclip file may contain any one or all of the top-level XML structures (\textit{aces:clipID}, \textit{aces:Config}, \textit{aces:TransformLibrary}). For any particular XML file, these are listed as optional. However, production requirements determine which structures must be present in an ACESclip file.
% This file contains the content for a main section
\regularsectionformat	% Change formatting to that of a main, numbered section
%% Modify below this line %%

% Define a local macro - FieldName, description, comment
\newcommand{\xmlfield}[3]{
	\TabPositions{2em,1.75in,2.5in,2.6in}
	\tab\texttt{#1} \tab#2 \tab// #3	 \par
}


\chapter{Data Model}

This section describes the data intended for use within the ACES Clip-level Metadata file.

	\tabto{0.3in}\{string\} are XML attributes

All top level structures shall be tagged as being within the "aces" namespace.

The format of the data in this section represents pseudo-code rather than the XML schema. Indentation of the Tags indicates they are sub-elements of the XML structure just above in indentation.

\section{Header Info}
\texttt{<ACESmetadata} \{xmlns\} \texttt{>} \tabto{2.5in}// root xml structure for file and namespace

	\tabto{2em}attribute \hspace{2em} xmlns \hspace{2em} ACES namespace ``xmlns:aces=/oscars.org/aces/ref/acesmetadata''

\section{ACESclip File Information}
This section defines the file as an asset and shall be present as the first entries in the file.

	\TabPositions{2em,2.2in,4in}
	\tab\texttt{ContainerFormatVersion}\tab(minOccurs=1 maxOccurs=1)\tab// AMPAS version of container spec

	\tab\texttt{ModificationTime}\tab(minOccurs=1 maxOccurs=1)\tab// the last time this file was changed

	\tab\texttt{UUID} \{type\}\tab(minOccurs=0 maxOccurs=1)\tab// a unique ID for asset manager uses

\section{Clip Identification -- Archival Information}
This section provides an archival identification for the first creation and use of a clip in ACES.

The ClipID is intended to point to a specific image sequence as a reference but does not maintain timeline information (e.g., MarkIn, MarkOut reference points) as this is the role of EDLs and ALE files. The ClipID is not required to maintain clip edits and conversion history, but it is possible to have several ClipIDs with the oldest date being the creation clip and the newest date being the current version. This information also associates this particular ACESclip file with a clip sequence for re-association if required. Because of the variability of editor's organization and naming schemes, the contents of the strings for ClipName and Source\_MediaID are not specified, but are assumed to be used in a consistent manner within a production. Absence of a ClipID indicates that the transform state in \textit{aces:Config} is applicable to all of the image files in a directory.

\texttt{<aces:ClipID>}\tabto{1.5in}(minOccurs=0 maxOccurs=unbounded)

    \xmlfield{ClipName}{string}{clip name - production dependent}
    \xmlfield{Source\_MediaID}{string}{ID of source directory or media}
    \xmlfield{ClipDate}{DateTime}{date of clip creation (allows for fixes to a \\ 
    	\tab\tab\tab\tab\tab sequence to be timestamped)}
    \xmlfield{Note}{string}{note field - general purpose}    

\texttt{</aces:ClipID>}

\section{ACES Transform Pipeline Configuration Information}
Essential metadata for ACES pipeline configuration in support of the above use cases must be captured.

In this section, the attribute \textbf{\{TransformID\}} is an identifier for the reference definition for a particular transform according to the ACES versioning system (see Academy S-2014-002). A ``functionally equivalent'' transform that implements a named, particular transform must exist.\footnote{For standard ACES transforms, appending the ``.ctl'' suffix to the TransformID will create a filename that has the transform in the ACES CTL distribution.}

The LinkTransform fields identify a functional transform (most likely a LUT) that implements the referenced transform.  The attribute \textbf{\{id\}} is the name of the file or XML structure containing the linked transform. These links may be to a transform provided in the standard ACES release, or they may be custom versions that have a unique name. Organization of the location of transforms is outside of the scope of this document.

The attribute \textbf{\{status\}} is set to `preview' unless the transform has already been applied to the image data in which case status shall be set to `applied'. If the status is `applied', this transform shall be disregarded for the current clip.

The working space is assumed to be ACES except for the GradeRef where a small set of ACES transforms may be used to change the working space before application of the ASC CDL. LMTs may have their own working space defined in the ProcessList.

\texttt{<aces:Config>}\tabto{1.5in}(minOccurs=1 maxOccurs=1)

	\tabto{2em}\texttt{TimeStamp}\tabto{3.5in}// DateTime type: any change in this section\\\tabto{3.6in}shall be timestamped \par
	\tabto{2em}\texttt{ACESrelease\_Version}\{version\} \par
	\tabto{2em}\texttt{aces:InputTransformList}\tabto{3in}(minOccurs=0 maxOccurs=1)\tabto{3.5in}// conversion into aces \par
	\tabto{4em}\texttt{aces:IDTref} \{TransformID, status\} \par
	\tabto{6em}\texttt{LinkTransform} \{id\} \tabto{3.5in}// pointer to ProcessList of just the IDT\par
	\tabto{4em}\texttt{aces:GradeRef} \{status\} \par
	\tabto{6em}\texttt{Convert\_to\_WorkSpace} \{TransformID\} \par
	\tabto{6em}\texttt{ColorDecisionList} \{xmnls=ASC\} \tabto{3.5in}// portable grade in ASC CDL\\\tabto{3.6in}XML format v2.1\par
	\tabto{6em}\texttt{Convert\_from\_WorkSpace} \{TransformID\} \par
	\tabto{4em}\texttt{LinkInputTransformList} \{id\} \tabto{3.5in}// pointer to combined ProcessList \par
	\tabto{2em}\texttt{/aces:InputTransformList} \par
	\tabto{2em}\texttt{aces:PreviewTransformList} \tabto{3in}(minOccurs=1 maxOccurs=1)\par
	\tabto{4em}\texttt{aces:LMTref} \{TransformID, status\} \tabto{3in}(minOccurs=0 maxOccurs=3)\par
	\tabto{6em}\texttt{LinkTransform} \{id\} \tabto{3.5in}// pointer to ProcessList of just the LMT\par
	\tabto{4em}\texttt{aces:RRTref} \{TransformID\} \tabto{3in}(minOccurs=1 maxOccurs=1)\tabto{3.5in}// consistent with the ACESrelease\_Version\par
	\tabto{4em}\texttt{aces:ODTref} \{TransformID\} \tabto{3in}(minOccurs=1 maxOccurs=1)\par
	\tabto{6em}\texttt{LinkTransform} \{id\} \tabto{3.5in}// pointer to ProcessList containing above \\\tabto{3.6in}RRT+ODT\par
	\tabto{4em}\texttt{LinkPreviewTransform} \{id\} \tabto{3.5in}// pointer to overall combination ProcessList \\\tabto{3.6in}of all of the above LMT+RRT+ODT\par
	\tabto{2em}\texttt{/aces:PreviewTransformList} \par

\texttt{</aces:Config>}

\section{Info Block Information}
This section is for optional user-defined metadata.  Customized information may be included but other applications are not required to read or use these fields. It is required only that the fields be preserved when the ACESclip file is read and then rewritten. 

\texttt{<aces:Info>}\tabto{1.5in}(minOccurs=0 maxOccurs=1)

	\tabto{4em}\texttt{Application} \{version\} \tabto{2.25in}(recommended) \tabto{3.25in}// identify last software that modified the \\\tabto{3.35in}clip container\par
	\tabto{4em}\texttt{Comment} \tabto{2.25in}(optional) \par
	\tabto{4em}[Extension Fields]
	
\texttt{</aces:Info>}

\section{Carriage of XML-based Color Transform Files}
The Transform Library is intended to provide a portable mechanism for transforms, particularly those listed in the aces:Config.  Transforms may take the form of a CDL or a Common LUT Format ProcessList. (A CTL implementation is provided for reference, but production support of CTL files is not currently required).

Each transform in this section is required to have an {id} attribute for reference from the ACESconfig.

The Transform Library may be present in the ACESclip file or it may be a stand-alone file.

If transforms are not present in the file, they must be made available through some other means such as a Production Library distribution.

\texttt{<aces:Transform\_Library>}\tabto{2in}(minOccurs=0 maxOccurs=1)

	\tabto{4em}\texttt{<ColorDecisionList} \{id\}\texttt{>} \tabto{3in}(minOccurs=0 maxOccurs=unbounded)\par
	\tabto{4em}\texttt{<ProcessList} \{id\}\texttt{>} \tabto{3in}(minOccurs=0 maxOccurs=unbounded)\par
	\tabto{4em}\texttt{<CTL file} \{name, id\}\texttt{>} \tabto{3in}(minOccurs=0 maxOccurs=unbounded) \\\tabto{3in}// multiple -- CDATA wrapper. CTL support \\\tabto{3.1in}not required\par
    \tabto{6em}\texttt{<!}[CDATA[CTL FILE ----\\
    \tabto{6.1em}***** include .ctl file *****\\
    \tabto{6em}]]\texttt{>}\par
	\tabto{4em}\texttt{</CTL file} \{name\}
	
\texttt{</aces:Transform\_Library>}
\newpage
% This file contains the content for a main section
\regularsectionformat	% Change formatting to that of a main, numbered section
%% Modify below this line %%
\chapter{External References}

The ACESclip file may be externally referenced in an EDL file to assign different ACES pipeline configurations to different segments of a timeline.    Use the following comment field in an EDL for this purpose:

001 Clipname  V             %___________  __________ ___________ ___________

*ACES\_ClipXML:   myshow\_LMTnight\_A0001.xml

The file name convention is user-defined.

Applications importing an EDL with such a comment field should set its viewing pipeline based on the \textit{aces:Config} value for a particular marked section of the timeline.


\begin{appendices}
    \appendixchapter{Example ACESclip File XML}{i}
\label{appendixA}

\begin{lstlisting}
<?xml version="1.0" encoding="UTF-8"?>
<aces:ACESmetadata xmlns:aces="http://www.oscars.org/aces/ref/acesmetadata">
<ContainerFormatVersion>1.0</ContainerFormatVersion>
<UUID type="urn:uuid:f81d4fae-7dec-11d0-a765-00a0c91e6bf6"/>
<ModificationTime>2014-11-24T10:20:13-8:00</ModificationTime>
<aces:Info>
    <Application version="2014">Vendor_ACESmaker v3</Application>
    <Comment>Day 1 Camera A: Indoors at Area 51</Comment>
</aces:Info>
<aces:ClipID>
    <ClipName>A0001B0003FX</ClipName>
    <Source_MediaID>A0001</Source_MediaID>
    <ClipDate>2014-11-20T12:24:13-8:00</ClipDate>
</aces:ClipID>
<aces:Config>
    <ACESrelease_Version>1.0</ACESrelease_Version>
    <Timestamp>2014-11-29T23:55:13-8:00</Timestamp>
    <aces:InputTransformList status="applied">
        <aces:IDTref TransformID="IDT.Sony.Slog2.v1">
            <LinkTransform>Slog2toACES.clf</LinkTransform>
        </aces:IDTref>
        <aces:GradeRef status="preview">
            <Convert_to_WorkSpace TransformID="ACEScsc.ACES_to_ACEScc.a1"/>
            <ColorDecisionList>
                <InputDescriptor>ACEScc</InputDescriptor>
                <ASC_CDL id="cc01234" inBitDepth="16f" outBitDepth="16f">
                    <SOPNode>
                        <Description>not a real example</Description>
                        <Slope>1.0000 1.0000 0.800000</Slope>
                        <Offset>0.0000 0.0000 0.000000</Offset>
                        <Power>1.0000 1.0000 1.000000</Power>
                    </SOPNode>
                    <SatNode>
                        <Saturation>0.9</Saturation>
                    </SatNode>
                </ASC_CDL>
            </ColorDecisionList>
            <Convert_from_WorkSpace { TransformID="ACEScc_to_ACES.a1"/>
        </aces:GradeRef>
        <LinkInputTransformList>F65presetgrade_seq4.clf</LinkInputTransformList>
    </aces:InputTransformList>
    <aces:PreviewTransformList status="preview">
        <aces:LMTref TransformID="LMT.Vendor.Locon.v1.ctl">
            <LinkTransform>f34d4fae-7dec-11d0-a765-00a0c91e6bf6</LinkTransform>
        </aces:LMTref>
            <aces:RRTref TransformID="RRT.Academy.a1"/>
        <aces:ODTref TransformID="ODT.Academy.Rec709_100nits_dim.a1">
            <LinkTransform>lut1023.clf</LinkTransform>
        </aces:ODTref>
        <LinkPreviewTransformList>V1_LMTlocon.clf</LinkPreviewTransformList>
    <aces:PreviewTransformList>
</aces:Config>

<aces:TransformLibrary>
    <ProcessList id="urn:uuid:f34d4fae-7dec-11d0-a765-00a0c91e6bf6">
        <Description>Sample 3DLUT for an LMT</Description>
        <InputDescriptor>ACES</InputDescriptor>
        <OutputDescriptor>ACES</OutputDescriptor>
        <LUT3D id="lmt_prodv2" name="LMT sequence 1 day exterior" 
                interpolation="tetrahedral" inBitDepth="16f" outBitDepth="12i">
            <Description>LMT Test File</Description>
            <Array dim="33 33 33 3">
                0    0    0
                1    1    1
                [...data omitted]
            </Array>
        </LUT3D>
    </ProcessList>
</aces:TransformLibrary>
</aces:ACESmetadata>
\end{lstlisting}

\end{appendices}

\end{document}