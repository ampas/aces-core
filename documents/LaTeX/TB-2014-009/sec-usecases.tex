% This file contains the content for a main section
\regularsectionformat	% Change formatting to that of a main, numbered section
%% Modify below this line %%
\chapter{Use Cases}

Image sequences are formed at several stages of production:

\begin{itemize}
	\item On-set from digital motion picture cameras, on-set dailies systems and on-set look management systems
	\item From film scanners and telecines
	\item In visual effects and animation production
	\item In production and post-production from editorial and color correction systems	
\end{itemize}

ACES image sequences are collections of related image files that have been converted to the ACES Image Container format SMPTE ST 2065-4:2013  (a.k.a. ``OpenEXR'').

In the context of ACES color-managed clips, a sequence may be a single frame, a collection of sequential frames gathered in a directory, or a ``packaged'' set of frames gathered in a file such as an MXF file. Image files do not need to be encoded as ACES image sequences to be ACES color-managed; they may be camera-native file formats or other encodings if they have associated Input Device Transforms (IDTs) so they may be displayed using an ACES viewing pipeline.

\section{Correspondence of ACESclip Files with Camera Image Sequences}
Image file sequences generated by a digital motion picture camera and recorded by a digital recorder are generally written in one of two ways:

\begin{itemize}
	\item as a collection of individual image files to a file directory, generally one directory for each shot or take
	\item as packaged sequence files using wrappers such as MXF, with one or more sequence files per file directory
\end{itemize}

An ACESclip file is generated on-set for each collection of individual image files or packaged sequence. Each ACESclip file contains metadata that describes the essential ACES transforms required to properly configure the ACES viewing pipeline for the image files it references:

\begin{itemize}
	\item The IDT used to convert camera-native image files to ACES2065-1 encoding
	\item If a Look Management System was used, the ASC-CDL values used for that sequence and the ACES Output Transform used to view the referenced sequence
	\item The LMT or LMTs for that sequence, if used	
\end{itemize}

ACESclip files are located in the same file directory as the image file collections or sequences that they describe, and they are associated with image file collections or sequences via matched filenames, e.g., ACESclip.MySequence.xml is associated with MySequence001.DPX through MySequence.100.DPX, where the numbers 001 and 100 are the range of frame counts for a 100 frame sequence.

Multiple ACESclip files, image collections and sequences in a single directory are possible by using this associative file naming approach.

Recommendations on naming conventions are outside of the scope of this document.
ALE and EDL files generated on-set may reference ACESclip files as an additional method of association.

\section{Use of ACESclip File in Visual Effects and Animation}
ACESclip files for image sequences generated by using computerized tools are handled in the same manner as for sequences generated on-set: an ACESclip file is created for each image sequence and populated with the required metadata that describes how that sequence was viewed when it was created.  This enables transmission of viewing pipeline information to a subsequent artist or facility so the image sequence may be viewed correctly.

For delivery of ACES image sequences to visual effects and 3D conversion facilities, it is recommended that image sequences be split into individual shots, and that a single ACESclip be present for each shot.

\section{Use of ACESclip File in Post-production}
ACES image sequences that arrive at the DI suite with an ACESclip file have all of the information necessary for an ACES-compatible color correction system to automatically configure itself to correctly display the sequence.

\section{Use of ACESclip File in Editorial}
Individual ACESclip files may be referred to in an EDL note field to enable application of different LMTs to different parts of an edited sequence.  For this reason, it is possible that more than one ACESclip file may be in a directory.

\section{Use of ACESclip File for Production Color Management}
The color transforms created in a production may be transferred between users and departments using the ACESclip file together with LUTs in the CLF format, and/or with ASC CDL metadata.

\section{Use of ACESclip File for Clip and Archive Management}
ACESclip files that incorporate a ClipID to reference an image sequence are easily re-attached to their image files should they become separated (it is common for related files to become accidentally separated during production). Using the ClipID throughout production also provides additional and useful information to archivists about originating source media.